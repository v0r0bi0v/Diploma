\documentclass[12pt,a4paper]{article}
\usepackage[utf8x]{inputenc}
\usepackage[english, russian]{babel}
\usepackage{indentfirst}
\usepackage{xcolor}
\usepackage{soul}
\usepackage{amsmath}
\usepackage{amsthm}
\usepackage{amssymb}
\usepackage{amscd}
\usepackage{bm}
\usepackage{mathrsfs}
\usepackage{mathtools}
\usepackage{tikz-cd}
\usepackage{hyperref}
\usepackage{floatrow,graphicx,calc}
\usepackage{ragged2e}
\usepackage[backend=biber]{biblatex}\addbibresource{diploma.bib}

\newtheorem{definition}{Определение}[subsection]
\newtheorem{remark}{Замечание}[subsection]
\newtheorem{lemma}{Лемма}[subsection]
\newtheorem{proposition}{Предложение}[subsection]
\newtheorem{theorem}{Теорема}[subsection]
\newtheorem{colloraly}{Следствие}[subsection]
\newtheorem{conjecture}{Гипотеза}[subsection]

%\newtheorem*{theorem*}{Теорема}[subsection]

\newcommand{\R}{\ensuremath{\mathbb{R}}}
\newcommand{\Z}{\ensuremath{\mathbb{Z}}}
\newcommand{\Q}{\ensuremath{\mathbb{Q}}}
\newcommand{\LQ}{\ensuremath{\mathcal{L}_{\mathbb{Q}_p}}}
\newcommand{\LZ}{\ensuremath{\mathcal{L}_{\mathbb{Z}_p}}}
\newcommand{\LQn}{\ensuremath{\mathcal{L}^{(n)}_{\mathbb{Q}_p}}}
\newcommand{\LZn}{\ensuremath{\mathcal{L}^{(n)}_{\mathbb{Z}_p}}}
\newcommand{\Sbf}{\ensuremath{\mathbf{S}}}
\newcommand{\rank}{\ensuremath{\mathrm{rank}}}
\newcommand{\xoo}{\ensuremath{x_{1,1}}}
\newcommand{\xod}{\ensuremath{x_{1,2}}}
\newcommand{\xdo}{\ensuremath{x_{2,1}}}
\newcommand{\xdd}{\ensuremath{x_{2,2}}}
\newcommand{\yoo}{\ensuremath{y_{1,1}}}
\newcommand{\yod}{\ensuremath{y_{1,2}}}
\newcommand{\ydo}{\ensuremath{y_{2,1}}}
\newcommand{\ydd}{\ensuremath{y_{2,2}}}





\setlength{\topsep}{0pt}\setlength{\parindent}{0pt}
\renewcommand{\arraystretch}{1.3}



\begin{document}

%%
%% ===========================================================================
%%
    \thispagestyle{empty}

    \begin{adjustwidth}
        \center{\fontsize{16pt}{18.0pt}\selectfont Федеральное государственное автономное\\ образовательное учреждение высшего образования\\ «Национальный исследовательский университет\\ «Высшая школа экономики»\par}\par
    \end{adjustwidth}

    \vspace{\baselineskip}
    \begin{adjustwidth}
        \center{\fontsize{15pt}{18.0pt}\selectfont Факультет математики\par}\par
    \end{adjustwidth}


    \vspace{5\baselineskip}
    \begin{adjustwidth}
        \begin{FlushLeft}
            \center{\fontsize{13.5pt}{15.6pt}\selectfont \textbf{Воробьев Иван Евгеньевич}\par}
        \end{FlushLeft}\par

    \end{adjustwidth}


    \vspace{3\baselineskip}
    \begin{adjustwidth}
        \center{\fontsize{16pt}{18.0pt}\selectfont \textbf{Проблема Шпехта и гипотеза Гельфанда}\par}\par
    \end{adjustwidth}


    \vspace{2\baselineskip}
    \begin{adjustwidth}
        \begin{FlushLeft}
            \center{\fontsize{14.5pt}{16.8pt}\selectfont Выпускная квалификационная работа студента 4 курса\\ образовательной программы бакалавриата «Математика»\par}
        \end{FlushLeft}\par

    \end{adjustwidth}


    \vspace{4.5\baselineskip}
    \begin{figure}[h!]
        \begin{minipage}[h]{0.5\linewidth}
        \end{minipage}
        \hfill
        \begin{minipage}[h]{0.5\linewidth}
            \begin{FlushLeft}
            {\fontsize{12pt}{16.8pt}\selectfont Научный руководитель:\\ Доктор физико-математических наук, профессор\\ Канель-Белов Алексей Яковлевич\\ \vspace{1\baselineskip}
            \fontsize{12pt}{16.8pt}\selectfont Научный соруководитель:\\ Кандидат физико-математических наук,\\ Хорошкин Антон Сергеевич\\ \par}
            \end{FlushLeft}\par
        \end{minipage}
    \end{figure}

    \vspace{1.0\baselineskip}
    \begin{adjustwidth}
        \begin{Center}
            \center{\fontsize{14.5pt}{16.8pt}\selectfont Москва 2024\par}
        \end{Center}\par

    \end{adjustwidth}
    \pagebreak
%%
%% ===========================================================================
%%


    \begin{abstract}
        Пусть $F$ свободная некоммутативная про-$p$ группа, и пусть $\Delta$ коммутативное нетерово полное локальное кольцо с максимальным идеалом $I$, такое что
        $\Delta/I$ конечное поле характиристики $p$\\
        Определим группу
        \[GL_d^1(\Delta) = \mathrm{ker}\left( GL_d(\Delta) \xrightarrow{\Delta\to\Delta/I} GL_2(\Delta/I) \right)\]
        А.Н.\ Зубков доказал, что $F$ не может быть непрерывно вложена в $GL_2^1(\Delta)$ для $p\neq2$.\\
        Д.\ Бен-Эзра и Е.\ Зельманов, показали, что и для $p=2$, $\mathrm{char}(\Delta)=2$ имеет место такой же результат.\\
        Цель данной статьи обобщить подход для $p=2$ и $\mathrm{char}(\Delta)=4$.\\
        Кроме того, Зельманов показал в ~\cite{Zelmanov1}, что гипотеза о нелинейности про-$p$ групп тесно связана с PI-теорией.

        Во второй части данной статьи мы изучаем связь между PI-теорией и гипотизей Гельфанда о конечномерности гомологий алгебр Ли векторных полей.

        Таким образом, можно видеть, что работа в основном посвящена изучению комбинаторики подстановок.
    \end{abstract}
    \tableofcontents


    \section{Введение}
    Мы рассмотрим применения теории полиномиальных тождеств (PI-теории для краткости) в двух, казалось бы, несвязанных с ней областях:
    \begin{itemize}
        \item В 2005, Е.\ Зельманов представил набросок доказательства нелинейности некоторых свободных про-$p$ групп.
        Доказательство во многом опирается на стандартные подходы PI-теории (см.\ ~\cite{Zelmanov1}).
        \item Дополнительно мы рассмотрим неожиданную связь между гипотезой Гельфанда и PI-теорией, а именно с методами Гришина (\cite{Grishin}).
    \end{itemize}

    Приведем краткую историческую справку и формулировку основных проблем.

    \subsection{О нелинейности свободных про-$p$-групп}
    Проблема линейности топологических групп изучалась много лет.
    Одной из естественных топологий, наряду с $\mathbb{R}^n$ и дискретной, является про-$p$ топология.

    Сформулируем центральную гипотезу данной теории, а далее приведем все необходимые определения:

    \vskip 0.1in\noindent
    \begin{conjecture}
        Некоммутативная свободная про-$p$ группа не может быть вложена в $GL_d(\Delta)$ ни для какого про-$p$ кольца $\Delta$.
    \end{conjecture}
    \vskip 0.1in\noindent

    \vskip 0.1in\noindent
    \begin{definition}
        Обратный (проективный) предел конечных $p$-групп называется про-$p$ группой.
    \end{definition}
    \vskip 0.1in\noindent
    Топология индуцируется с топологии тихоновского произведения.

    \vskip 0.1in\noindent
    \begin{definition}
        Коммутативное нетерово полное локальное кольцо $\Delta$ с максимальным идеалом $I$ называется про-$p$ кольцом, если $\Delta/I$ конечное поле характеристики $p$.
    \end{definition}
    \vskip 0.1in\noindent

    Таким образом:
    \[
        \Delta \cong \varprojlim \Delta/I^n
    \]

    Рассмотрим конгуренц-подгруппу:
    \[
        GL_d^1(\Delta) = \mathrm{ker}\left( GL_d(\Delta) \xrightarrow{\Delta\to\Delta/I} GL_d(\Delta/I) \right)
    \]
    Можно видеть, что это про-$p$ группа.

    Основной вопрос состоит в том, может ли свободная про-$p$ группа быть непрерывна вложена в $GL_d^1(\Delta)$.

    Свободную про-$p$ группу можно определить классическим способом через универсальное свойство или конструктивно:

    \vskip 0.1in\noindent
    \begin{definition}
        Свободная про-$p$ группа $F_p(X)$ является пополнением дискретной свободной группы $F(X)$ относительно топологии всех нормальных подгрупп индекса степени $p$.
    \end{definition}
    \vskip 0.1in\noindent

    Существует множество частичных результатов:
    \begin{itemize}
        \item В 1987, А.Н.\ Зубков (\cite{Zubkov}) доказал гипотезу для $d=2, p\neq2$.
        \item В 1999, используя глубокие результаты Пинка (\cite{Pink}), Й.\ Барнеа, М.\ Ларсен (\cite{Barnea-Larsen}) подтвердили гипотезу для $\Delta=\left( \mathbb{Z}/p\mathbb{Z} \right)[[t]]$
        \item В 1991, Д.\ Диксон, А.\ Манн, М.П.Ф.\ Ду Сатой, Д.\ Сигал (\cite{DMSD}) доказали гипотезу для всех размеров матриц для целых $p$-адических чисел $\Delta=\mathbb{Z}_p$, $GL_d^1(\mathbb{Z}_p)=\mathrm{ker}\left( GL_2(\mathbb{Z}_p) \xrightarrow{\mathbb{Z}_p\to\mathbb{F}_p} GL_2(\mathbb{F}_p) \right)$
        \item В 2005, E.\ Зельманов (\cite{Zelmanov1},~\cite{Zelmanov2}) анонсировал доказательство гипотезы для $p\gg d$, однако до сих пор существует только набросок доказательства.
        \item В 2020, Д.\ Бен-Эзра, Е.\ Зельманов (\cite{Ben-Ezra-Zelmanov}) обобщили результат Зубкова для $d=2, p=2$ и $\mathrm{char}(\Delta)=2$.
    \end{itemize}

    Мы сконцентрируемся на случае $2\times 2$ матриц, то есть на результатах Зубкова и Бен-Эзры-Зельманова.

    Сейчас опишем общий план доказательства, он отчасти реализован для произвольных размеров матриц в\ \cite{Zelmanov1},\ \cite{Zelmanov2}:

    Для начала сузим множество рассматриваемых колец.
    Можно построить, так называемое, универсальное представление в общие матрицы.
    Оказывается, что классическими алгебраическими рассуждениями можно показать, что если какое-то представление точно, то и это универсальное представление точно.

    Таким образом, остается исследовать это универсальное представление.
    Это делается путем изучения алгебры Ли общих матриц, которая множество раз возникала в PI-теории.
    В завершение строится связь между этой алгеброй Ли и образом нашего представления ~--- подобно связи между группой Ли и алгеброй Ли.

    \subsection{Гипотеза Гельфанда}
    В 2022 была обнаружена замечательная связь между PI-теорией (точнее методами Гришина) и гипотезой Гельфанда сформулированной на ICM’70 (см\ \cite{Gelfand}).
    \vskip 0.1in\noindent
    \begin{conjecture}[Гельфанд, 1970]
        Гомологии подалгебры Ли конечной коразмерности алгебры Ли алгебраических векторных полей на афинном алгебраическом многообразии конечномерны.
    \end{conjecture}
    \vskip 0.1in\noindent
    Эта интересная связь была найдена в результате совместной работы А.С.\ Хорошкина, А.Я. Канель-Белова с некоторым участием автора.
    Можно найти набросок доказательства Хорошкина в ~\cite{Feigin-Kanel-Khoroshkin},~\cite{Centrone-Kanel-Khoroshkin-Vorobiov}.

    Дополнительно, мы заметим, что именно результат из последней курсовой работы автора (частный случай методов Гришина\ \cite{Grishin}) помогает в доказательстве гипотезы Гельфанда.


    \section{О нелинейности свободных про-$p$ групп}
    Для начала необходимо привести полные доказательства Зубкова и Бен-Эзры-Зельманова.

    \subsection{Подход Зубкова}
    \begin{theorem}[Зубков, 1989]\label{thm:Zubkov-main}
        Некоммутативная свободная про-$p$ группа не может быть непрерывно вложена в $GL^1_2(\Delta)$ для $p\neq 2$.
    \end{theorem}

    \subsubsection{Универсальное представление}
    Определим алгебру общих матриц над $p$-адическими числами.

    Мы будем работать с матрицами $2\times2$, однако все результаты и конструкции этого параграфа дословно переносятся на матрицы произвольного размера.

    Рассмотрим формальные степенные ряды от свободных коммутирующих переменных $x_{i,j}, y_{i,j}$ для $i,j \in \{ 1, 2 \}$:
    \[
        S = \Z_{p}\langle x_{1,1}, y_{1,1}, \ldots, x_{2,2}, y_{2,2} \rangle
    \]
    Введем стандартную градуировку $\deg$: любой элемент $S$ записывается в виде $\sum f_i$ где $\deg{(f_i)} = i$.

    Рассмотрим идеалы вида
    \[
        S_k = \{\sum\limits_k^{\infty} f_i \}
    \]

    \vskip 0.1in\noindent
    \begin{proposition}
        Следующие идеалы
        \[B_{k,n} = S \cdot p^n + S_k \]
        являются идеалами конечного индекса
    \end{proposition}
    \begin{proof}
        Достаточно доказать, что $\Z_p / (p^n)$ --- конечное кольцо.
        \[
            \ker{(\Z_p \to Z_p / (p^n) )} = (p_n)= \{ 0,\ldots, 0, a_{k+1}, a_{k+2}, \ldots \}
        \]
        Ясно, что множество целых $p$-адических чисел отличающихся на элементы такого вида конечно и
        $\Z_p / (p^n) \cong \Z / p^n\Z$
    \end{proof}

    Снабдим $S$ топологией с базой окрестностей нуля состоящей из идеалов $B_{k,n}$.
    $S$ является про-$p$ кольцом:
    \[
        S / B_{1,1} \cong \Z/p\Z
    \]

    Наконец, рассмотрим матричное кольцо $M_2(S)$.
    Наделив его топологией c базой окрестностей нуля конгруенц-идеалов
    \[
        \ker{(M_2(S) \to M_2(S / B_{k,n}))}
    \]
    получим, что $M_2(S)$ ~--- про-$p$ кольцо.
%    todo тут хотелось бы аккуратно все эти вещи доказать, почему это про-$p$ кольцо и почему сейчас получится даже про-$p$ группа.
    \vskip 0.1in\noindent
    \begin{proposition}
        Множество
        \[1 + \ker{(M_2(S) \to M_2(S / S_1))}\]
        является про-$p$ группой.
    \end{proposition}
    \begin{proof}
        Заметим, что $\ker{(S\to S/S_1)}$ состоит из рядов без свободного члена.
        Получается, что $1 + \ker{(M_2(S) \to M_2(S / S_1))}$ является группой, так как ряд обратим тогда и только тогда, когда его свободный член обратим.

        Также можно заметить, что эта группа полна относительно определенной выше топологии, то есть является про-$p$ группой.
    \end{proof}
    Рассмотрим общие матрицы $X, Y \in \ker{(M_2(S) \to M_2(S / S_1))}$:
    \[
        X=
        \begin{pmatrix}
            x_{1,1} & x_{1,2} \\
            x_{2,1} & x_{2,2}
        \end{pmatrix},
        \quad
        Y=
        \begin{pmatrix}
            y_{1,1} & y_{1,2} \\
            y_{2,1} & y_{2,2}
        \end{pmatrix}
    \]
    Пусть $F$ --- свободная про-$p$ группа порожденная $x, y$.

    Наконец определим универсальное представление:
    \[
        \pi:
        \left\{
        \begin{array}{l}
            x \mapsto 1 + X \\
            y \mapsto 1 + Y
        \end{array}
        \right.
    \]

    Продолжим его на дискретную подгруппу, порожденную $x, y$:
    \[
        \pi: \quad \langle x, y \rangle \to \langle 1+X, 1+Y \rangle \subseteq 1 + \ker{(M_2(S) \to M_2(S / S_1))}
    \]
    А затем можно непрерывно доопределить $\pi$ на всей $F$, построив замыкание $\langle 1+X, 1+Y \rangle$ в топологии $1 + \ker{(M_2(S) \to M_2(S / S_1))}$:
    \[
        \pi:\quad F \to G \subseteq 1 + \ker{(M_2(S) \to M_2(S / S_1))}
    \]

    Итак, следующая теорема позволяет нам изучать только универсальное представление, не задумываясь о других про-$p$ кольцах.

    \vskip 0.1in\noindent
    \begin{theorem}[Зубков, 1987]
        Пусть $F$ ~--- свободная про-$p$ группа порожденная $x, y$.
        Если существует инъективный непрерывный гомоморфизм $\varphi: F \to GL^1_2(\Delta)$, то и универсальное представление $\pi$ инъективно.
    \end{theorem}
    \begin{proof}
        Напомним
        \[
            GL^1_2(\Delta) = \ker{(GL(\Delta \to \Delta / I))}
        \]

        Рассмотрим образы $x, y$:
        \[
            \varphi(x) = 1 + A, \quad
            \varphi(y) = 1 + B
        \]
        Заметим, что
        \[
            A, B \in \ker{(M_2(\Delta)\to M_2(\delta/I))}
        \]
        так как $1 + A, 1 + B \in \ker{(GL(\Delta \to \Delta / I))}$.
        Пусть
        \[
            A=
            \begin{pmatrix}
                a_{1,1} & a_{1,2} \\
                a_{2,1} & a_{2,2}
            \end{pmatrix},
            \quad
            B=
            \begin{pmatrix}
                b_{1,1} & b_{1,2} \\
                b_{2,1} & b_{2,2}
            \end{pmatrix}
        \]
        Ясно, что $\lim a_{i,j}^n = \lim b_{i,j}^n = 0$.
        Тогда можно построить гомоморфизм $\zeta: x_{i,j} \mapsto a_{i,j}, y_{i,j} \mapsto b_{i,j}$, он индуцирует эпиморфизм
        $\hat{\zeta}: G \to \mathrm{Im}\hspace{0.1cm}{\varphi}$.
        Наконец, получаем коммутативную диаграмму
        \begin{center}
            \begin{tikzcd}
                & F \arrow{dl}{\pi} \arrow{dr}{\varphi} & \\
                G \arrow{rr}{\hat{\zeta}} & & GL_2^1(\Delta)
            \end{tikzcd}
        \end{center}
        Следовательно:
        \[
            \ker{\pi} \subseteq \ker{\varphi}
        \]

    \end{proof}

    \subsubsection{Неточность универсального представления}
    Введем обозначения:
    \begin{itemize}
        \item $\mathbf{S}$ ~--- кольцо степенных рядов от общих матриц $X, Y$ над $\Z_p$
        \item $\LQ$ ~--- алгебра Ли порожденная общими матрицами $X, Y$ над $\Q_p$
        \item $\LZ$ ~--- алгебра Ли порожденная общими матрицами $X, Y$ над $\Z_p$
        \item $\LQ^{(n)}$ ~--- векторное пространство над $\Q_p$ однородных элементов степени $n$ в алгебре $\LQ$.
        \item $\LZ^{(n)}$ ~---  $\Z_p$-модуль однородных элементов степени $n$ в алгебре $\LZ$.
        \item Для $g\in G$: $\min{g}$ ~--- однородная компонента наименьшей ненулевой степени (можно записать $g=1 + a_n + a_{n+1} + \ldots$)
        \item Будем записывать коммутатор веса $n$ следующим образом $[l_1, \ldots, l_n] = [[l_1, l_2, \ldots, l_{n-1}], l_n]$
    \end{itemize}
    Приведем сначала план доказательства, чтобы была понятна мотивация каждой леммы:
    \begin{enumerate}
        \item Определим $G\supseteq G_n$, вложенную последовательность нормальных подгрупп попадающую в любую окрестность единицы, такую что для любого $g\in G_n$
        \[
            \min{g} \in \LQn\cap\Sbf = \LZn
        \]
        \item Таким образом, можно изучать $G_n/G_{n+1}$ как $\Z_p$-модуль $\LZn$.
        Докажем, что $\rank_{\Z_p}{\LZn} = f(n)$ для некоторой $f$.
        \item Доказав, что $G_n$ ~--- нижний центральный ряд $G$ получим противоречие с формулой Витта (см. \cite{Lubotzky}):
        \begin{proposition}[Витт, \cite{Lubotzky}]\label{thm:Vitt}
            Пусть F ~--- свободная про-$p$ группа порожденная $m$ образующими, тогда $n$-ый фактор нижнего центрального ряда имеет ранг (как $\Z_p$-модуль)
            \[
                \frac{1}{n}\sum\limits_{d\mid n} \mu(d) m^{n / d}
            \]
            где $\mu$ ~--- функция Мебиуса.
        \end{proposition}
    \end{enumerate}
    Итак, докажем следующее техническое утверждение:
    \vskip 0.1in\noindent
    \begin{proposition}\label{thm:LQn-to-LZn}
        При $p\neq 2$:
        \[
            \LQn \cap \Sbf = \LZn
        \]
    \end{proposition}
    \begin{remark}
        Данное предложение не верно для $p=2$, что влечет существенные сложности, возникающие для $p=2$.
    \end{remark}
    \begin{proof}
        Пусть
        \begin{align*}
            & a = 4\xod\xdo + (\xoo - \xdd)^2 \\
            & b = 2(\yod\xdo + \xod\ydo) + (\xoo - \xdd)(\yoo-\ydd)\\
            & c = 4\yod\ydo + (\yoo - \ydd)^2
        \end{align*}
        Легко проверить, что
        \begin{align*}
            & [x,y,x,x] = a[x,y] \\
            & [x, y, y, x] = b[x,y] \\
            & [x,y,y,y] = c[x,y] \\
            & [x,y,x,y] = [x,y,y,x]
        \end{align*}
        Получаем, что любой $l\in \LQn$ имеет вид
        \begin{equation}
            l =
            \begin{cases}
                \sum\limits_{i_a + i_b + i_c = (n - 2) / 2} \lambda_{i_a,i_b,i_c} a^{i_a}b^{i_b}c^{i_c}[x,y], & \text{если $n$ четно} \\
                \sum\limits_{i_a + i_b + i_c = (n - 3) / 2} \alpha_{i_a,i_b,i_c} a^{i_a}b^{i_b}c^{i_c}[x,y,x] +
                \beta_{i_a,i_b,i_c} a^{i_a}b^{i_b}c^{i_c}[x,y,y], & \text{если $n$ нечетно}
            \end{cases}\\
            \label{eq:not-direct-sum}
        \end{equation}
        где $\lambda_{i_a,i_b,i_c},\alpha_{i_a,i_b,i_c},\beta_{i_a,i_b,i_c}\in\Q_p$

        Разберем случай нечетного $n$.
        Рассмотрим какое-то $l\in\LQn \cap \Sbf$ с нецелыми $p$-адическими $\alpha_{i_a,i_b,i_c},\beta_{i_a,i_b,i_c}$.

        Далее рассуждение аналогично классическому доказательству теоремы Гильберта о базисе.
        Введем лексикографический порядок на мономах порожденный отношением
        \[
            \xod>\xoo>\yod>\yoo>\xdo>\xdd>\ydo>\ydd
        \]
        Старшие члены у $a,b,c$: $4\xod\xdo, 2\xod\ydo, 4\yod\ydo$ соответственно.
        Тогда старший член элемента, стоящего в левом верхнем углу $l$ равен старшему члену выражения
        \[
            \sum\limits_{i_a + i_b + i_c = (n - 3) / 2}  2^{2i_a + 2i_c + i_b}
            \left(
            \alpha_{i_a,i_b,i_c}
            \xoo\yod^{i_c}\xdo^{i_a}\ydo^{i_b+i_c+1} +
            \beta_{i_a,i_b,i_c}
            \yod^{i_c}\yoo\xdo^{i_a}\ydo^{i_b+i_c+1}\right)
        \]
        Старшие члены различны и $2^{k}$ обратим в кольце $\Z_p$ (при $p\neq 2$!).

        Следовательно, так как $\alpha_{i_a,i_b,i_c}, \beta_{i_a,i_b,i_c} \in \Q_p\\ \Z_p$, а значит старший член левого верхнего угла $l$ не лежит в $\Z_p$.
        Получаем противоречие с тем, что $l\in \Sbf$.

        Случай четного $n$ разбирается аналогично.
        \begin{remark}
            Попутно мы доказали, что суммы в формуле~\eqref{eq:not-direct-sum} на самом деле прямые.
        \end{remark}
    \end{proof}

    Сразу же получаем следствие
    \begin{colloraly}\label{thm:LZn-rank}
        \[
            rank_{\Z_p} \LZn = \dim_{\Q_p} \LQn =
            \begin{cases}
                \frac{n(n+2)}{8}, & \text{при четном $n$} \\
                \frac{(n-1)(n+1)}{4}, & \text{при нечетном $n$}
            \end{cases}
        \]
    \end{colloraly}


    Итак, пусть
    \[
        G_n = G \cap \ker{(GL(S) \to GL(S / S_n))}
    \]

    Следующая лемма является ключевой.
    \begin{lemma}
        Пусть $g\in G_n$, тогда
        \[
            \min g \in \LZn
        \]
    \end{lemma}
    \begin{proof}
        В силу предложения~\ref{thm:LQn-to-LZn} достаточно доказать, что
        $\min g \in \LQn$.







    \end{proof}





        \printbibliography
    \end{document}
