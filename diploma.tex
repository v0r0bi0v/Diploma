%! suppress = EscapeAmpersand
%! suppress = EscapeUnderscore
\documentclass[12pt,a4paper]{article}
\usepackage[utf8x]{inputenc}
\usepackage[english, russian]{babel}
\usepackage{indentfirst}
\usepackage{xcolor}
\usepackage{soul}
\usepackage{amsmath}
\usepackage{amsthm}
\usepackage{amssymb}
\usepackage{amscd}
\usepackage{bm}
\usepackage{mathrsfs}
\usepackage{mathtools}
\usepackage{tikz-cd}
\usepackage{hyperref}
\usepackage{floatrow,graphicx,calc}
\usepackage{ragged2e}
\usepackage[style=numeric,backend=biber,,maxbibnames=99]{biblatex}\addbibresource{diploma.bib}
\renewcommand*{\finalnamedelim}{\addcomma\space}
\renewcommand*{\finalnamedelim}{\addcomma\space}

\newtheorem{definition}{Определение}[section]
\newtheorem{remark}{Замечание}[section]
\newtheorem{lemma}{Лемма}[section]
\newtheorem{proposition}{Предложение}[section]
\newtheorem{theorem}{Теорема}[section]
\newtheorem*{theorem*}{Теорема}
\newtheorem{colloraly}{Следствие}[section]
\newtheorem{conjecture}{Гипотеза}[section]


\newcommand{\R}{\ensuremath{\mathbb{R}}}
\newcommand{\Z}{\ensuremath{\mathbb{Z}}}
\newcommand{\Q}{\ensuremath{\mathbb{Q}}}
\newcommand{\LQ}{\ensuremath{\mathcal{L}_{\mathbb{Q}_p}}}
\newcommand{\LZ}{\ensuremath{\mathcal{L}_{\mathbb{Z}_p}}}
\newcommand{\LQn}{\ensuremath{\mathcal{L}^{(n)}_{\mathbb{Q}_p}}}
\newcommand{\LZn}{\ensuremath{\mathcal{L}^{(n)}_{\mathbb{Z}_p}}}
\newcommand{\LQcentralr}{\ensuremath{\mathcal{L}_{\mathbb{Q}_p, n}}}
\newcommand{\Sbf}{\ensuremath{\mathbf{S}}}
\newcommand{\rank}{\ensuremath{\mathrm{rank}}}
\newcommand{\xoo}{\ensuremath{x_{1,1}}}
\newcommand{\xod}{\ensuremath{x_{1,2}}}
\newcommand{\xdo}{\ensuremath{x_{2,1}}}
\newcommand{\xdd}{\ensuremath{x_{2,2}}}
\newcommand{\yoo}{\ensuremath{y_{1,1}}}
\newcommand{\yod}{\ensuremath{y_{1,2}}}
\newcommand{\ydo}{\ensuremath{y_{2,1}}}
\newcommand{\ydd}{\ensuremath{y_{2,2}}}
\newcommand{\Wpn}{\ensuremath{\mathcal{W}^{\mathrm{pol}}_n}}
\newcommand{\Wn}{\ensuremath{\mathcal{W}_n}}
\newcommand{\Wp}{\ensuremath{\mathcal{W}^{\mathrm{pol}}}}
\newcommand{\W}{\ensuremath{\mathcal{W}}}
\newcommand{\Lp}{\ensuremath{\mathcal{L}^{\mathrm{pol}}}}
\newcommand{\SQ}{\ensuremath{S_{\mathbb{Q}_p}}}
\newcommand{\SQn}{\ensuremath{S_{\mathbb{Q}_p, n}}}
\newcommand{\SQr}{\ensuremath{S_{\mathbb{Q}_p, r}}}
\newcommand{\SQo}{\ensuremath{S_{\mathbb{Q}_p, 1}}}

\renewcommand{\L}{\ensuremath{\mathcal{L}}}
\renewcommand{\char}{\ensuremath{\mathrm{char}}}


\renewcommand{\arraystretch}{1.3}

\setlength{\topsep}{0.01cm}
\setlength{\parindent}{0.7cm}

\begin{document}

    \begin{center}
{\scshape Федеральное государственное автономное\\
образовательное учреждение высшего образования\\
    <<Национальный исследовательский университет\\
    <<Высшая школа экономики>>\\[1ex]
Факультет математики\par}

    \par\vfill

    \textbf{\large Воробьев Иван Евгеньевич}

    \vspace{1.5cm}

    {\Large\bfseries
    Проблема Шпехта и гипотеза Гельфанда
    \par}

    \vspace{1.5cm}

    Выпускная квалификационная работа "--- бакалаврская работа\\[1ex]
    по направлению подготовки 01.03.01 "--- Математика,\\[1ex]
    образовательная программа <<Математика>>
    \par\vfill
    \noindent\parbox[t]{0.48\textwidth}{%
        Рецензент:\\[3pt]
        доктор физико-математических наук,\\
        профессор\\
        Горчинский Сергей Олегович
    }\hspace{0.04\textwidth}\parbox[t]{0.48\textwidth}{%
        Научный руководитель:\\[3pt]
        доктор физико-математических наук,\\
        профессор\\
        Канель-Белов Алексей Яковлевич\\[2ex]
        %% comment if needed
        Консультант:\\[3pt]
        Кандидат физико-математических наук, доцент\\
        Хорошкин Антон Сергеевич
    }
    \par\vfill
    Москва 2024
\end{center}
\thispagestyle{empty}
\pagebreak





    \begin{abstract}
        Пусть $F$ свободная некоммутативная про-$p$ группа, и пусть $\Delta$ коммутативное нетерово полное локальное кольцо с максимальным идеалом $I$, такое что
        $\Delta/I$ конечное поле характеристики $p$\\
        Определим группу
        \[GL_d^1(\Delta) = \ker\left( GL_d(\Delta) \xrightarrow{\Delta\to\Delta/I} GL_2(\Delta/I) \right)\]
        А.Н.\ Зубков доказал, что $F$ не может быть непрерывно вложена в $GL_2^1(\Delta)$ для $p\neq2$.\\
        Д.\ Бен-Эзра и Е.\ Зельманов, показали, что и для $p=2$, $\mathrm{char}(\Delta)=2$ имеет место такой же результат.\\
        Цель данной статьи обобщить подход для $p=2$ и $\mathrm{char}(\Delta)=4$.\\
        Кроме того, Зельманов показал в ~\cite{Zelmanov1}, что гипотеза о нелинейности про-$p$ групп тесно связана с PI-теорией.

        Во второй части данной статьи мы изучаем связь между PI-теорией и гипотезой Гельфанда о конечномерности гомологий алгебр Ли векторных полей.

        Таким образом, можно видеть, что работа в основном посвящена изучению комбинаторики подстановок.
    \end{abstract}
    \tableofcontents


    \section{Введение}\label{sec:introduction}
    Мы рассмотрим применения теории полиномиальных тождеств (PI-теории для краткости) в двух, казалось бы, несвязанных с ней областях:
\begin{itemize}
    \item В 2005, Е.И.\ Зельманов представил набросок доказательства нелинейности некоторых свободных про-$p$ групп.
    Доказательство во многом опирается на стандартные подходы PI-теории (см.\ ~\cite{Zelmanov1}).
    \item Дополнительно мы рассмотрим неожиданную связь между гипотезой Гельфанда и PI-теорией, а именно с методами А.В.\ Гришина (\cite{Grishin}).
\end{itemize}

Приведем краткую историческую справку и формулировку основных проблем.
Отметим, что параграфы 2, 3, 4 и параграфы 5, 6 независимы.

\subsection{О нелинейности свободных про-$p$ групп}\label{subsec:introduction-pro-p}
Проблема линейности топологических групп изучалась много лет.
Одной из естественных топологий, наряду с $\mathbb{R}^n$ и дискретной, является про-$p$ топология.

Сформулируем центральную гипотезу данной теории, а далее приведем все необходимые определения:

\vskip 0.1in\noindent
\begin{conjecture}
    Некоммутативная свободная про-$p$ группа не может быть вложена в $GL^1_d(\Delta)$ ни для какого про-$p$ кольца $\Delta$.
\end{conjecture}
\vskip 0.1in\noindent

\vskip 0.1in\noindent
\begin{definition}
    Обратный (проективный) предел конечных $p$-групп называется про-$p$ группой.
\end{definition}
\vskip 0.1in\noindent
Топология индуцируется с топологии тихоновского произведения.

\vskip 0.1in\noindent
\begin{definition}
    Коммутативное нетерово $I$-полное локальное кольцо $\Delta$ с максимальным идеалом $I$ называется про-$p$ кольцом, если $\Delta/I$ конечное поле характеристики $p$.
\end{definition}
\vskip 0.1in\noindent

Таким образом:
\[
    \Delta \cong \varprojlim \Delta/I^n
\]

Рассмотрим конгруенц-подгруппу:
\[
    GL_d^1(\Delta) = \ker\left( GL_d(\Delta) \xrightarrow{\Delta\to\Delta/I} GL_d(\Delta/I) \right)
\]
Можно заметить, что это про-$p$ группа.


Основной вопрос состоит в том, может ли свободная про-$p$ группа быть непрерывна вложена в $GL_d^1(\Delta)$.
Мотивируя данную постановку, отметим, что А.Н.\ Зубков заметил (\cite{Zubkov}), что, доказывая нелинейность про-$p$ групп, мы можем ограничиться рассмотрением только про-$p$ колец и строить вложение в конгруенц-подгруппу.

Свободную про-$p$ группу можно определить классическим способом через универсальное свойство или конструктивно:

\vskip 0.1in\noindent
\begin{definition}
    Свободная про-$p$ группа $F_p(X)$ является пополнением дискретной свободной группы $F(X)$ относительно топологии всех нормальных подгрупп индекса степени $p$.
\end{definition}
\vskip 0.1in\noindent

Существует множество частичных результатов:
\begin{itemize}
    \item В 1987, А.Н.\ Зубков (\cite{Zubkov}) доказал гипотезу для $d=2, p\neq2$.
    \item В 1999, используя глубокие результаты Р.\ Пинка (\cite{Pink}), Й.\ Барнеа, М.\ Ларсен (\cite{Barnea-Larsen}) подтвердили гипотезу для $\Delta=\left( \mathbb{Z}/p\mathbb{Z} \right)[[t]]$
    \item В 1991, Д.\ Диксон, А.\ Манн, М.П.Ф.\ Ду Сатой, Д.\ Сигал (\cite{DMSD}) доказали гипотезу для всех размеров матриц для целых $p$-адических чисел $\Delta=\mathbb{Z}_p$, $GL_d^1(\mathbb{Z}_p)=\ker\left( GL_2(\mathbb{Z}_p) \xrightarrow{\mathbb{Z}_p\to\mathbb{F}_p} GL_2(\mathbb{F}_p) \right)$
    \item В 2005, E.И.\ Зельманов (\cite{Zelmanov1}, ~\cite{Zelmanov2}) анонсировал доказательство гипотезы для $p\gg d$, однако до сих пор существует только набросок доказательства.
    \item В 2020, Д.\ Бен-Эзра, Е.И.\ Зельманов (\cite{Ben-Ezra-Zelmanov}) обобщили результат Зубкова для $d=2, p=2$ и $\mathrm{char}(\Delta)=2$.
\end{itemize}

Мы сконцентрируемся на случае $2\times 2$ матриц, то есть на результатах А.Н.\ Зубкова и Бена-Эзры\textemdash Зельманова.

Сейчас опишем общий план доказательства, он частично реализован для произвольных размеров матриц в\ \cite{Zelmanov1}, ~\cite{Zelmanov2}:

Для начала сузим множество рассматриваемых колец.
Можно построить, так называемое, универсальное представление в общие матрицы.
Оказывается, что классическими алгебраическими рассуждениями можно показать, что если какое-то представление точно, то и это универсальное представление точно.

Таким образом, остается исследовать это универсальное представление.
Это делается путем изучения алгебры Ли общих матриц, которая множество раз возникала в PI-теории.
В завершение строится связь между этой алгеброй Ли и образом нашего представления ~--- подобно связи между группой Ли и алгеброй Ли.

\subsection{PI-теория и гипотеза Гельфанда}\label{subsec:introduction-gelfand}
В 1980-х годах решение проблемы Шпехта А.Р.\ Кемером стало значительным прорывом в теории полиномиальных тождеств (\cite{Kemer}, см.\ также упрощенную версию доказательства А.Р.\ Кемера в~\cite{SimpleKemer}, ~\cite{Procesi}):
\vskip 0.1in\noindent
\begin{theorem*} [А.Р. Кемер, 1987]
    Любая ассоциативная алгебра над полем характеристики ноль имеет конечный базис тождеств.
\end{theorem*}
\vskip 0.1in\noindent

Существует хорошо известная переформулировка теоремы А.Р.\ Кемера:

\vskip 0.1in\noindent
\begin{theorem*}
    Любой $T$-идеал алгебры $k\langle X\rangle$, где $X$ — счетный алфавит, а $k$ — поле характеристики ноль, конечно базируем.
\end{theorem*}
\vskip 0.1in\noindent

$T$-идеал — это идеал в $k\langle X\rangle$, который замкнут относительно любого эндоморфизма $k\langle X\rangle$.
$T$-пространство — это векторное подпространство в $k\langle X\rangle$, которое замкнуто относительно любого эндоморфизма $k\langle X\rangle$.

Следующий естественный вопрос: можно ли заменить $T$-идеал на $T$-пространство в теореме Кемера?

А.В.\ Гришин заметил, что доказательство А.Р.\ Кемера для систем обобщённых многочленов определённого типа, использует исключительно линейные комбинации и подстановки, а умножение является избыточным (см. \cite{Grishin}, ~\cite{Grishin2}, а также обзор~\cite{GrishinSchigolev}).

Методы А.В.\ Гришина хорошо подходят для полупростых алгебр.
А.Я.\ Канель-Белов заметил, что лемма Артина-Риса (классическая лемма в коммутативной алгебре) и теорема Ю.П.\ Размыслова (см. \cite{GrishinSchigolev}, кроме того эти переходы содержатся в курсовой работе автора прошлого года) могут дополнить рассуждения А.В.\ Гришина.
Он осуществил переход к алгебре меньшей сложности, индуктивно повторяя методы Гришина.
Однако этот подход применим только для локальной проблемы Шпехта (когда $X$ — конечный алфавит).

В 2001 году В.В. Щиголев объединил методы А.Я.\ Канеля-Белова и А.В.\ Гришина, дополнительно заметив, что подход, приведенный у А.Р.\ Кемера, можно применить к локализации проблемы Шпехта для $T$-пространств.
И, наконец, он доказал (см. \cite{Shchigolev}):
\vskip 0.1in\noindent
\begin{theorem*} [В.В. Щиголев, 2001]
    Любое $T$-пространство алгебры $k\langle X\rangle$, где $X$ — счетный алфавит, а $k$ — поле характеристики ноль, конечно базируемо.
\end{theorem*}
\vskip 0.1in\noindent

Существует множество неассоциативных постановок этой проблемы: для алгебр Ли (см. \cite{Lie}), для Йордановых алгебр (см. \cite{Jordan}), для супералгебр (см. \cite{Super}).
Также существуют постановки над полем положительной характеристики\ (см. контрпримеры для $T$-идеалов в работах \cite{ConterKanel}, \cite{ConterGrishin}, \cite{ConterShchigolev} А.Я.\ Канеля-Белова, А.В.\ Гришина и В.В.\ Щиголева соответственно).
И для алгебр градуированных конечной группой (см.\ работу~\cite{GradedKanel} и работу~\cite{GradedSviridova} для случая коммутативной конечной группы).

В 2022 была обнаружена замечательная связь между PI-теорией (точнее методами А.В.\ Гришина) и гипотезой И.М.\ Гельфанда сформулированной на ICM’70 (см\ \cite{Gelfand}).
\vskip 0.1in\noindent
\begin{conjecture}[И.М.\ Гельфанд, 1970]
    \label{Gelfand}
    Гомологии подалгебры Ли конечной коразмерности алгебры Ли алгебраических векторных полей на афинном алгебраическом многообразии конечномерны.
\end{conjecture}
\vskip 0.1in\noindent
Эта интересная связь была найдена в результате совместной работы А.С.\ Хорошкина, А.Я. Канель-Белова с некоторым участием автора.
Можно найти набросок доказательства А.С.\ Хорошкина в ~\cite{Feigin-Kanel-Khoroshkin}, ~\cite{Centrone-Kanel-Khoroshkin-Vorobiov}.

Дополнительно, мы заметим, что именно результат, которого касалась курсовая работа автора после второго курса (частный случай методов В.А.\ Гришина~\cite{Grishin}, описанных элементарными методами) помогает в доказательстве гипотезы Гельфанда.


    \section{Подход Зубкова, $p\neq2$}\label{sec:Zubkov}
    %! suppress = EscapeUnderscore
%! suppress = EscapeAmpersand
\begin{theorem}[Зубков, 1989]
    \label{thm:Zubkov-main}
    Некоммутативная свободная про-$p$ группа не может быть непрерывно вложена в $GL^1_2(\Delta)$ для $p\neq 2$.
\end{theorem}

\subsection{Универсальное представление}\label{subsec:zubkov-universal}
Определим алгебру общих матриц над $p$-адическими числами.

Мы будем работать с матрицами $2\times2$, однако все результаты и конструкции этого параграфа дословно переносятся на матрицы произвольного размера.

Рассмотрим формальные степенные ряды от свободных коммутирующих переменных $x_{i,j}, y_{i,j}$ для $i,j \in \{ 1, 2 \}$:
\[
    S = \Z_{p}\langle x_{1,1}, y_{1,1}, \ldots, x_{2,2}, y_{2,2} \rangle
\]
Введем стандартную градуировку $\deg$: любой элемент $S$ записывается в виде $\sum f_i$ где $\deg{(f_i)} = i$.

Рассмотрим идеалы вида
\[
    S_k = \{\sum\limits_k^{\infty} f_i \}
\]

\vskip 0.1in\noindent
\begin{proposition}
    Следующие идеалы
    \[B_{k,n} = S \cdot p^n + S_k \]
    являются идеалами конечного индекса
\end{proposition}
\begin{proof}
    Достаточно доказать, что $\Z_p / p^n\Z_p$ --- конечное кольцо.
    \[
        \ker{(\Z_p \to Z_p / p^n\Z_p )} = \Z_p p_n = \left( 0,\ldots, 0, a_{k+1}, a_{k+2}, \ldots \right)
    \]
    Ясно, что множество целых $p$-адических чисел отличающихся на элементы такого вида конечно и
    $\Z_p / \Z_p p^n \cong \Z / p^n\Z$
\end{proof}

Снабдим $S$ топологией с базой окрестностей нуля состоящей из идеалов $B_{k,n}$.
Ясно, что $S$ является про-$p$ кольцом:
\[
    S / B_{1,1} \cong \Z/p\Z
\]

Наконец, рассмотрим матричное кольцо $M_2(S)$.
Наделив его топологией с базой окрестностей нуля конгруенц-идеалов
\[
    \ker{(M_2(S) \to M_2(S / B_{k,n}))}
\]
Получим, что $M_2(S)$ ~--- про-$p$ кольцо.
%    todo тут хотелось бы аккуратно все эти вещи доказать, почему это про-$p$ кольцо и почему сейчас получится даже про-$p$ группа.
\vskip 0.1in\noindent
\begin{proposition}
    Множество
    \[1 + \ker{(M_2(S) \to M_2(S / S_1))}\]
    является про-$p$ группой.
\end{proposition}
\begin{proof}
    Заметим, что $\ker{(S\to S/S_1)}$ состоит из рядов без свободного члена.
    Получается, что $1 + \ker{(M_2(S) \to M_2(S / S_1))}$ является группой, так как ряд обратим тогда и только тогда, когда его свободный член обратим.

    Также можно заметить, что эта группа полна относительно определенной выше топологии, то есть является про-$p$ группой.
\end{proof}
Рассмотрим общие матрицы $X, Y \in \ker{(M_2(S) \to M_2(S / S_1))}$:
\[
    X=
    \begin{pmatrix}
        x_{1,1} & x_{1,2} \\
        x_{2,1} & x_{2,2}
    \end{pmatrix},
    \quad
    Y=
    \begin{pmatrix}
        y_{1,1} & y_{1,2} \\
        y_{2,1} & y_{2,2}
    \end{pmatrix}
\]
Пусть $F$ --- свободная про-$p$ группа порожденная $x, y$.

Наконец определим универсальное представление:
\[
    \pi:
    \left\{
    \begin{array}{l}
        x \mapsto 1 + X \\
        y \mapsto 1 + Y
    \end{array}
    \right.
\]

Продолжим его на дискретную подгруппу, порожденную $x, y$:
\[
    \pi: \quad \langle x, y \rangle \to \langle 1+X, 1+Y \rangle \subseteq 1 + \ker{(M_2(S) \to M_2(S / S_1))}
\]
А затем можно непрерывно доопределить $\pi$ на всей $F$, построив замыкание $\langle 1+X, 1+Y \rangle$ в топологии $1 + \ker{(M_2(S) \to M_2(S / S_1))}$:
\[
    \pi:\quad F \to G \subseteq 1 + \ker{(M_2(S) \to M_2(S / S_1))}
\]

Итак, следующая теорема позволяет нам изучать только универсальное представление, не задумываясь о других про-$p$ кольцах.

\vskip 0.1in\noindent
\begin{theorem}[Зубков, 1987]
    Пусть $F$ ~--- свободная про-$p$ группа порожденная $x, y$.
    Если существует инъективный непрерывный гомоморфизм $\varphi: F \to GL^1_2(\Delta)$, то и универсальное представление $\pi$ инъективно.
\end{theorem}
\begin{proof}
    Напомним
    \[
        GL^1_2(\Delta) = \ker{(GL(\Delta)\to GL(\Delta/I))}
    \]

    Рассмотрим образы $x, y$:
    \[
        \varphi(x) = 1 + A, \quad
        \varphi(y) = 1 + B
    \]
    Заметим, что
    \[
        A, B \in \ker{(M_2(\Delta)\to M_2(\Delta/I))}
    \]
    так как $1 + A, 1 + B \in \ker{(GL(\Delta \to \Delta / I))}$.
    Пусть
    \[
        A=
        \begin{pmatrix}
            a_{1,1} & a_{1,2} \\
            a_{2,1} & a_{2,2}
        \end{pmatrix},
        \quad
        B=
        \begin{pmatrix}
            b_{1,1} & b_{1,2} \\
            b_{2,1} & b_{2,2}
        \end{pmatrix}
    \]
    Ясно, что $\lim a_{i,j}^n = \lim b_{i,j}^n = 0$.
    Тогда можно построить гомоморфизм $\zeta: x_{i,j} \mapsto a_{i,j}, y_{i,j} \mapsto b_{i,j}$, он индуцирует эпиморфизм
    $\hat{\zeta}: G \to \mathrm{Im}\hspace{0.1cm}{\varphi}$.
    Наконец, получаем коммутативную диаграмму
    \begin{center}
        \begin{tikzcd}
            & F\arrow{dl}{\pi} \arrow{dr}{\varphi} & \\
            G\arrow{rr}{\hat{\zeta}} & & GL_2^1(\Delta)
        \end{tikzcd}
    \end{center}
    Следовательно:
    \[
        \ker{\pi} \subseteq \ker{\varphi} \qedhere
    \]

\end{proof}

\subsection{Неточность универсального представления}\label{subsec:zubkov-non-injective}
Введем обозначения:
\begin{itemize}
    \item $\mathbf{S}$ ~--- кольцо степенных рядов от общих матриц $X, Y$ над $\Z_p$
    \item $\LQ$ ~--- алгебра Ли порожденная общими матрицами $X, Y$ над $\Q_p$
    \item $\LZ$ ~--- алгебра Ли порожденная общими матрицами $X, Y$ над $\Z_p$
    \item $\LQ^{(n)}$ ~--- векторное пространство над $\Q_p$ однородных элементов степени $n$ в алгебре $\LQ$.
    \item $\LZ^{(n)}$ ~---  $\Z_p$-модуль однородных элементов степени $n$ в алгебре $\LZ$.
    \item Для $g\in 1 + \ker{(M_2(S) \to M_2(S / S_1))}$: $\min{g}$ ~--- однородная компонента наименьшей ненулевой степени (можно записать $g=1 + a_n + a_{n+1} + \ldots$)
    \item Будем записывать коммутатор веса $n$ следующим образом $[l_1, \ldots, l_n] = [[l_1, l_2, \ldots, l_{n-1}], l_n]$
\end{itemize}

\begin{remark}
    \label{rmk:group-vs-lie-commutator}
    Полезно заметить, что
    \begin{gather*}
        \min{(1 + X)(1+Y)} = X + Y\\
        \min{(1 + X)(1+Y)(1+X)^{-1}(1+Y)^{-1}} = [X, Y]\\
    \end{gather*}
\end{remark}

Приведем сначала план доказательства, чтобы была понятна мотивация каждой леммы:
\begin{enumerate}
    \item Определим $G\supseteq G^{(n)}$, вложенную последовательность нормальных подгрупп попадающую в любую окрестность единицы, такую что для любого $g\in G^{(n)}$
    \[
        \min{g} \in \LQn\cap\Sbf = \LZn
    \]
    \item Таким образом, можно изучать $G^{(n)}/G^{(n+1)}$ как $\Z_p$-модуль $\LZn$.
    Докажем, что $\rank_{\Z_p}{\LZn} = f(n)$ для некоторой $f$.
    \item Доказав, что $G^{(n)}$ ~--- нижний центральный ряд $G$ получим противоречие с формулой Э.\ Витта (см. \cite{Lubotzky}):~\begin{proposition}[Витт, \cite{Lubotzky}]
                                                                                                                                      \label{thm:Vitt}
                                                                                                                                      Пусть F ~--- свободная про-$p$ группа порожденная $m$ образующими, тогда $n$-ый фактор нижнего центрального ряда имеет ранг (как $\Z_p$-модуль)
                                                                                                                                      \[
                                                                                                                                          \frac{1}{n}\sum\limits_{d\mid n} \mu(d) m^{n / d}
                                                                                                                                      \]
                                                                                                                                      где $\mu$ ~--- функция Мебиуса.
    \end{proposition}
\end{enumerate}
Итак, докажем следующее техническое утверждение:
\vskip 0.1in\noindent
\begin{proposition}
    \label{thm:LQn-to-LZn}
    При $p\neq 2$:
    \[
        \LQn \cap \Sbf = \LZn
    \]
\end{proposition}
\begin{remark}
    Данное предложение не верно для $p=2$, что влечет существенные сложности, возникающие для $p=2$.
\end{remark}
\begin{proof}
    Пусть
    \begin{align*}
        & a = 4\xod\xdo + (\xoo - \xdd)^2 \\
        & b = 2(\yod\xdo + \xod\ydo) + (\xoo - \xdd)(\yoo-\ydd)\\
        & c = 4\yod\ydo + (\yoo - \ydd)^2
    \end{align*}
    Легко проверить, что
    \begin{align*}
        & [x,y,x,x] = a[x,y] \\
        & [x, y, y, x] = b[x,y] \\
        & [x,y,y,y] = c[x,y] \\
        & [x,y,x,y] = b[x,y]
    \end{align*}
    Получаем, что любой $l\in \LQn$ имеет вид
    \begin{equation}
        l =
        \begin{cases}
            \sum\limits_{i_a + i_b + i_c = (n - 2) / 2} \lambda_{i_a,i_b,i_c} a^{i_a}b^{i_b}c^{i_c}[x,y], & \text{если $n$ четно} \\
            \sum\limits_{i_a + i_b + i_c = (n - 3) / 2} \alpha_{i_a,i_b,i_c} a^{i_a}b^{i_b}c^{i_c}[x,y,x] +
            \beta_{i_a,i_b,i_c} a^{i_a}b^{i_b}c^{i_c}[x,y,y], & \text{если $n$ нечетно}
        \end{cases}\\
        \label{eq:not-direct-sum}
    \end{equation}
    где $\lambda_{i_a,i_b,i_c},\alpha_{i_a,i_b,i_c},\beta_{i_a,i_b,i_c}\in\Q_p$

    Разберем случай нечетного $n$.
    Рассмотрим какое-то $l\in\LQn \cap \Sbf$ с нецелыми $p$-адическими $\alpha_{i_a,i_b,i_c},\beta_{i_a,i_b,i_c}$.

    Далее рассуждение аналогично классическому доказательству теоремы Гильберта о базисе.
    Введем лексикографический порядок на мономах порожденный отношением
    \[
        \xod>\xoo>\yod>\yoo>\xdo>\xdd>\ydo>\ydd
    \]
    Старшие члены у $a,b,c$: $4\xod\xdo, 2\xod\ydo, 4\yod\ydo$ соответственно.
    Тогда старший член элемента, стоящего в левом верхнем углу $l$ равен старшему члену выражения
    \[
        \sum\limits_{i_a + i_b + i_c = (n - 3) / 2}  2^{2i_a + 2i_c + i_b}
        \left(
        \alpha_{i_a,i_b,i_c}
        \xoo\yod^{i_c}\xdo^{i_a}\ydo^{i_b+i_c+1} +
        \beta_{i_a,i_b,i_c}
        \yod^{i_c}\yoo\xdo^{i_a}\ydo^{i_b+i_c+1}\right)
    \]
    Старшие члены различны и $2^{k}$ обратим в кольце $\Z_p$ (при $p\neq 2$!).

    Следовательно, так как $\alpha_{i_a,i_b,i_c}, \beta_{i_a,i_b,i_c} \in \Q_p\setminus \Z_p$, а значит старший член левого верхнего угла $l$ не лежит в $\Z_p$.
    Получаем противоречие с тем, что $l\in \Sbf$.

    Случай четного $n$ разбирается аналогично.
    \begin{remark}
        Попутно мы доказали, что суммы в формуле~\eqref{eq:not-direct-sum} на самом деле прямые.
    \end{remark}
\end{proof}

Сразу же получаем следствие
\begin{colloraly}
    \label{thm:LZn-rank}
    \[
        rank_{\Z_p} \LZn = \dim_{\Q_p} \LQn =
        \begin{cases}
            \frac{n(n+2)}{8}, & \text{при четном $n$} \\
            \frac{(n-1)(n+1)}{4}, & \text{при нечетном $n$}
        \end{cases}
    \]
\end{colloraly}


Итак, пусть
\[
    G^{(n)} = G \cap \ker{(GL(S) \to GL(S / S_n))}
\]

Следующая лемма является ключевой.
\begin{lemma}
    Пусть $g\in G^{(n)}$, тогда
    \[
        \min g \in \LZn
    \]
\end{lemma}
\begin{proof}
    В силу предложения~\ref{thm:LQn-to-LZn} достаточно доказать, что $\min g \in \LQn$.

    Введем обозначения.
    Аналогично предыдущему параграфу $\SQ$ ~--- алгебра формальных степенных рядов от переменных $x_{i,j}, y_{i,j}$ над $\Q$, $\SQn$ ~--- идеалы конечного индекса.

    Зафиксируем произвольное $r>n$, чтобы работать в алгебре $M_2(\SQo) / M_2(\SQr)$ и общие матрицы были нильпотентны.
    Таким образом можем ввести стандартное экспоненциальное отображение:

    \begin{gather*}
        \exp X \mapsto \sum\limits_{k=0}^{\infty} X^k / k!\\
        \exp: M_2(\SQo) / M_2(\SQr)\to 1 + M_2(\SQo) / M_2(\SQr)
    \end{gather*}

    Кроме того, можно ввести и обратное отображение
    \begin{gather*}
        \log 1 + X \mapsto \sum\limits_{k=1}^{\infty} (-X)^k / k\\
        \log: 1 + M_2(\SQo) / M_2(\SQr)\to M_2(\SQo) / M_2(\SQr)
    \end{gather*}
    Пусть $\LQcentralr$ ~--- алгебры Ли нижнего центрального ряда $\LQ$.

    Далее остается применить формулу Бейкера-Кэмпбелла-Хаусдорфа.
    Применим экспоненту к алгебре $\LQ / \LQcentralr$, обозначим образ через $H$.
    $H$ ~--- подгруппа $M_2(\SQo) / M_2(\SQr)$.
    Тогда
    \begin{multline*}
        H^{(n)} = H\cap (1 + M_2(\SQn) / M_2(\SQr)) = \exp{\left( \LQ/\LQcentralr \cap M_2(S_n)/M_2(S_r) \right)} = \\
        \exp{(\Lqn \oplus \ldots \oplus \LQ_{r-1})}
    \end{multline*}
    Пусть $\varphi$ ~--- отображение факторизации $M_2(\SQ)\to M_2(\SQ)/M_2(\SQr)$.
    Таким образом, для каждого элемента $h \in H^{(n)}$, минимум $\min h$ представляется в виде коммутатора веса $n$ от $A=\varphi(X), B=\varphi(Y)$.

    Обозначим
    \begin{gather*}
        -\sum\limits_{k=1}^{\infty} (-X)^k / k =
        \begin{pmatrix}
            x_{1,1} + x_{1,1}^* & x_{1,2} + x_{1,2}^* \\
            x_{2,1} + x_{2,1}^* & x_{2,2} + x_{2,2}^*
        \end{pmatrix}\\
        -\sum\limits_{k=1}^{\infty} (-Y)^k / k =
        \begin{pmatrix}
            y_{1,1} + y_{1,1}^* & y_{1,2} + y_{1,2}^* \\
            y_{2,1} + y_{2,1}^* & y_{2,2} + y_{2,2}^*
        \end{pmatrix}\\
    \end{gather*}
    где $x_{i,j}, y_{i,j} \in S_{\mathbb{Q}, 2}$.
    Заметим, что
    \begin{gather*}
        \log(1 + A) = \varphi
        \begin{pmatrix}
            x_{1,1} + x_{1,1}^* & x_{1,2} + x_{1,2}^* \\
            x_{2,1} + x_{2,1}^* & x_{2,2} + x_{2,2}^*
        \end{pmatrix}\\
        \log(1 + B) = \varphi
        \begin{pmatrix}
            y_{1,1} + y_{1,1}^* & y_{1,2} + y_{1,2}^* \\
            y_{2,1} + y_{2,1}^* & y_{2,2} + y_{2,2}^*
        \end{pmatrix}\\
    \end{gather*}



\end{proof}

Пусть $G=G_1,\ldots,G_n,\ldots$ ~--- нижний центральный ряд.
Аналогично замечанию~\ref{rmk:group-vs-lie-commutator} получаем следующее тривиальное предложение.
\begin{proposition}
    \label{prp:G_n-contains-L_n}
    \[
        \min F_n \supseteq \LZn
    \]
\end{proposition}

Следующее предложение завершает доказательство теоремы~\ref{thm:Zubkov-main}
\begin{proposition}
    Нижний центральный ряд совпадает с пересечениями группы $G$ с конгруенц-подгруппами по идеалу $S_n$, то есть
    \[
        G_n = G^{(n)}
    \]
\end{proposition}
\begin{proof}
    Ясно, что $G_n \subseteq G^{(n)}$ по определению $S_n$.\\
    С другой стороны, пусть $g\in G^{(n)}$:
    \[
        g = 1 + v_n(X, Y) + \text{старшие члены}
    \]
    где $v_n(X,Y)$ ~--- линейная комбинация коммутаторов веса $n$ над $\Z_p$.

\end{proof}



    \section{Подход Бена-Эзры\textemdash Зельманова, $p = 2, \char(\Delta) = 2$}\label{sec:Ben-Ezra-Zelmanov}
    Д.\ Бен-Эзра и Е.И.\ Зельманов доказали следующую теорему
\begin{theorem}[А.Н.\ Зубков, 1989]
    \label{thm:ben-ezra-zelmanov-main}
    Некоммутативная свободная про-$2$ группа не может быть непрерывно вложена в $GL^1_2(\Delta)$ для про-$2$ кольца $\Delta$ характеристики $2$.
\end{theorem}
Авторы начинают доказательство с все той же идеи универсального объекта.

\subsection{Универсальное представление}\label{subsec:ben-ezra-zelmanov-universal}
Для построения кольца общих матриц авторы взяли основное кольцо $\Z/2\Z$ вместо $2$-адических чисел, как это было у Зубкова.
Введем аналогичные~\ref{subsec:zubkov-universal} обозначения.

\begin{itemize}
    \item $S = \Z/2\Z [[ x_{1,1}, y_{1,1}, \ldots, x_{2,2}, y_{2,2} ]]$
    \item $S_k = \{\sum\limits_k^{\infty} f_i \}$, где $f_i\in S$ однородные многочлены степени $i$.
    \item Наделив $S$ аналогичной топологией и рассмотрим про-$2$ группу $1 + \ker{(M_2(S) \to M_2(S / S_1))}$
    \item Опять же
    \[
        X=
        \begin{pmatrix}
            x_{1,1} & x_{1,2} \\
            x_{2,1} & x_{2,2}
        \end{pmatrix},
        \quad
        Y=
        \begin{pmatrix}
            y_{1,1} & y_{1,2} \\
            y_{2,1} & y_{2,2}
        \end{pmatrix}
    \]
    \item $F$ ~--- свободная про-$p$ группа порожденная $x,y$
    \item Определим универсальное представление $\pi: x \mapsto 1 + X, y \mapsto 1 + Y$ и, как и раньше, продолжим его на всю $F$, образ будет про-$p$ подгруппой $G\subseteq 1 + \ker{(M_2(S) \to M_2(S / S_1))}$.
\end{itemize}

\begin{theorem}
    Пусть $F$ ~--- свободная про-$2$ группа порожденная $x, y$.
    Если существует инъективный непрерывный гомоморфизм $\varphi: F \to GL^1_2(\Delta)$ для про-$2$ кольца $\Delta$ характеристики 2, то и универсальное представление $\pi$ инъективно.
\end{theorem}
\begin{proof}
    Доказательство аналогично теореме\ref{thm:zubkov-universal} основано на универсальности общих матриц.

    Пусть $\sigma: F\to GL_2^1(\Delta)$.
    \[
        \sigma (x) \mapsto 1 +
        \begin{pmatrix}
            a_{1,1} & a_{1,2} \\
            a_{2,1} & a_{2,2}
        \end{pmatrix}, \quad
        \sigma (y) \mapsto 1 +
        \begin{pmatrix}
            b_{1,1} & b_{1,2} \\
            b_{2,1} & b_{2,2}
        \end{pmatrix},
    \]
    Тогда, пользуясь тем, что $\char \Delta = 2$ определим, отображение $\zeta: x_{i,j}\mapsto a_{i,j}$, которое все так же индуцирует эпиморфизм $\hat{\zeta}: G \to \mathrm{Im}\hspace{0.1cm}{\varphi}$.
    И диаграмма коммутативна
    \begin{center}
        \begin{tikzcd}
            & F\arrow{dl}{\pi} \arrow{dr}{\varphi} & \\
            G\arrow{rr}{\hat{\zeta}} & & GL_2^1(\Delta)
        \end{tikzcd}
    \end{center}
    Следовательно:
    \[
        \ker{\pi} \subseteq \ker{\varphi} \qedhere
    \]
\end{proof}

\subsection{Разница со случаем $p>2$}\label{subsec:ben-ezra-zelmanov-difference}
В основе доказательства А.Н.\ Зубкова лежит предложение~\ref{thm:LQn-to-LZn} о том,
если элемент алгебры Ли оказался многочленом с целыми $p$-адическими коэффициентами, то он лежит и в алгебре Ли с целыми $p$-адическими коэффициентами.\\
Из этого предложения удается доказать, что нахождение элемента $g$ в нижнем центральный ряд группы соответствует тому, что $\min g$ представляется в виде линейной комбинации коммутаторов.
Однако для центрального ряда формула Витта, с которой получаем противоречие.

Д.\ Бен-Эзра и Е.И.\ Зельманов в приложении своей работы показали, почему подход Зубкова напрямую нельзя обобщить на случай $p=2$: в некотором смысле про-$2$ группа общих матриц ближе к тому, чтобы быть свободной:
\begin{proposition}[Д. Бен-Эзра, Е.И.\ Зельманов, 2020]
    \label{prp:ben-ezra-zelmanov-difficulty}
    В обозначениях предыдущего параграфа для $p=2$, $G_6 / G_7$ ~--- абелева порожденная не менее, чем $9$ образующими.
\end{proposition}
Заметим, что формула Витта как раз дает значение $9$.
Для контраста еще раз приведем предложение, доказанное А.Н.\ Зубковым, оно противоречит формуле Витта для $n=6$:
\begin{proposition}
    При $p>2$ $G_n / G_{n+1} \cong \Z_p^{f(n)}$, где\\
    \[
        f(n) =
        \begin{cases}
            p^{\frac{n(n+2)}{8}}, & \text{при четном $n$} \\
            p^{\frac{(n-1)(n+1)}{4}}, & \text{при нечетном $n$}
        \end{cases}
    \]
\end{proposition}

В дополнение авторы приводят полное описание $\LQn \cap \Sbf$ для $p=2$, которым однако не пользуются в основном тексте, а доказывают через него предложение~\ref{prp:ben-ezra-zelmanov-difficulty}.

В следующем параграфе мы приведем план доказательства Бена-Эзры\textemdash Зельманова ~--- полное доказательство содержит более 20-и страниц вычислений, так что мы постараемся привести основные идеи, опуская вычисления.

\subsection{Набросок доказательства}\label{subsec:ben-ezra-zelmanov-non-injective}



    \section{Случай $p=2, \char(\Delta) = 4$}\label{sec:char-4}
    В данном параграфе мы приводим план исследования следующих проблем.

\begin{conjecture}
    Некоммутативная свободная про-$2$ группа не может быть непрерывно вложена в $GL^1_2(\Delta)$ для про-$2$ кольца $\Delta$ характеристики $4$.
\end{conjecture}
\begin{conjecture}
    Некоммутативная свободная про-$2$ группа не может быть непрерывно вложена в $GL^1_2(\Delta)$ для про-$2$ кольца $\Delta$ характеристики $2^n$.
\end{conjecture}
\begin{conjecture}
    Некоммутативная свободная про-$2$ группа не может быть непрерывно вложена в $GL^1_2(\Delta)$ ни для какого для про-$2$ кольца $\Delta$.
\end{conjecture}

Естественно все также имеют место следующие утверждения

\begin{proposition}
    Тогда если существует непрерывное вложение про-$2$ группы в $GL_2^1(\Delta)$ для какого-то про-$2$ кольца $\Delta$ характеристики 4, то и универсальное представление в про-$2$ группу, порожденную $1 + \text{общие матрицы над кольцом} \Z_4$ (определенное как раньше) инъективно.
\end{proposition}
\begin{proposition}

\end{proposition}
\begin{proposition}

\end{proposition}
%\subsection{Универсальное представление}\label{subsec:char-4-universal}
%\subsection{Набросок доказательства}\label{subsec:char-4-non-injective}


    \section{Методы Гришина}\label{sec:Grichin}
    Пусть $k$ — поле характеристики ноль, а $F = k\langle x_1, \ldots, x_i, \ldots \rangle$ — свободная, счётно порождённая ассоциативная алгебра над полем $k$.
Пусть $T$ — полугруппа эндоморфизмов (подстановок) $F$.
И $X = \{ x_1, \ldots, x_i, \ldots \}$.\\
Теперь приведем несколько классических определений.

\vskip 0.1in\noindent
\begin{definition}
    Эндоморфизм $\tau$ алгебры $F$, определяемый правилом $x_i \mapsto g_i$, где $g_i \in F$, называется подстановкой типа $(x_1, \ldots, x_i, \ldots) \mapsto (g_1, \ldots, g_i, \ldots)$.
\end{definition}
\vskip 0.1in\noindent

\vskip 0.1in\noindent
\begin{definition}
    $T$-пространство в $F$ — это векторное подпространство $F$, замкнутое относительно подстановок.
\end{definition}
\vskip 0.1in\noindent

\vskip 0.1in\noindent
\begin{definition}
    $T$-идеал в $F$ — это идеал $F$, который одновременно является $T$-пространством.
\end{definition}
\vskip 0.1in\noindent

Следующая теорема (а вернее, метод ее доказательство) очень полезна при изучении гипотезы Гельфанда:
\vskip 0.1in\noindent
\begin{theorem}
    \label{main}
    Пусть $M$ ~--- подмножество кольца многочленов от $n$ переменных над полем характеристики 0.
    Тогда существует конечное подмножество $M$, такое что любой многочлен из $M$ можно получив из элементов этого подмножества, конечное число раз применив следующие операции:
    \begin{itemize}
        \item Если в множестве есть несколько многочленов, то можно добавить их линейную комбинацию.
        \item Если в множестве был многочлен $F(x_1,\ldots,x_n)$, то можно добавить $F(t(x_1),\ldots,t(x_n))$, где $t$ -- любой многочлен одной переменной.
    \end{itemize}
\end{theorem}
\vskip 0.1in\noindent
В следующих параграфах приведено элементарное доказательство этой теоремы, отражающее суть методов Гришина.

\subsection{Основные леммы}\label{subsec:grishin-main-lemmas}

Для начала докажем три простых леммы.\vskip 0.1in\noindent
\begin{lemma}
    \label{closure}
    Достаточно доказать теорему~\ref{main} для случая, когда $M$ замкнуто относительно приведенных операций, то есть является $T$-пространством.
\end{lemma}
\begin{proof}
    Рассмотрим $\overline{M}$ --- множество, получаемое из $M$ применением конечного количества последовательных операций.\\
    Ясно, что $\overline{M}$ замкнуто относительно них.
    Действительно, если взять любую линейную комбинацию многочленов из $\overline{M}$, то она получается из $M$ последовательным применением конечного числа операций и, следовательно, лежит в $\overline{M}$.
    То же самое верно и про подстановку.
    Осталось доказать, что если $\overline{M}$ конечно базируемо, то и $M$ --- тоже.
    Пусть $\overline{M_0}$ --- конечная база $\overline{M}$.
    Тогда $M_0$ --- это множество многочленов, из которых мы получили $\overline{M_0}$.
    По построению $M_0$ конечно и является базой $M$.
\end{proof}
Следующая лемма также сужает класс множеств для которых мы будем доказывать теорему: можно считать, что $M$ содержит исключительно однородные многочлены.\vskip 0.1in\noindent
\begin{lemma}
    \label{homogen}
    Из многочлена $F$ разрешенными операциями можно получить его однородные компоненты.
\end{lemma}
\begin{proof}
    Пусть $m$ -- степень $F$.\\
    Сделаем подстановку: $F(\alpha x_1, \alpha x_2, \ldots, \alpha x_n)$.
    Заметим, что при такой подстановке однородная компонента $k$-ой степени умножается на $\alpha^k$.
    Тогда пусть $v_0,\ldots,v_n$ -- однородные компоненты $F$. \\
    Заметим, что все подстановки указанного вида лежат в линейной оболочке $<v_0,v_1,\ldots,v_n>$.
    Тогда заметим, что у многочлена $F(\alpha x_1, \alpha x_2, \ldots, \alpha x_n)$ в нашем базисе это $(\alpha^0,\alpha^1,\ldots,\alpha^n)$.
    Если вспомнить, что определитель матрицы Вандермонда не $0$, то получаем, что если подставить любые $m+1$ различных $\alpha$, то сможем выразить любое $v_i$.
\end{proof}
Теперь можно считать, что множество $M$ состоит только из однородных многочленов, однако возникает проблема: нам было бы удобно считать и то, что множество замкнуто относительно операций, и то, что оно состоит из однородных многочленов.
Однако сейчас эти условия, конечно, несовместимы.\\
Эта проблема решается следующим образом: разрешим применять только такую последовательность операций, которая из однородного многочлена получает однородный.
Такие последовательности операций будем называть однородной подстановкой.\\
Когда решается задача про конечную базируемость множества многочленов (например, можно провести аналогию с базисом Грёбнера из теоремы Гильберта о базисе) полезной является лемма про светильники -- она помогает следующим образом: если ввести порядок на старших членах многочленов и научиться выражать большие старшие члены из маленьких, то из этого получим конечную базируемость.\\
Следующая лемма очевидным образом следует из теоремы Гильберта о базисе, но так как в курсе алгебры часто выводится следствие в обратную сторону, то приведем все-таки независимое доказательство этой леммы.\vskip 0.1in\noindent
\begin{lemma}
    \label{lamp}
    Пусть в $\mathbb{Z}_+^n$ дано некоторое множество светильников -- точка с координатами $(x_1,\ldots,x_n)$ освещает все $(y_1,\ldots,y_n), y_i\geq x_i \forall i$.
    Тогда можно выбрать конечное количество светильников, которые освещают все остальные.
\end{lemma}
\begin{proof}
    Докажем индукцией по размерности пространства.\\
    При $n=1$ утверждение леммы тривиально.\\
    Пусть лемма верна для $n-1$, докажем для $n$. \\
    Спроецируем все светильники на гиперплоскость $x_n=0$.
    Выберем конечный набор проекций $m_1',\ldots,m_k'$, который освещает все остальные проекции (такой есть в силу индукционного предположения).
    Теперь найдем $m_i$ --- светильник с наименьшей координатой по $x_n$ из тех, что спроецировались в $m_i'$.\\
    Пусть $l$ --- наибольшая координата по $x_n$ среди $m_i$.
    Заметим, что для все светильники с координатой по $x_n$, не меньшей $l$ освещены множеством $m_1,\ldots,m_k$.
    Осталось для каждой гиперплоскости $x_n=c$, где $c=0,\ldots,l-1$ выбрать конечные наборы светильников, которые освещают все остальные, лежащие в соответствующей гиперплоскости.
\end{proof}

\subsection{Ключевая подстановка}\label{subsec:key-substitution}
Следующая подстановка является ключевой:
\begin{equation}
    \label{eq:key-sub}
    P(x_1,\ldots,x_n)\mapsto P_{(t, 1)}(x_1+t(x_1)x_1,\ldots,x_n+t(x_n)x_n)
\end{equation}
где индекс $(t,1)$ обозначает, что мы линеаризуем по $t$ после того, как, сделаем приведенную подстановку.\\
Докажем, что мы можем отлинеаризовать по $t$.
Будем рассматривать многочлен $P(x_1+t(x_1)x_1,\ldots,x_n+t(x_n)x_n)$ как многочлен от $2n$ переменных -- от $x_i$ и $t(x_i)$.
Причем понятно, что переменные $t(x_i)$ можно умножать на константу все сразу -- в подстановке~\ref{eq:key-sub} надо взять $x_i\mapsto x_i+c\cdot t(x_i)x_i$.
Далее все аналогично доказательству леммы~\ref{homogen}.\\
Итак, разберемся, что получилось после отлинеаризованной подстановки (\ref{eq:key-sub}). Для этого поймем, что произойдет с мономом $x_1^{\alpha_1}\ldots x_n^{\alpha_n}$:
\[x_1^{\alpha_1}\ldots x_n^{\alpha_n}\mapsto ((x_1+t(x_1)x_1)^{\alpha_1}\ldots (x_n+t(x_n)x_n)^{\alpha_n}))_{(t,1)}\]Ясно, что после линеаризации по $t$ останутся только те члены, в которых ровно из одной скобки произведения взяли $t(x_i)$.
То есть при $t(x_i)$ будет $\alpha_i\cdot x_1^{\alpha_1}\ldots x_n^{\alpha_n}$.
Итак:
\begin{equation}
    \label{eq:key_monom}
    x_1^{\alpha_1}\ldots x_n^{\alpha_n}\mapsto\sum\limits_{i=1}^n t(x_i)(\alpha_i\cdot x_1^{\alpha_1}\ldots x_n^{\alpha_n})
\end{equation}
Следовательно, однородный многочлен $P$ степени $m$ перейдет в:
\begin{equation}
    \label{eq:key_t}
    \sum\limits_{i=1}^n t(x_i)P_i(x_1,\ldots,x_n)
\end{equation}
Причем, если вместо $t$ подставить $\frac{1}{m}$, то из последней формулы~\eqref{eq:key_t} получится ровно многочлен $P$.\\
Далее, нам потребуются лишь $t(x)=x^k$.
Каждому многочлену из множества $M$ мы сопоставили некоторое семейство многочленов:
\begin{equation}
    \label{eq:key_sum}
    \sum\limits_{i=1}^n x_i^k P_i(x_1,\ldots,x_n)
\end{equation}
где $k$ пробегает все целые неотрицательные числа.\\
Обозначим сумму~\eqref{eq:key_sum} как $(P_1,\ldots,P_n)$ и назовем ее координатным представлением многочлена.\\
Пока не очень понятно, чего мы хотим добиться и зачем мы рассмотрели это семейство.
На самом деле мы будем делать однородные подстановки специального вида уже к семействам.
Причем будем делать их так, чтобы они инвариантно действовали на каждую координату.
Таким образом, мы сможем доказать конечную базируемость по каждой координате, а потом уже выведем из этого глобальную конечную базируемость.
Ключевая идея заключается в том, что для каждой отдельной координаты доказывать конечную базируемость куда проще, чем для бескоординатного представления, так как наличие $i$-ой координаты разрешает нам умножать многочлены внутри нее на $x_i$, что существенно упрощает задачу.\\

Мы будем доказывать конечную базируемость по каждой координате, но для начала докажем, что из этого будет следовать теорема~\ref{main}.

\subsection{Глобальная конечная базируемость из конечной базируемости по каждой координате}\label{subsec:grishin-global-from-local}
Мы рассматриваем вместо каждого многочлена $P(x_1,\ldots,x_n)$ соответствующее ему семейство:
\[\sum\limits_{i=1}^n x_i^k P_i(x_1,\ldots,x_n)\] Пусть $N$ --- множество координатных представлений многочленов из $M$.
Аналогично лемме~\ref{closure}, рассмотрим вместо $N$, его замыкание $\overline{N}$.
То есть будем рассматривать некоторое множество систем многочленов, уже возможно слабо связанное с изначальными многочленами, но содержащее в себе координатное представление каждого многочлена.\\
Теперь осталось явно описать конечный базис $\overline{N}$: возьмем конечное количество семейств многочленов, у которых первая координата является конечным базисом первых координат из $\overline{N}$.
Далее, возьмем $\overline{N}_1\subset \overline{N}$ состоящее только из семейств с нулевой первой координатой (ясно, что это подмножество тоже замкнуто относительно подстановок и линейных комбинаций).
Теперь возьмем те семейства из $\overline{N}_1$, у которых вторая координата является конечным базисом вторых координат.
В итоге на $i$-ом шаге мы берем семейства с нулевыми первыми $i-1$-ой координатами и берем конечное множество систем, у которых $i$-ая координата является конечной базой $i$-ых координат соответствующего множества.
Ясно, что в итоге мы действительно построим конечный базис всех систем.
Действительно: чтобы получить какую-то систему из выбранных, мы можем последовательно занулить все ее координаты.

\subsection{Конечная базируемость по каждой координате}\label{subsec:local}
Ключевая подстановка решает задачу в случае двух переменных, однако в общем случае нужны тонкие индукционные рассуждения.\\
Докажем, что внутри каждой координаты можно опять сделать подстановку (\ref{eq:key-sub}).
\begin{multline*}
(p_1,...,p_n)
    =\sum\limits_{i=1}^n x^k_i p_i(x_1,\ldots,x_n)\mapsto\\\mapsto [\sum\limits_{i=1}^n (x_i+r(x_i)x_i)^k\cdot p_i(x_1+r(x_1)x_1,\ldots,x_n+r(x_n)x_n))]_{(r,1)}
\end{multline*}Здесь индекс $(r,1)$ означает линеаризацию по $r$, как и раньше.
Рассмотрим первую координату.
Она, конечно, полностью получается из члена
\[[(x_1+r(x_1)x_1)^k\cdot p_1(x_1+r(x_1)x_1,\ldots,x_n+r(x_n)x_n)]_{(r,1)}\]Раскрыв первые скобки получаем
\begin{equation}
    \label{eq:sub_interm}
    x_1^k\cdot(k\cdot r(x_1) p_1(x_1,\ldots,x_n)+ [p_1(x_1+r(x_1)x_1,\ldots,x_n+r(x_n)x_n)]_{(r,1)})
\end{equation}

Заметим, что первое слагаемое нам совсем не нужно, ведь оно преобразует только степень мономов $p_1$ по $x_1$, а мы добиваемся увеличения других степеней.
Так что вычтем из семейства, полученного при подстановке~\eqref{eq:sub_interm} изначальное семейство умноженное на $r(x_i)k$ покоординатно.
Получим, что по первой координате у нас:
\[[p_1(x_1+r(x_1)x_1,\ldots,x_n+r(x_n)x_n)]_{(r,1)}\]Сравним это с подстановкой (\ref{eq:key-sub}), специализуем $r(x)=x^m$ и сделаем вывод, что это равно
\begin{equation}
    \label{eq:key_inside}
    \sum\limits_{i=1}^n x_i^m p_{1i}(x_1,\ldots,x_n)
\end{equation}
Итак, внутри каждой координаты можно опять делать ключевую подстановку.
Назовем ее $\varphi$.\\
Докажем теперь теорему в следующем частном случае.

\subsubsection{Доказательство в случае двух переменных.}
Итак, пусть у нас есть многочлен \[x_1^{m_1}p_1+x_2^{m_1}p_2\]
Мы рассматриваем только случай, когда в $p_1$ все мономы делятся $x_1$, ведь иначе первой координаты вообще не существовало бы, то есть она была бы нулевой (что видно из~\eqref{eq:key_monom})
Мы хотим узнать, что из многочлена $x_1^{m_1}p_1+x_2^{m_1}p_2$ можно получить по первой координате.
Здесь потребуется лемма, которая сведет всю задачу к лемме о светильниках (\ref{lamp}).\vskip 0.1in\noindent
\begin{lemma}
    Если у $p_1$ старший член в лексикографическом порядке $x_2\succ x_1$ это $x_1^{\alpha_1}x_2^{\alpha_2}$, причем $\alpha_2>0$, то для любого монома, делящегося на данный, мы можем получить многочлен, у которого по первой координате будет старшим этот моном.
\end{lemma}
\begin{proof}
    Пусть нам нужно получить моном $x_1^{\alpha_1+\beta_1}x_2^{\alpha_2+\beta_2}$.

    Рассмотрим случай, когда $\beta_2=0$.
    Тогда сдвинем %! suppress = VerticallyCenteredColon
    $m_1 := m_1+\beta_1$.
    Естественно, мы получим многочлен с нужным старшим членом
    Пусть теперь $\beta_2>0$.
    Сделаем нашу подстановку еще раз, как мы описали в~\eqref{eq:key_inside}.
    Получится многочлен
    \[x_1^{m_1}(x_1^{m_2} p_{11}+x_2^{m_2} p_{12})\]Причем и в $p_{11}$, и в $p_{22}$ старший член --- это $x_1^{\alpha_1}x_2^{\alpha_2}$.
    Тогда возьмем $m_2=\beta_2$ и, опять же, заменим $m_1$ на $m_1+ \beta_1$.
    Получается
    \[x_1^{m_1}x_1^{\beta_1}(x_1^{\beta_2} p_{11}+x_2^{\beta_2} p_{12})=x_1^{m_1}(x_1^{\beta_2}x_1^{\beta_1}p_{11}+x_2^{\beta_2}x_1^{\beta_1} p_{12})\]Ясно, что старший моном содержится только во втором слагаемом (ведь $\beta_2>0$). Причем он имеет вид $x_1^{\alpha_1+\beta_1}x_2^{\alpha_2+\beta_2}$.
\end{proof}
\begin{remark}
    \label{remark}
    Полезно заметить, что если все же $\alpha_2=0$, то утверждение леммы верно для монома вида $x_1^{\alpha_1+\beta_1}x_2^{\alpha_2}=x_1^{\alpha_1+\beta_1}$.
\end{remark}
\vskip 0.1in\noindent
Итак, осталось свести к лемме о светильниках (\ref{lamp}). Рассмотрим множество $M$ и каждому многочлену из него сопоставим точку в $\mathbb{Z}_+^2$ -- набор степеней в его старшем члене по первой координате.
В этой точке расположим светильник: если ордината не равна 0, то он освещает все точки, которые не меньше его по абсциссе и ординате; если же ордината равна 0, то он освещает все точки, которые совпадают с ним по ординате и не меньше по абсциссе (см замечание~\ref{remark}).\\
Здесь видно, что нам не хватает леммы о светильниках и нужна ее расширенная версия:\vskip 0.1in\noindent
\begin{lemma}
    Пусть, как и в лемме~\ref{lamp}, дано множество светильников, только теперь каждый $(x_1,\ldots,x_n)$ освещает точки $(y_1,\ldots,y_n)$, для которых выполнено два условия: $y_i\geq x_i$ и $y_i=0$, если $x_i=0$.
    Тогда, опять, же существует конечное подмножество светильников, которое освещает все остальные.
\end{lemma}
\begin{proof}
    Пусть $K_i$ --- множество точек, у которых ровно $i$ координат ненулевые.
    Внутри $K_i$ можно найти соответствующий ему конечный набор светильников по классической лемме.
    Остается заметить, что $\mathbb{Z}_+^n=\bigcup_{k=0}^n K_i$.
\end{proof}
Теперь, в силу леммы, можно выбрать конечное подмножество светильников, которые мы сопоставили многочленам, освещающее все остальные.\\
Осталось доказать, что многочлены, которым соответствуют эти светильники являются конечной базой по первой координате (обозначим множество этих многочленов за $M_0$) множества $M$. \\
Предположим противное.
Рассмотрим многочлен с наименьшим старшим членом по первой координате, первая координата которого не получается из $M_0$.\\
Однако мы можем получить многочлен с таким же старшим членом по первой координате.
Вычитая один из другого, получаем меньший многочлен, у которого первая координата не получается из $M_0$.
Противоречие.

\subsubsection{Доказательство в общем случае}
Как уже говорилось в этом параграфе,когда мы делаем подстановку $\varphi$ внутри первой координаты, у нас возникает сумма
\[\sum\limits_{i=1}^n x_i^m p_{1i}(x_1,\ldots,x_n)\]
Видно, что первое слагаемое нам совсем не нужно, ведь оно разрешает умножать на $x_1$, а это мы и так делать можем, в виду того, что работаем с первой координатой.
Так что хотелось бы считать $x_1$ коэффициентом и воспользоваться конечной базируемостью уже для меньшего числа переменных.\\
Эта идея реализуется следующим образом.
Во-первых, разрешим делать только $\varphi$ подстановки, чтобы не потерять однородность.
А во вторых, будем доказывать теорему 1 не над полем, а над произвольным нетеровым кольцом, содержащем подкольцо изоморфное $\mathbb{Q}$ (последнее обобщение позволит нам в индукционном переходе считать $x_1$ коэффициентом, а не переменной).\vskip 0.1in\noindent
\begin{theorem}
    \label{generalization}
    Пусть $M$ --- множество многочленов от $n$ переменных над нетеровым кольцом $R$, содержащем подкольцо изоморфное $\mathbb{Q}$.
    Тогда $M$ конечно базируемо относительно линейных комбинаций и $\varphi$-подстановок.
\end{theorem}\vskip 0.1in\noindent
Ввиду того, что $R$ содержит поле характеристики $0$, леммы~\ref{closure} и~\ref{homogen} дословно переносятся на этот обобщенный случай (ведь рассуждение с матрицей Вандермонда задействует только частный случай $\varphi$-подстановки).
Значит, опять же можно доказывать теорему только для однородных многочленов.\\
Также мы все еще можем разбить многочлен на координаты~\ref{eq:key_sum} и доказать только конечную базируемость по каждой: параграф 4 нигде не использовал того, что множество коэффициентов --- поле.\\
Итак, будем доказывать теорему~\ref{generalization} по индукции по числу переменных.

\vskip 0.1in\noindent
{\large\textbf{База индукции.}}

Пусть $n=1$.\\
Заметим, что каждый многочлен с помощью $\varphi$-подстановки можно умножить на $x$, а с помощью линейной комбинации умножать на элемент кольца.\\
Следовательно из каждого многочлена из $M$ можно получить идеал порожденный им в $R[x]$.
Получается, что база индукции эквивалентна теореме Гильберта о базисе.

\vskip 0.1in\noindent
{\large\textbf{Переход индукции.}}

Пусть теорема верна для $n-1$ переменной, докажем для $n$.\\
Рассмотрим координатные представления многочленов и, без ограничения общности, будем доказывать конечную базируемость по первой.\\
Первая координата имеет вид $x_1^m P$.
То есть, как и раньше, мы можем умножать многочлены внутри первой координаты на $x_1$.
Следовательно, достаточно доказать утверждение теоремы~\ref{generalization} с дополнительной операцией: теперь кроме линейных комбинаций и $\varphi$-подстановок также разрешим умножать на $x_1$.\\
Как описывалось в начале параграфа, мы хотели бы на время забыть, что $\varphi$-подстановка действует на первую координату. \\Введем $\psi$-подстановку:
\[\psi : P(x_1,\ldots,x_n) \mapsto P_{(r,1)}(x_1,x_2+r(x_2)x_2,\ldots,x_n+r(x_n)x_n)\]\vskip 0.1in\noindent
\begin{lemma}
    \label{ind_step}
    Множество многочленов от $n$ переменных конечно базируемо относительно $\psi$-подстановок, умножений на $x_1$ и линейных комбинаций.
\end{lemma}
\begin{proof}
    Заметим, что это множество можно рассматривать, как множество многочленов от $n-1$ переменной ($x_2,\ldots,x_n$) над кольцом $R[x_1]$.
    Причем все условия теоремы 2 в точности соблюдены.
    Так что по предположению индукции получаем требуемое.
\end{proof}
Итак, осталось применить лемму.
Назовем шириной монома его суммарную степень по переменным $x_2,\ldots,x_n$.\\
Пусть нам нужно доказать конечную базируемость множества $M$, причем опять же, как и раньше, будем считать, что оно замкнуто относительно $\varphi$-подстановок и линейных комбинаций.
Тогда пусть $N$ --- множество получаемое из $M$, заменой каждого многочлена на него же, но с удаленными мономами не максимальной ширины.\\
Применив лемму~\ref{ind_step} получаем, что $N$ конечно базируемо относительно $\psi$-подстановок.
Пусть $f'_1,\ldots,f'_k$ конечный базис.\\
Рассмотрим $f_1,\ldots,f_k$ --- какие-то многочлены, которые заменились на $f'_1,\ldots,f'_k$ при переходе от $M$ к $N$.
Предположим, что это не базис $M$.
Тогда возьмем многочлен минимальной ширины $p\in M$, который нельзя получить с помощью $\varphi$-подстановок и линейных комбинаций (здесь мы пользуемся замкнутостью $M$.
Пусть $p'$ --- соответствующий ему однородный по ширине многочлен из $N$.\\
Однако проделаем все те же операции, $\{f_i\}$, что делали с $\{f'_i\}$, чтобы получить $p'$, заменив подстановку вида $\psi$ на соответствующую ей $\varphi$-подстановку.
Заметим, что сначала мы можем применить операции подстановок, а потом уже один раз сделать линейную комбинацию, ведь $\varphi$ и $\psi$ -подстановки линейны.\\
Итак, сделаем все подстановки и заметим, что после линейной комбинации мономы старшей ширины совпадают с $p'$.
Действительно, каждый раз делая  $\varphi$-подстановку вместо $\psi$ у нас возникают <<лишние>> члены не максимальной ширины (ведь отличие между ними в том, что одна бьет по $x_1$ --- переменной не увеличивающей ширину, а другая нет).
Также члены максимальной ширины не сократятся друг с другом или с какими-то другими, потому что они всегда совпадают с теми, что мы получали из $\{f'_i\}$.\\
Получается, что раз $p$ не порождается базисом $\{f_i\}$, а какой-то многочлен с такими же мономами старшей ширины мы получить можем, то существует многочлен $q$ меньшей ширины, чем $p$, который тоже не порождается базисом.
Противоречие.
\vskip 0.1in\noindent
\begin{remark}
    В теореме 2 существенно, что мы доказываем конечную базируемость относительно однородных подстановок.
    Иначе возникли бы эффекты, связанные с тем, что мономы старшей ширины, получаемые из $f'_1,\ldots,f'_k$ могли бы занулиться, что не дало бы свести $\varphi$-подстановки к $\psi$-подстановкам. \\Действительно, мы бы знали только то, что из $f'_1,\ldots,f'_k$ получаются все старшие по ширине компоненты.
    Однако они могли бы не совпасть со старшими компонентами после подстановок и линейной комбинации.
    Ведь последняя могла сократить старшие компоненты, полученные после подстановок (именно в этом месте мы существенно воспользовались однородностью подстановок), и разница между подстановками $\varphi$ и $\psi$ повлияла бы на итоговые старшие компоненты.
\end{remark}


    \section{Связь методов Гришина с гипотезой Гельфанда}\label{sec:Gelfand}
    Пусть $\Wn$ ~--- алгебра Ли формальных векторных полей, и $\Wpn$ ~--- алгебра Ли полиномиальных векторных полей.
Хорошо известно, что
\[
    \Wn \cong \prod\limits_{k=0}^{\infty} S^k V\otimes V^*, \quad \Wpn \cong \bigoplus\limits_{k=0}^{\infty} S^k V\otimes V^*
\]
Обозначим подалгебры конечного индекса:
\[
    \L_d(n) = \prod\limits_{k=d}^{\infty} S^k V\otimes V^*, \quad \Lp_d(n) = \bigoplus\limits_{k=d}^{\infty} S^k V\otimes V^*
\]
Пусть $V_{\lambda}$ ~--- произвольное неприводимое представление $\mathfrak{gl}_n$, ему соответствует $\L_0(n)$-модуль $V_{\lambda}$ и тривиальное действие на подалгебре $\L_1(n)\subset\L_0(n)$.
Обозначим индуцированный и коиндуцированный $\Wn$- и $\Wpn$- модули:
\begin{gather*}
    T_{\lambda} = \mathrm{Ind}^{\Wn}_{\L_0(n)} V_{\lambda} = U(\mathcal{W}_n^{pol}) \otimes_{U(L_0(n))} V_{\lambda} \cong \bigoplus_{k=0}^{\infty} S^k V^*\otimes V_{\lambda} \\
    \mathcal{T}_{\lambda} = \mathrm{CoInd}^{\Wn}_{\L_0(n)} V_{\lambda} = \operatorname{Hom}_{U(L_0(n))}(U(\mathcal{W}_n), V_{\lambda}) \cong \bigoplus_{k=0}^{\infty} S^k V\otimes V_{\lambda}
\end{gather*}

Методами гомологической алгебры (которые здесь будут опущены, см.\ одну из классических монографий~\cite{Fuks}) можно свести гипотезу Гельфанда~\ref{Gelfand} к более общему утверждению

\vskip 0.1in\noindent
\begin{conjecture}
    Модуль $\mathcal{T}_{\lambda_1}\otimes\ldots\otimes\mathcal{T}_{\lambda_r}$ нетеров как $\L_d(n)$-модуль.
\end{conjecture}
\vskip 0.1in\noindent

Оказывается, что для $\lambda_i=0, d=1$ данная гипотеза уже весьма нетривиальна.
В дополнение в~\cite{Feigin-Kanel-Khoroshkin} А.С.\ Хорошкин привел набросок доказательства общего случая во многом похожий на приведенный ниже подход для случая $\mathcal{T}_{0}\otimes\ldots\otimes\mathcal{T}_{0}$.

Наконец, случай $\lambda_i=0, d=1$ соответствует $\L_1(n)$-нетеровости $\mathbb{F}[x_1, \ldots, x_n]$.
Рассмотрим инфинитезимальные подстановки вида
\[
    \lambda_p: x_i \mapsto x + \varepsilon p(x_i)
\]
для $\varepsilon\to 0$.\\
Инфинитезимальность эквивалентна линеаризации по $\varepsilon$, а значит эта подстановка есть не что иное, как ключевая подстановка~\ref{eq:key-sub} из доказательства теоремы~\ref{main}.
Кроме того эта подстановка соответствует действию диагонального векторного поля
\[
    \sigma_p: p(x_1)\partial_1 + \ldots + p(x_n)\partial_n
\]

Остается заметить, что это и есть ключевая подстановка~\ref{eq:key-sub} из доказательства теоремы~\ref{main}.

    \printbibliography
\end{document}
