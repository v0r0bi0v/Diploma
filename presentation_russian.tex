%%%%%%%%%%%%%%%%%%%%%%%%%%%%%%%%%%%%%%%%%
% Beamer Presentation
% LaTeX Template
% Version 1.0 (10/11/12)
%
% This template has been downloaded from:
% http://www.LaTeXTemplates.com
%
% License:
% CC BY-NC-SA 3.0 (http://creativecommons.org/licenses/by-nc-sa/3.0/)
%
%%%%%%%%%%%%%%%%%%%%%%%%%%%%%%%%%%%%%%%%%

%----------------------------------------------------------------------------------------
%	PACKAGES AND THEMES
%----------------------------------------------------------------------------------------

\documentclass{beamer}
\usepackage[english, russian]{babel}

\mode<presentation> {

% The Beamer class comes with a number of default slide themes
% which change the colors and layouts of slides. Below this is a list
% of all the themes, uncomment each in turn to see what they look like.

%\usetheme{default}
%\usetheme{AnnArbor}
%\usetheme{Antibes}
%\usetheme{Bergen}
%\usetheme{Berkeley}
%\usetheme{Berlin}
%\usetheme{Boadilla}
%\usetheme{CambridgeUS}
%\usetheme{Copenhagen}
%\usetheme{Darmstadt}
%\usetheme{Dresden}
%\usetheme{Frankfurt}
%\usetheme{Goettingen}
%\usetheme{Hannover}
%\usetheme{Ilmenau}
%\usetheme{JuanLesPins}
%\usetheme{Luebeck}
    \usetheme{Madrid}
%\usetheme{Malmoe}
%\usetheme{Marburg}
%\usetheme{Montpellier}
%\usetheme{PaloAlto}
%\usetheme{Pittsburgh}
%\usetheme{Rochester}
%\usetheme{Singapore}
%\usetheme{Szeged}
%\usetheme{Warsaw}

% As well as themes, the Beamer class has a number of color themes
% for any slide theme. Uncomment each of these in turn to see how it
% changes the colors of your current slide theme.

%\usecolortheme{albatross}
%\usecolortheme{beaver}
%\usecolortheme{beetle}
%\usecolortheme{crane}
%\usecolortheme{dolphin}
%\usecolortheme{dove}
%\usecolortheme{fly}
%\usecolortheme{lily}
%\usecolortheme{orchid}
%\usecolortheme{rose}
%\usecolortheme{seagull}
%\usecolortheme{seahorse}
%\usecolortheme{whale}
%\usecolortheme{wolverine}

%\setbeamertemplate{footline} % To remove the footer line in all slides uncomment this line
%\setbeamertemplate{footline}[page number] % To replace the footer line in all slides with a simple slide count uncomment this line

%\setbeamertemplate{navigation symbols}{} % To remove the navigation symbols from the bottom of all slides uncomment this line
}

\newtheorem{remark}{Remark}
\newtheorem{conjecture}{Conjecture}
\newtheorem{theorem*}{Theorem}
\newtheorem{formula}{Formula}

\usepackage{graphicx} % Allows including images
\usepackage{booktabs} % Allows the use of \toprule, \midrule and \bottomrule in tables
\usepackage{textpos}
%----------------------------------------------------------------------------------------
%	TITLE PAGE 1
%----------------------------------------------------------------------------------------
\begin{document}


%\begin{frame}
%\titlepage % Print the title page as the first slide
%\end{frame}


%----------------------------------------------------------------------------------------
%	TITLE PAGE
%----------------------------------------------------------------------------------------


    \title[PI-theory]{Проблема Шпехта и Гипотеза Гельфанда} % The short title appears at the bottom of every slide, the full title is only on the title page


% Your name
    \institute[HSE] % Your institution as it will appear on the bottom of every slide, may be shorthand to save space
    {
        \\

        \centering\Large{Воробьев Иван Евгеньевич}\\
        \vspace{0.5cm}
        \centering\small{Научный руководитель, доктор физ.-мат. наук, профессор}\\ Алексей Яковлевич Канель-Белов\\
        \vspace{0.5cm}
        \centering\small{Со-руководитель, кандидат физ.-мат. наук, доцент}\\ Антон Сергеевич Хорошкин\\
        \vspace{0.5cm}
        \centering\small{Рецензент, доктор физ.-мат. наук, профессор}\\ Сергей Олегович Горчинский
    }


    \begin{document}


        \begin{frame}
            \titlepage % Print the title page as the first slide
        \end{frame}

        \begin{frame}{Strucure of the work}
            There are two non-trivial applications of PI-theory those will be presented:
            \begin{itemize}
                \item Non-linearity of free pro-$p$ groups
                \item Gelfand's conjecture
            \end{itemize}
            Structure:
            \begin{enumerate}
                \item Preliminaries for pro-$p$ structures
                \item A brief historical review
                \item Problem statement (on non-linearity of free pro-$p$ group over 2-by-2 matrices)
                \item Review of Zubkov's approach
                \item Review of Zelmanov, Ben-Ezra's approach
                \item Modification of Zelmanov, Ben-Ezra's approach for $\mathrm{char} \Delta=4$
                \item Gelfand conjecture, statement and reformulation: whether some module is Noetherian
                \item Connection between Gelfand conjecture and PI-theory
            \end{enumerate}
        \end{frame}

        \begin{frame}{Preliminaries}
            \begin{definition}
                The inverse (projective) limit of the projective system of finite groups (rings) is called a profinite group (ring).
            \end{definition}
            \begin{definition}
                The inverse limit of the projective system of $p$-groups is called pro-$p$ group.
            \end{definition}
            \begin{definition}
                Commutative Noetherian complete local ring $\Delta$ with a maximal ideal $I$ is called pro-$p$ ring if $\Delta/I$ is a finite field of characteristic $p$.
            \end{definition}
            In that case:
            \[ \Delta = \varprojlim \Delta/I^n \]


        \end{frame}
        \begin{frame}{Preliminaries}
            \begin{definition}
                Let $F$ be a free group generated by alphabet $\mathcal{S}$.
                Consider the completion $\widetilde{F}_p$ of $F$ with respect to topology, defined by all normal subgroups of a finite index $p^l$, $\forall l\in\mathbb{N}$ which have almost all generators from $\mathcal{S}$.
                Then $\widetilde{F}_p$ is called a free pro-$p$-group.
            \end{definition}
            \begin{remark}
                Here and later the completion with respect to normal subgroups means the inverse limit of the system of factorgroups.
            \end{remark}
            Let $\Delta$ be pro-$p$ ring
            \[GL_d^1(\Delta) = \mathrm{ker}\left( GL_d(\Delta) \xrightarrow{\Delta\to\Delta/I} GL_d(\Delta/I) \right)\]
            is a pro-$p$-group.
        \end{frame}

        \begin{frame}{Main problem}
            \begin{conjecture}
                A non-abelian free pro-$p$ group $\widetilde{F}_p$ cannot be continuously embedded in $GL_d^1(\Delta)$ for any pro-$p$ ring $\Delta$.
            \end{conjecture}
        \end{frame}


        \begin{frame}{Historical review}
            There are several partial results for certain $\widetilde{F}_p, \Delta, p$, that let us suppose that answer is positive for the general case:
            \begin{itemize}
                \item In 1987, A.N\ Zubkov (\cite{Zubkov}) demonstrated that for $d=2, p\neq2$ the conjecture holds true.
                \item In 1991, J.D.\ Dixon, A.\ Mann, M.P.F.\ du Sautoy, D.\ Segal (\cite{DMSD}) established the conjecture for $\Delta=\mathbb{Z}_p$, $GL_d^1(\mathbb{Z}_p)=\mathrm{ker}\left( GL_2(\mathbb{Z}_p) \xrightarrow{\mathbb{Z}_p\to\mathbb{F}_p} GL_2(\mathbb{F}_p) \right)$
                \item In 1999, utilizing the profound results of Pink (\cite{Pink}), Y.\ Barnea, M.\ Larsen (\cite{Barnea-Larsen}) proved the conjecture for $\Delta=\left( \mathbb{Z}/p\mathbb{Z} \right)[[t]]$
                \item In 2005, E.\ Zelmanov (\cite{Zelmanov}) announced that conjecture holds true for $p\gg d$.
                \item In 2020, D.\ Ben-Ezra, E.\ Zelmanov showed (\cite{Ben-Ezra-Zelmanov}) that for $d=2, p=2$ and $\mathrm{char}(\Delta)=2$  the conjecture holds true.
            \end{itemize}
        \end{frame}

        \begin{frame}{Zubkov's approach}
            \begin{theorem*}[Zubkov, 1987]
                Let $F$ be a free non-abelian pro-$p$ group, $\Delta$ is a pro-$p$ ring.
                $F$ cannot be continuously embedded in $GL^1_2(\Delta)$, when $p>2$.
            \end{theorem*}

            The first non-trivial idea is to introduce the following definition:
            \begin{definition}
                Let $F$ be a free pro-$p$ group, and $G$ be a pro-$p$ group.
                Then every $1\neq w\in F$ such that $w\in Ker(\varphi)$ for all continuous homomorphisms $\varphi: F\to G$ is called a pro-$p$ identity of G.
            \end{definition}
            Zubkov consider the natural homomorphism to algebra of generic matrices:\\
            Let $x,y \in \widetilde{F}_p$~--- generators, $\pi: x \mapsto 1+x^*, y \mapsto 1+y^*$, where $x^*, y^*$ are the generic matrices over $\mathbb{Z}_p$.
            And one can continue $\pi$ to the completion $\langle \langle x, y \rangle \rangle$, and it will map on closure of $\langle1+x^*, 1+y^*\rangle$.
        \end{frame}

        \begin{frame}{Zubkov's approach}
            Homomorphism $\pi$ is called the universal representation, and that's why:
            \begin{lemma}
                Each $1\neq z \in \ker \pi$ is a pro-$p$ identity of $GL^1_d(\Delta)$ for all pro-$p$ rings $\Delta$.
            \end{lemma}
            This theorem can be proven with a standard commutative algebra approaches.
            And this implies that it would be enough to prove
            \begin{theorem}
                The universal representation of the degree $2$ is not injective for $p\neq 2$.
            \end{theorem}
            So we need to construct the pro-$p$ identity for generic matrices.\\
        \end{frame}
        \begin{frame}{Zubkov's approach}
            Zubkov investigated the lower central series of Lie algebra of generic matrices.\\
            And he encountered a contradiction with the classical Witt's formula:
            \begin{formula}[Witt]
                Rank of $r$-th factor of the lower central series of $\widetilde{F}_p$ (as a $\mathbb{Z}_p-module$):
                \[\frac{1}{r}\sum\limits_{m \mid r} \mu(m) \cdot 2^{\frac{r}{m}}\]
            \end{formula}
        \end{frame}

        \begin{frame}{Ben-Ezra, Zelmanov's approach}
            \begin{theorem*}[Ben-Ezra, Zelmanov, 2020]
                Let $F$ be a free non-abelian pro-$2$ group, $\Delta$ is a pro-$2$ ring.
                $F$ cannot be continuously embedded in $GL^1_2(\Delta)$, when $\mathrm{char}\Delta=2$.
            \end{theorem*}
            Ben-Ezra and Zelmanov modified Zubkov's universal representation for the case $p=2$:
            analogous homomorphism to generic matrices over $\mathbb{Z}/2\mathbb{Z}$ (instead of $\mathbb{Z}_2$).

            Then the following lemma still holds and has a pretty simple proof:
            \begin{lemma}
                Each $1\neq z \in \ker \pi$ is a pro-$2$ identity of $GL^1_2(\Delta)$ for all pro-$2$ rings $\Delta$ with $\mathrm{char}\Delta=2$.
            \end{lemma}
            And the last theorem has a very hard proof:
            authors investigated Lie algebra using PI-theory approaches:
            \begin{theorem}
                The universal representation of the degree $2$ is not injective.
            \end{theorem}
        \end{frame}

        \begin{frame}{$\mathrm{char}\Delta = 4$}
            \begin{conjecture}
                Let $F$ be a free non-abelian pro-$2$ group, $\Delta$ is a pro-$2$ ring.
                $F$ cannot be continuously embedded in $GL^1_2(\Delta)$, when $\mathrm{char}\Delta=4$.
            \end{conjecture}
            We intend to prove it using the similar approaches, and believe that one can prove it even for the case $\mathrm{char}\Delta=2^l$.

            Furthermore, maybe the case $\mathrm{char}\Delta=0$ can be investigated if the above statement will be proved.
        \end{frame}

        \begin{frame}{PI-theory, preliminaries}
            Let $T$ be the endomorphism (substitution) semigroup of the free algebra $F = k\langle x_1,\ldots,x_i,\ldots\rangle$.
            \begin{definition}
                An endomorphism $\tau$ of $F$ defined by the rule $x_i \mapsto g_i, g_i \in F$, is called a substitution of type
                $(x_1,\ldots,x_i,\ldots) \mapsto (g_1,\ldots,g_i,\ldots)$.
            \end{definition}
            \begin{definition}
                $T$-space in $F$ is a vector subspace of $F$, that is closed under substitutions.
            \end{definition}
            \begin{definition}
                $T$-ideal in F is an ideal of F that is at the same time a T-space.
            \end{definition}
        \end{frame}
        \begin{frame}{PI-theory, preliminaries}
            Following theorem (the special case of Shchigolev's \cite{Shch}) is proved in author's last year coursework.
            \begin{theorem}
                Any $T$-space in algebra $k[x_1,\ldots,x_n]$ is finitely based.
            \end{theorem}
            Furthermore, one can prohibit some of the substitutions and show that $T$-spaces are finitely based using some $\widetilde{T}\subset T$

            The main idea is to use substitutions:
            \[f(x_1,\ldots,x_i,\ldots,x_n) \mapsto f(x_1, \ldots, 1 + \alpha_i P(x_i), \ldots, x_n)\]
            And then we linearize it on $\alpha_i$.
        \end{frame}
        \begin{frame}{Gelfand conjecture}
            \begin{conjecture}[Gelfand, 1970, \cite{Ge1}]
                The homology of the Lie subalgebra of finite codimension in the Lie algebra of algebraic vector fields on an affine algebraic manifold are finite-dimensional in each
                homological degree.
            \end{conjecture}
            We denote by $\mathcal{W}_n$ the Lie algebra of formal vector fields on an n-dimensional plane $V$.\\
            \[\mathcal{W}_n\simeq \prod\limits_{k=0}^{\infty}S^k V\otimes V^*\]
            The subalgebras $\prod\limits_{k=d}^{\infty}S^k V\otimes V^*$ of a finite codimension are denoted by $L_d(n)$.
        \end{frame}
        \begin{frame}{Gelfand conjecture}
            Using the classical considerations of homological algebra, one can reduce Gelfand conjecture to the following lemma:
            \begin{lemma}
                Any finitely generated $L_d(n)$-module is noetherian.
            \end{lemma}
            Then we will observe how to use Grishin's methods to prove this lemma.
        \end{frame}

        \begin{frame}{Bibliography}
            \tiny
            \begin{thebibliography}{99}


                \bibitem{Zubkov}
                A. Zubkov,
                \emph{Non-abelian free pro-p-groups cannot be represented by 2-by-2 matrices},
                \emph{Siberian Mathematical Journal},
                vol. 28, pp. 742--747,
                1987.

                \bibitem{Pink}
                R. Pink,
                \emph{Compact subgroups of linear algebraic groups},
                \emph{Journal of Algebra},
                vol. 206, pp. 438--504,
                1998.

                \bibitem{Barnea-Larsen}
                Y. Barnea and M. Larsen,
                \emph{A non-abelian free pro-p group is not linear over a local field},
                \emph{Journal of Algebra},
                vol. 214, pp. 338--341,
                1999.

                \bibitem{DMSD}
                J. Dixon, A. Mann, M. du Sautoy, and D. Segal,
                \emph{Analytic pro-p-groups},
                \emph{London Mathematical Society Lecture Note Series},
                Cambridge University Press,
                1991.

                \bibitem{Ben-Ezra-Zelmanov}
                D. Ben-Ezra and E. Zelmanov,
                \emph{On Pro-2 Identities of 2×2 Linear Groups},
                \emph{arXiv:1910.05805v2},
                2020.

                \bibitem{Zelmanov}
                E. Zelmanov,
                \emph{Infinite algebras and pro-p groups},
                \emph{Infinite groups: geometric, combinatorial and dynamical aspects},
                Progr. Math., vol. 248, pp. 403--413,
                2005.

                \bibitem{Gelfand}
                I.M. Gelfand,
                \emph{The cohomology of infinite dimensional Lie algebras; Some questions of integral geometry},
                \emph{Proceedings of ICM},
                vol. T.1, pp. 95--111,
                1970.

                \bibitem{Feigin-Kanel-Khoroshkin}
                B. Feigin, A. Kanel-Belov, and A. Khoroshkin,
                \emph{On finite dimensionality of homology of subalgebras of vector fields},
                \emph{arXiv:2211.08510v1},
                2022.

                \bibitem{Centrone-Kanel-Khoroshkin-Vorobiov}
                L. Centrone, A. Kanel-Belov, A. Khoroshkin, and I. Vorobiov,
                \emph{Specht property for systems of commutative polynomials and Gelfand conjecture},
                \emph{researchgate net},
                2022.

                \bibitem{Kemer}
                A. Kemer,
                \emph{Finite basability of identities of associative algebras},
                \emph{Algebra and Logics},
                vol. 26, no. 5, pp. 597--641,
                1987.

                \bibitem{Aljadeff-Kanel-Karasik}
                E. Aljadeff, A. Kanel-Belov, and Y. Karasik,
                \emph{Kemer’s theorem for affine PI algebras over a field of characteristic zero},
                \emph{Journal of Pure and Applied Algebra},
                vol. 220, no. 8, pp. 2771--2808,
                2016.

%                \bibitem{Procesi}
%                C. Procesi,
%                \emph{The geometry of polynomial identities},
%                \emph{Izv. Math.},
%                vol. 80, no. 5, pp. 910--953,
%                2016.

                \bibitem{Grishin}
                A. Grishin,
                \emph{On finitely based systems of generalized polynomials},
                \emph{Math. USSR-Izv.},
                vol. 37, no. 2, pp. 243--272,
                1991.

            \end{thebibliography}

        \end{frame}

%----------------------------------------------------------------------------------------


    \end{document}
