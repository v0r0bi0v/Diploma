%%%%%%%%%%%%%%%%%%%%%%%%%%%%%%%%%%%%%%%%%
% Beamer Presentation
% LaTeX Template
% Version 1.0 (10/11/12)
%
% This template has been downloaded from:
% http://www.LaTeXTemplates.com
%
% License:
% CC BY-NC-SA 3.0 (http://creativecommons.org/licenses/by-nc-sa/3.0/)
%
%%%%%%%%%%%%%%%%%%%%%%%%%%%%%%%%%%%%%%%%%

%----------------------------------------------------------------------------------------
%	PACKAGES AND THEMES
%----------------------------------------------------------------------------------------

\documentclass{beamer}
\usepackage[english, russian]{babel}

\newcommand{\R}{\ensuremath{\mathbb{R}}}
\newcommand{\Z}{\ensuremath{\mathbb{Z}}}
\newcommand{\Q}{\ensuremath{\mathbb{Q}}}
\newcommand{\LQ}{\ensuremath{\mathcal{L}_{\mathbb{Q}_p}}}
\newcommand{\LZ}{\ensuremath{\mathcal{L}_{\mathbb{Z}_p}}}
\newcommand{\LQn}{\ensuremath{\mathcal{L}^{(n)}_{\mathbb{Q}_p}}}
\newcommand{\LZn}{\ensuremath{\mathcal{L}^{(n)}_{\mathbb{Z}_p}}}
\newcommand{\LQcentralr}{\ensuremath{\mathcal{L}_{\mathbb{Q}_p, n}}}
\newcommand{\Sbf}{\ensuremath{\mathbf{S}}}
\newcommand{\rank}{\ensuremath{\mathrm{rank}}}
\newcommand{\xoo}{\ensuremath{x_{1,1}}}
\newcommand{\xod}{\ensuremath{x_{1,2}}}
\newcommand{\xdo}{\ensuremath{x_{2,1}}}
\newcommand{\xdd}{\ensuremath{x_{2,2}}}
\newcommand{\yoo}{\ensuremath{y_{1,1}}}
\newcommand{\yod}{\ensuremath{y_{1,2}}}
\newcommand{\ydo}{\ensuremath{y_{2,1}}}
\newcommand{\ydd}{\ensuremath{y_{2,2}}}
\newcommand{\Wpn}{\ensuremath{\mathcal{W}^{\mathrm{pol}}_n}}
\newcommand{\Wn}{\ensuremath{\mathcal{W}_n}}
\newcommand{\Wp}{\ensuremath{\mathcal{W}^{\mathrm{pol}}}}
\newcommand{\W}{\ensuremath{\mathcal{W}}}
\newcommand{\Lp}{\ensuremath{\mathcal{L}^{\mathrm{pol}}}}
\newcommand{\SQ}{\ensuremath{S_{\mathbb{Q}_p}}}
\newcommand{\SQn}{\ensuremath{S_{\mathbb{Q}_p, n}}}
\newcommand{\SQr}{\ensuremath{S_{\mathbb{Q}_p, r}}}
\newcommand{\SQo}{\ensuremath{S_{\mathbb{Q}_p, 1}}}
\newcommand{\trace}{\ensuremath{\mathrm{trace}}}


\renewcommand{\L}{\ensuremath{\mathcal{L}}}
\renewcommand{\char}{\ensuremath{\mathrm{char}}}

\mode<presentation> {

% The Beamer class comes with a number of default slide themes
% which change the colors and layouts of slides. Below this is a list
% of all the themes, uncomment each in turn to see what they look like.

%\usetheme{default}
%\usetheme{AnnArbor}
%\usetheme{Antibes}
%\usetheme{Bergen}
%\usetheme{Berkeley}
%\usetheme{Berlin}
%\usetheme{Boadilla}
%\usetheme{CambridgeUS}
%\usetheme{Copenhagen}
%\usetheme{Darmstadt}
%\usetheme{Dresden}
%\usetheme{Frankfurt}
%\usetheme{Goettingen}
%\usetheme{Hannover}
%\usetheme{Ilmenau}
%\usetheme{JuanLesPins}
%\usetheme{Luebeck}
    \usetheme{Madrid}
%\usetheme{Malmoe}
%\usetheme{Marburg}
%\usetheme{Montpellier}
%\usetheme{PaloAlto}
%\usetheme{Pittsburgh}
%\usetheme{Rochester}
%\usetheme{Singapore}
%\usetheme{Szeged}
%\usetheme{Warsaw}

% As well as themes, the Beamer class has a number of color themes
% for any slide theme. Uncomment each of these in turn to see how it
% changes the colors of your current slide theme.

%\usecolortheme{albatross}
%\usecolortheme{beaver}
%\usecolortheme{beetle}
%\usecolortheme{crane}
%\usecolortheme{dolphin}
%\usecolortheme{dove}
%\usecolortheme{fly}
%\usecolortheme{lily}
%\usecolortheme{orchid}
%\usecolortheme{rose}
%\usecolortheme{seagull}
%\usecolortheme{seahorse}
%\usecolortheme{whale}
%\usecolortheme{wolverine}

%\setbeamertemplate{footline} % To remove the footer line in all slides uncomment this line
%\setbeamertemplate{footline}[page number] % To replace the footer line in all slides with a simple slide count uncomment this line

%\setbeamertemplate{navigation symbols}{} % To remove the navigation symbols from the bottom of all slides uncomment this line
}

\newtheorem{remark}{Remark}
\newtheorem{conjecture}{Conjecture}
\newtheorem{theorem*}{Theorem}
\newtheorem{formula}{Formula}

\usepackage{graphicx} % Allows including images
\usepackage{booktabs} % Allows the use of \toprule, \midrule and \bottomrule in tables
\usepackage{textpos}
%----------------------------------------------------------------------------------------
%	TITLE PAGE 1
%----------------------------------------------------------------------------------------
\begin{document}


%\begin{frame}
%\titlepage % Print the title page as the first slide
%\end{frame}


%----------------------------------------------------------------------------------------
%	TITLE PAGE
%----------------------------------------------------------------------------------------


    \title[PI-theory]{Проблема Шпехта и Гипотеза Гельфанда} % The short title appears at the bottom of every slide, the full title is only on the title page


% Your name
    \institute[HSE] % Your institution as it will appear on the bottom of every slide, may be shorthand to save space
    {
        \\

        \centering\Large{Воробьев Иван Евгеньевич}\\
        \vspace{0.5cm}
        \centering\small{Научный руководитель, доктор физ.-мат. наук, профессор}\\ Алексей Яковлевич Канель-Белов\\
        \vspace{0.5cm}
        \centering\small{Соруководитель, кандидат физ.-мат. наук, доцент}\\ Антон Сергеевич Хорошкин\\
        \vspace{0.5cm}
        \centering\small{Рецензент, доктор физ.-мат. наук, профессор}\\ Сергей Олегович Горчинский
    }


    \begin{document}


        \begin{frame}
            \titlepage % Print the title page as the first slide
        \end{frame}

        \begin{frame}{Strucure of the work}
            Мы рассмотрим два применения PI-теории:
            \begin{itemize}
                \item Нелинейность свободных про-$p$ групп
                \item Гипотеза Гельфанда
            \end{itemize}
            Структура:
            \begin{enumerate}
                \item Предварительные сведения
                \item Историческая справка
                \item Постановка задачи (о нелинейности свободной про-$p$ группы)
                \item Обзор подхода А.Н.\ Зубкова ($d=2, p>2$)
                \item Обзор подхода Бена-Эзры\textemdash Зельманова ($p=2, d=2, \mathrm{char}(\Delta)=2$)
                \item Случай $p=2, d=2, \mathrm{char}(\Delta)=4$
                \item Методы Гришина
                \item Гипотеза Гельфанда
                \item Связь гипотезы Гельфанда с методами А.В.\ Гришина
            \end{enumerate}
        \end{frame}

        \begin{frame}{Preliminaries}
            \begin{definition}
                Обратный (проективный) предел проективной системы конечных групп называется проконечной группой.
            \end{definition}
            \begin{definition}
                Обратный (проективный) предел проективной системы конечных $p$-групп называется про-$p$ группой.
            \end{definition}
            \begin{definition}
                Коммутативное нетерово $I$-полное локальное кольцо $\Delta$ с максимальным идеалом $I$ называется про-$p$ кольцом, если $\Delta/I$ конечное поле характеристики $p$.
            \end{definition}
            \[ \Delta = \varprojlim \Delta/I^n \]


        \end{frame}
        \begin{frame}{Preliminaries}
            \begin{definition}
                Пусть $F$ свободная группа порожденная алфавитом $\mathcal{S}$.
                Рассмотрим пополнение $\widetilde{F}_p$ группы $F$ относительно топологии, определенной всеми нормальными подгруппами индекса $p^l$, $\forall l\in\mathbb{N}$.
                Тогда $\widetilde{F}_p$ называется свободной про-$p$ группой.
            \end{definition}
            \begin{remark}
                Здесь и далее под подобным пополнением мы имеем в виду обратный предел факторгрупп.
            \end{remark}
            Пусть $\Delta$ про-$p$ кольцо.
            \[GL_d^1(\Delta) = \mathrm{ker}\left( GL_d(\Delta) \xrightarrow{\Delta\to\Delta/I} GL_d(\Delta/I) \right)\]
            является про-$p$ группой.
        \end{frame}

        \begin{frame}{Основная гипотеза}
            \begin{conjecture}
                Некоммутативная свободная про-$p$ группа $\widetilde{F}_p$ не может быть непрерывно вложена в $GL_d^1(\Delta)$ для любого про-$p$ кольца $\Delta$.
            \end{conjecture}
        \end{frame}


        \begin{frame}{Историческая справка}
            Существует множество частичных результатов для различных $\widetilde{F}_p, \Delta, p$, которые дают надежду на положительный результат и в общем случае:
            \begin{itemize}
                \item В 1987, А.Н.\ Зубков (\cite{Zubkov}) доказал гипотезу для $d=2, p\neq2$.
                \item В 1991, J.D.\ Dixon, A.\ Mann, M.P.F.\ du Sautoy, D.\ Segal (\cite{DMSD}) доказали гипотезу для $\Delta=\mathbb{Z}_p$, $GL_d^1(\mathbb{Z}_p)=\mathrm{ker}\left( GL_2(\mathbb{Z}_p) \xrightarrow{\mathbb{Z}_p\to\mathbb{F}_p} GL_2(\mathbb{F}_p) \right)$
                \item В 1999, используя глубокие результаты Пинка (\cite{Pink}), Y.\ Barnea, M.\ Larsen (\cite{Barnea-Larsen}) доказали гипотезу для $\Delta=\left( \mathbb{Z}/p\mathbb{Z} \right)[[t]]$.
                \item В 2005, E.\ Zelmanov (\cite{Zelmanov1}) анонсировал доказательство гипотезы для $p\gg d$.
                \item В 2020, D.\ Ben-Ezra, E.\ Zelmanov доказали (\cite{Ben-Ezra-Zelmanov}) гипотезу для $d=2, p=2$ и $\mathrm{char}(\Delta)=2$.
            \end{itemize}
        \end{frame}

        \begin{frame}{Подход Зубкова}
            \begin{theorem}[А.Н.\ Зубков, 1989]
                \label{thm:Zubkov-main}
                Некоммутативная свободная про-$p$ группа не может быть непрерывно вложена в $GL^1_2(\Delta)$ для $p\neq 2$.
            \end{theorem}

            Зубков рассматривает естественный гомоморфизм в алгебру общих матриц:\\
            Пусть $x,y \in \widetilde{F}_p$~--- образующие, $\pi: x \mapsto 1+x^*, y \mapsto 1+y^*$, где $x^*, y^*$ общие матрицы $\mathbb{Z}_p$.
            Можно продолжить $\pi$ на замыкание $\langle \langle x, y \rangle \rangle$, и оно отобразится на замыкание $\langle1+x^*, 1+y^*\rangle$.
        \end{frame}

        \begin{frame}{Общие матрицы}
            \begin{align*}
                & S = \Z_{p} [[ x_{1,1}, y_{1,1}, \ldots, x_{2,2}, y_{2,2} ]]\\
                & S_k = \{\sum\limits_k^{\infty} f_i \}\\
                & B_{k,n} = S \cdot p^n + S_k\\
            \end{align*}
        \end{frame}

        \begin{frame}{Zubkov's approach}
            Гомоморфизм $\pi$ называется универсальным представлением:
            \begin{theorem}[А.Н.\ Зубков, 1987]
                \label{thm:zubkov-universal}
                Пусть $F$ ~--- свободная про-$p$ группа порожденная $x, y$.
                Если существует инъективный непрерывный гомоморфизм $\varphi: F \to GL^1_2(\Delta)$, то и универсальное представление $\pi$ инъективно.
            \end{theorem}

            \begin{theorem}
                Универсальное представление не инъективно.
            \end{theorem}
        \end{frame}

        \begin{frame}{Подход Зубкова}
            Обозначения:
            \begin{itemize}
                \item $\mathbf{S}$ ~--- кольцо степенных рядов от общих матриц $X, Y$ над $\Z_p$
                \item $\LQ$ ~--- алгебра Ли порожденная общими матрицами $X, Y$ над $\Q_p$
                \item $\LZ$ ~--- алгебра Ли порожденная общими матрицами $X, Y$ над $\Z_p$
                \item $\LQ^{(n)}$ ~--- векторное пространство над $\Q_p$ однородных элементов степени $n$ в алгебре $\LQ$.
                \item $\LZ^{(n)}$ ~---  $\Z_p$-модуль однородных элементов степени $n$ в алгебре $\LZ$.
                \item Для $g\in 1 + \ker{(M_2(S) \to M_2(S / S_1))}$: $\min{g}$ ~--- однородная компонента наименьшей ненулевой степени (можно записать $g=1 + a_n + a_{n+1} + \ldots$)
                \item Будем записывать коммутатор веса $n$ следующим образом $[l_1, \ldots, l_n] = [[l_1, l_2, \ldots, l_{n-1}], l_n]$
            \end{itemize}
        \end{frame}
        \begin{frame}{Zubkov's approach}
            Zubkov investigated the lower central series of Lie algebra of generic matrices.\\
            And he encountered a contradiction with the classical Witt's formula:
            \begin{formula}[Witt]
                Rank of $r$-th factor of the lower central series of $\widetilde{F}_p$ (as a $\mathbb{Z}_p-module$):
                \[\frac{1}{r}\sum\limits_{m \mid r} \mu(m) \cdot 2^{\frac{r}{m}}\]
            \end{formula}
        \end{frame}

        \begin{frame}{Ben-Ezra, Zelmanov's approach}
            \begin{theorem*}[Ben-Ezra, Zelmanov, 2020]
                Let $F$ be a free non-abelian pro-$2$ group, $\Delta$ is a pro-$2$ ring.
                $F$ cannot be continuously embedded in $GL^1_2(\Delta)$, when $\mathrm{char}\Delta=2$.
            \end{theorem*}
            Ben-Ezra and Zelmanov modified Zubkov's universal representation for the case $p=2$:
            analogous homomorphism to generic matrices over $\mathbb{Z}/2\mathbb{Z}$ (instead of $\mathbb{Z}_2$).

            Then the following lemma still holds and has a pretty simple proof:
            \begin{lemma}
                Each $1\neq z \in \ker \pi$ is a pro-$2$ identity of $GL^1_2(\Delta)$ for all pro-$2$ rings $\Delta$ with $\mathrm{char}\Delta=2$.
            \end{lemma}
            And the last theorem has a very hard proof:
            authors investigated Lie algebra using PI-theory approaches:
            \begin{theorem}
                The universal representation of the degree $2$ is not injective.
            \end{theorem}
        \end{frame}

        \begin{frame}{$\mathrm{char}\Delta = 4$}
            \begin{conjecture}
                Let $F$ be a free non-abelian pro-$2$ group, $\Delta$ is a pro-$2$ ring.
                $F$ cannot be continuously embedded in $GL^1_2(\Delta)$, when $\mathrm{char}\Delta=4$.
            \end{conjecture}
            We intend to prove it using the similar approaches, and believe that one can prove it even for the case $\mathrm{char}\Delta=2^l$.

            Furthermore, maybe the case $\mathrm{char}\Delta=0$ can be investigated if the above statement will be proved.
        \end{frame}

        \begin{frame}{PI-theory, preliminaries}
            Let $T$ be the endomorphism (substitution) semigroup of the free algebra $F = k\langle x_1,\ldots,x_i,\ldots\rangle$.
            \begin{definition}
                An endomorphism $\tau$ of $F$ defined by the rule $x_i \mapsto g_i, g_i \in F$, is called a substitution of type
                $(x_1,\ldots,x_i,\ldots) \mapsto (g_1,\ldots,g_i,\ldots)$.
            \end{definition}
            \begin{definition}
                $T$-space in $F$ is a vector subspace of $F$, that is closed under substitutions.
            \end{definition}
            \begin{definition}
                $T$-ideal in F is an ideal of F that is at the same time a T-space.
            \end{definition}
        \end{frame}
        \begin{frame}{PI-theory, preliminaries}
            Following theorem (the special case of Shchigolev's \cite{Shch}) is proved in author's last year coursework.
            \begin{theorem}
                Any $T$-space in algebra $k[x_1,\ldots,x_n]$ is finitely based.
            \end{theorem}
            Furthermore, one can prohibit some of the substitutions and show that $T$-spaces are finitely based using some $\widetilde{T}\subset T$

            The main idea is to use substitutions:
            \[f(x_1,\ldots,x_i,\ldots,x_n) \mapsto f(x_1, \ldots, 1 + \alpha_i P(x_i), \ldots, x_n)\]
            And then we linearize it on $\alpha_i$.
        \end{frame}
        \begin{frame}{Gelfand conjecture}
            \begin{conjecture}[Gelfand, 1970, \cite{Ge1}]
                The homology of the Lie subalgebra of finite codimension in the Lie algebra of algebraic vector fields on an affine algebraic manifold are finite-dimensional in each
                homological degree.
            \end{conjecture}
            We denote by $\mathcal{W}_n$ the Lie algebra of formal vector fields on an n-dimensional plane $V$.\\
            \[\mathcal{W}_n\simeq \prod\limits_{k=0}^{\infty}S^k V\otimes V^*\]
            The subalgebras $\prod\limits_{k=d}^{\infty}S^k V\otimes V^*$ of a finite codimension are denoted by $L_d(n)$.
        \end{frame}
        \begin{frame}{Gelfand conjecture}
            Using the classical considerations of homological algebra, one can reduce Gelfand conjecture to the following lemma:
            \begin{lemma}
                Any finitely generated $L_d(n)$-module is noetherian.
            \end{lemma}
            Then we will observe how to use Grishin's methods to prove this lemma.
        \end{frame}

        \begin{frame}{Библиография}
            \tiny
            \begin{thebibliography}{99}
                \bibitem{Sanov}
                I. Sanov,
                \emph{The property of one free group representation},
                \emph{Doklady Akademii Nauk USSR},
                vol. 57, no. 7, pp. 657--659,
                1947.

                \bibitem{Kanel}
                A. Kanel-Belov,
                \emph{Local finite basability and local representability of varieties of associative rings},
                \emph{Doklady Akademii Nauk},
                vol. 432, no. 6, pp. 727--731,
                2010.

                \bibitem{Zubkov}
                A. Zubkov,
                \emph{Non-abelian free pro-p-groups cannot be represented by 2-by-2 matrices},
                \emph{Siberian Mathematical Journal},
                vol. 28, pp. 742--747,
                1987.

                \bibitem{Pink}
                R. Pink,
                \emph{Compact subgroups of linear algebraic groups},
                \emph{Journal of Algebra},
                vol. 206, pp. 438--504,
                1998.

                \bibitem{Barnea-Larsen}
                Y. Barnea and M. Larsen,
                \emph{A non-abelian free pro-p group is not linear over a local field},
                \emph{Journal of Algebra},
                vol. 214, pp. 338--341,
                1999.

                \bibitem{DMSD}
                J. Dixon, A. Mann, M. du Sautoy, and D. Segal,
                \emph{Analytic pro-p-groups},
                \emph{London Mathematical Society Lecture Note Series},
                Cambridge University Press,
                1991.

                \bibitem{Ben-Ezra-Zelmanov}
                D. Ben-Ezra and E. Zelmanov,
                \emph{On Pro-2 Identities of 2×2 Linear Groups},
                \emph{arXiv:1910.05805v2},
                2020.

                \bibitem{Zelmanov1}
                E. Zelmanov,
                \emph{Infinite algebras and pro-p groups},
                \emph{Infinite groups: geometric, combinatorial and dynamical aspects},
                Progr. Math.,
                vol. 248, pp. 403--413,
                2005.

                \bibitem{Zelmanov2}
                E. Zelmanov,
                \emph{Groups with identities},
                \emph{Note. Mat.},
                vol. 36, pp. 101--113,
                2016.

                \bibitem{Gelfand}
                I.M. Gelfand,
                \emph{The cohomology of infinite dimensional Lie algebras; Some questions of integral geometry},
                \emph{Proceedings of ICM},
                vol. T.1, p. 106,
                1970.

                \bibitem{Feigin-Kanel-Khoroshkin}
                B. Feigin, A. Kanel-Belov, and A. Khoroshkin,
                \emph{On finite dimensionality of homology of subalgebras of vector fields},
                \emph{arXiv:2211.08510v1},
                2022.
            \end{thebibliography}

        \end{frame}

        \begin{frame}{Библиография}
            \tiny
            \begin{thebibliography}{99}
                \bibitem{Centrone-Kanel-Khoroshkin-Vorobiov}
                L. Centrone, A. Kanel-Belov, A. Khoroshkin, and I. Vorobiov,
                \emph{Specht property for systems of commutative polynomials and Gelfand conjecture},
                \url{https://www.researchgate.net/publication/355916110_Gelfand_conjecture_and_the_method_of_proof_of_Specht_problem},
                2022.

                \bibitem{Kemer}
                A. Kemer,
                \emph{Finite basability of identities of associative algebras},
                \emph{Algebra and Logics},
                vol. 26, no. 5, pp. 597--641,
                1987.

                \bibitem{Procesi}
                C. Procesi,
                \emph{The geometry of polynomial identities},
                \emph{Izv. Math.},
                vol. 80, no. 5, pp. 910--953,
                2016.

                \bibitem{Grishin}
                A. Grishin,
                \emph{On finitely based systems of generalized polynomials},
                \emph{Math. USSR-Izv.},
                vol. 37, no. 2, pp. 243--272,
                1991.

                \bibitem{Shchigolev}
                V. Shchigolev,
                \emph{Finite-basis property of T-spaces over fields of characteristic zero},
                \emph{Izv. Ross. Akad. Nauk, Ser. Mat.},
                vol. 65, no. 5, pp. 1041--1071,
                2001.

                \bibitem{Lubotzky}
                A. Lubotzky,
                \emph{Combinatorial group theory for PRO-p groups},
                \emph{Pure and Applied Algebra},
                vol. 25, pp. 311--325,
                1982.

                \bibitem{SimpleKemer}
                E. Aljadeff, A. Kanel-Belov, and Y. Karasik,
                \emph{Kemer’s theorem for affine PI algebras over a field of characteristic zero},
                \emph{Pure and Applied Algebra},
                vol. 220, pp. 2771--2808,
                2016.

                \bibitem{Grishin2}
                A. Grishin,
                \emph{On finitely based systems of generalized polynomials},
                \emph{Math. USSR-Izv.},
                vol. 37, no. 2, pp. 243--272,
                1991.

                \bibitem{GrishinSchigolev}
                A. Grishin and V. Shchigolev,
                \emph{T-spaces and their applications},
                \emph{Math. Sci., New York},
                vol. 134, no. 1, pp. 1799--1878,
                2004.

                \bibitem{Lie}
                I. Benediktovich and A. Zalesskii,
                \emph{T-ideals of free Lie algebras with polynomial growth of a sequence of codimensions},
                \emph{Proceedings of the National Academy of Sciences of Belarus. Series of Physical-Mathematical Sciences},
                vol. 3, pp. 5--10,
                1980.


            \end{thebibliography}
        \end{frame}

        \begin{frame}{Библиография}
            \tiny
            \begin{thebibliography}{99}
                \bibitem{Jordan}
                A. Vais and E. Zelmanov,
                \emph{Kemer’s theorem for finitely generated Jordan algebras},
                \emph{Izv. Vyssh. Uchebn. Zved. Mat.},
                vol. 33, no. 6, pp. 42--51,
                1989. Note: Translation: Soviet Math. (Iz. VUZ) 33(6) (1989), 38--47.

                \bibitem{Super}
                L. Centrone, A. Estrada, and A. Ioppolo,
                \emph{On PI-algebras with additional structures: rationality of Hilbert series and Specht’s problem},
                \emph{J. Algebra},
                vol. 592, pp. 300--356,
                2022.

                \bibitem{ConterKanel}
                A. Kanel-Belov,
                \emph{Counterexamples to the Specht problem},
                \emph{Sb. Math.},
                vol. 191, no. 3, pp. 13--24,
                2000. Note: Translation: Sb. Math. 131(3-4) (2000), 329--340.

                \bibitem{ConterGrishin}
                A. Grishin,
                \emph{Examples of T-spaces and T-ideals over a field of characteristic 2 without the finite basis property},
                \emph{Fundam. Prikl. Mat.},
                vol. 5, no. 1, pp. 101--118,
                1999.

                \bibitem{ConterShchigolev}
                V. Shchigolev,
                \emph{Examples of infinitely based T-ideals},
                \emph{Fundam. Prikl. Mat.},
                vol. 5, no. 1, pp. 307--312,
                1999.

                \bibitem{GradedKanel}
                E. Aljadeff and A. Kanel-Belov,
                \emph{Representability and Specht problem for G-graded algebras},
                \emph{Adv. Math.},
                vol. 225, no. 5, pp. 2391--2428,
                2010.

                \bibitem{GradedSviridova}
                I. Sviridova,
                \emph{Identities of pi-algebras graded by a finite abelian group},
                \emph{Comm. Algebra},
                vol. 39, no. 9, pp. 3462--3490,
                2011.

                \bibitem{Fuks}
                D. B. Fuks,
                \emph{Cohomology of Infinite-Dimensional Lie Algebras},
                Springer Science \& Business Media,
                2012.
            \end{thebibliography}
        \end{frame}

%----------------------------------------------------------------------------------------


    \end{document}
