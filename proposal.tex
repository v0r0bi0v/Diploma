\documentclass[12pt,a4paper]{article}
\usepackage[utf8]{inputenc}
\usepackage[english]{babel}
\usepackage[svgnames,table]{xcolor}
\usepackage{enumitem, enumerate, changepage, enumitem, adjustbox, ragged2e, longtable, setspace, hhline, multicol, tabto, float, multirow, makecell, amsmath, amsfonts,amssymb,amsthm, gensymb, young, booktabs, array, paralist, verbatim, subfig, fancyhdr, indentfirst, misccorr, graphicx, calc, contour, ulem, tikz, epsfig, epstopdf, titling, url, array, tikz-cd, latexsym, soul, extarrows, titlesec, sectsty, expdlist}
\usepackage{lipsum}
\usepackage[toc,page]{appendix}
\usepackage[hidelinks]{hyperref}
\usepackage[vcentermath]{youngtab}
\usepackage[all]{xy}
\usepackage{mathtools}
\newtheorem{theorem}{Theorem}[subsection]
\newtheorem{lemma}{Lemma}[subsection]
\newtheorem{remark}{Remark}
\newtheorem*{theorem*}{Theorem}
\newtheorem*{conjecture*}{Conjecture}
\newtheorem{definition}{Definition}[subsection]
\setlistdepth{9}
\renewlist{enumerate}{enumerate}{9}
\setlist[enumerate,1]{label=\arabic*)}
\setlist[enumerate,2]{label=\alph*)}
\setlist[enumerate,3]{label=(\roman*)}
\setlist[enumerate,4]{label=(\arabic*)}
\setlist[enumerate,5]{label=(\Alph*)}
\setlist[enumerate,6]{label=(\Roman*)}
\setlist[enumerate,7]{label=\arabic*}
\setlist[enumerate,8]{label=\alph*}
\setlist[enumerate,9]{label=\roman*}

\renewlist{itemize}{itemize}{9}
\setlist[itemize]{label=$\cdot$}
\setlist[itemize,1]{label=\textbullet}
\setlist[itemize,2]{label=$\circ$}
\setlist[itemize,3]{label=$\ast$}
\setlist[itemize,4]{label=$\dagger$}
\setlist[itemize,5]{label=$\triangleright$}
\setlist[itemize,6]{label=$\bigstar$}
\setlist[itemize,7]{label=$\blacklozenge$}
\setlist[itemize,8]{label=$\prime$}

\setlength{\topsep}{0pt}\setlength{\parindent}{0pt}
\renewcommand{\arraystretch}{1.3}


\begin{document}
    \thispagestyle{empty}

%%
%% Title page
%%
    \begin{center}
    {\scshape Federal State Autonomous Educational Institution\\
    for the Higher Education\\
    National Research University ``Higher School of Economics''\\[1ex]
    Faculty of Mathematics\par}

        \par\vfill

        \textbf{\large Vorobiov Ivan Evgenievich}

        \vspace{1.5cm}

        {\Large\bfseries
        Specht Problem and Gelfand Conjecture\\
        Проблема Шпехта и гипотеза Гельфанда
        \par}

        \vspace{1.5cm}

        \textbf{\large Bachelor's thesis}


        \vspace{1cm}

        Field of study: 01.03.01 --- Mathematics,\\[1ex]
        Degree programme: bachelor's educational programme ``Mathematics''


        \par\vfill
        \noindent\parbox[t]{0.48\textwidth}{%
            Reviewer:\\[3pt]
            academic degree\\
            full name
        }\hspace{0.04\textwidth}\parbox[t]{0.48\textwidth}{%
            Scientific advisor:\\[3pt]
            Candidate of Sciences\\
            Anton Sergeevich Khoroshkin\\[3pt]
%% Uncomment if needed
            Co-advisor:\\[3pt]
            Doctor of Sciences, professor\\
            Alexei Yakovlevich Kanel-Belov\\
        }%
        \par\vfill\vfill
        \textbf{Moscow 2024}
    \end{center}
    \thispagestyle{empty}
    \pagebreak
%%
%% ===========================================================================
%%

    \begin{abstract}
        Let $F$ be a free pro-$p$ non-abelian group, and let $\Delta$ represent a commutative Noetherian complete local ring with maximal ideal $I$ such that
        $\mathrm{char}(\Delta/I)=p$.\\
        We define the group
        \[GL_2^1(\Delta) = \mathrm{ker}\left( GL_2(\Delta) \xrightarrow{\Delta\to\Delta/I} GL_2(\Delta/I) \right)\]
        A.N.\ Zubkov proved that $F$ cannot be continuously embedded in $GL_2^1(\Delta)$ for $p\neq 2$.\\
        D.\ Ben-Ezra and E.\ Zelmanov further established that this embedding is not possible for $p = 2$ and $\mathrm{char}(\Delta) = 2$.\\
        In this paper we aim to extend this result for $\mathrm{char}(\Delta)=4$.\\
        In the second part we will investigate the connection between PI-theory and the old-standing Gelfand conjecture.
    \end{abstract}


    \section{Introduction}

    \subsection{On non-linearity of free non-abelian pro-$p$ groups}
    The problem of linearity of topological groups is a natural one and has studied for many years.
    It is well known that discrete free groups are linear (\cite{Sanov}).
    Moreover, they can be embedded in $GL_2(\mathbb{Z})$.

    So it's also quite natural to inquire whether a free pro-$p$ non-abelian group is linear.

    \vskip 0.1in\noindent
    \begin{definition}
        Commutative Noetherian complete local ring $\Delta$ with a maximal ideal $I$ is called pro-$p$ ring if $\Delta/I$ is a finite field of characteristic $p$.
    \end{definition}
    \vskip 0.1in\noindent

    Consider the congruence subgroup:
    \[GL_2^1(\Delta) = \mathrm{ker}\left( GL_2(\Delta) \xrightarrow{\Delta\to\Delta/I} GL_2(\Delta/I) \right)\]
    One can see that $GL_2^1(\Delta)$ is a pro-p-group.
    Thus, the main conjecture of this theory can be formulated as follows:
    \vskip 0.1in\noindent
    \begin{conjecture*}
        Non-abelian free pro-$p$ group cannot be continuously embedded in $GL_d^1(\Delta)$ for any pro-$p$ ring $\Delta$.
    \end{conjecture*}
    \vskip 0.1in\noindent

    There have been a lot of partial results for certain $\Delta, d$ and $p$.
    Let us list them out.
    \begin{itemize}
        \item In 1987, A.N\ Zubkov (\cite{Zubkov}) demonstrated that for $d=2, p\neq2$ the conjecture holds true.
        \item In 1999, utilizing the profound results of Pink (\cite{Pink}), Y.\ Barnea, M.\ Larsen (\cite{Barnea-Larsen}) proved the conjecture for $\Delta=\left( \mathbb{Z}/p\mathbb{Z} \right)[[t]]$
        \item In 1991, J.D.\ Dixon, A.\ Mann, M.P.F.\ du Sautoy, D.\ Segal (\cite{DMSD}) established the conjecture for $\Delta=\mathbb{Z}_p$, $GL_d^1(\mathbb{Z}_p)=\mathrm{ker}\left( GL_2(\mathbb{Z}_p) \xrightarrow{\mathbb{Z}_p\to\mathbb{F}_p} GL_2(\mathbb{F}_p) \right)$
        \item In 2020, D.\ Ben-Ezra, E.\ Zelmanov showed (\cite{Ben-Ezra-Zelmanov}) that for $d=2, p=2$ and $\mathrm{char}(\Delta)=2$  the conjecture holds true.
        \item In 2005, E.\ Zelmanov (\cite{Zelmanov1},~\cite{Zelmanov2}) announced that conjecture holds true for $p\gg d$.
    \end{itemize}


    One can observe that this subject has been extensively researched by many mathematicians.
    Therefore, first of all, we will provide a review of their methods.

    We will primarily focus on Zubkov's approach, as well as Ben-Ezra and Zelmanov's methods.
    Zubkov's proof based on standard approaches of commutative algebra and the idea of generic matrices.
    Zelmanov and Ben-Ezra adopted Zubkov's method for the case $p=2$ using the trace identities which dates back to polynomial identities theory (PI-theory for short).

    Additionally, we intend to expand Zelmanov and Ben-Ezra's approach for the case where $d=2, p=2$ and $\mathrm{char}(\Delta)=4$.
    Moreover, we anticipate that extending it to cases where $\mathrm{char}(\Delta)=2^l$ and possibly even $\mathrm{char}(\Delta)=0$ should be relatively straightforward.

    \subsection{Gelfand conjecture}
    In 2022 the remarkable connection between PI-theory (to be more precisely Grishin's methods) and Gelfand Conjecture stated at ICM’70 (see \cite{Gelfand}) was found.
    \vskip 0.1in\noindent
    \begin{conjecture*}[Gelfand]
        The homology of the Lie subalgebra of finite codimension in the Lie algebra of algebraic vector fields on an affine algebraic manifold are finite-dimensional in each
        homological degree.
    \end{conjecture*}
    \vskip 0.1in\noindent
    This interesting connection was found during joint conversation between A.S.\ Khoroshkin, A.Ya. Kanel-Belov and with some assistance from the author.
    One can find Khoroshkin's sketch in~\cite{Feigin-Kanel-Khoroshkin},~\cite{Centrone-Kanel-Khoroshkin-Vorobiov}.
    Additionally, we will incorporate findings from the author's coursework on finitely based $T$-spaces of commutative polynomials from the previous year.


    \section{Preliminaries}

    \subsection{Profinite Objects}
    Let us provide a brief overview of the classical definitions in the theory of profinite groups.
    \vskip 0.1in\noindent
    \begin{definition}
        Inverse (projective) limit of finite groups is called a profinite group.
        In the case of finite $p$-groups we obtain pro-$p$ group.
    \end{definition}
    \vskip 0.1in\noindent
    It is clear that profinite groups can be endowed with the topology induced by the Tikhonov's product topology.

    \vskip 0.1in\noindent
    \begin{definition}
        The free pro-$p$ group $F_p(X)$ is the completion of the discrete free group $F(X)$ with respect to a topology defined by all normal subgroups $N \subseteq F(X)$ whose indices are equal to the order of $p$ and which contain almost all generators of $F(X)$.
    \end{definition}
    \vskip 0.1in\noindent

    One can also define the free pro-$p$ group in a classical manner using the universal property in the category of pro-$p$ groups.

    Also note that if $\Delta$ is a pro-$p$ ring as defined in introduction, and $I$ is a maximal ideal, then
    \[ \Delta = \varprojlim \Delta/I^n \]

    \subsection{PI-theory}
    Now, let's introduce some basic definitions of PI-theory.

    Let $k$ be a field of characteristic zero and $F = k\langle x_1,\ldots,x_i,\ldots\rangle$ be a free, countably generated, associative algebra over a field $k$ and $T$ be the endomorphism (substitution) semigroup of $F$. $X = \{ x_1,\ldots,x_i,\ldots\}$\\
    Now let us give some classical definitions.

    \vskip 0.1in\noindent
    \begin{definition}
        An endomorphism $\tau$ of $F$ defined by the rule $x_i \mapsto g_i, g_i \in F$, is called a substitution of type
        $(x_1,\ldots,x_i,\ldots) \mapsto (g_1,\ldots,g_i,\ldots)$.
    \end{definition}
    \vskip 0.1in\noindent

    \vskip 0.1in\noindent
    \begin{definition}
        $T$-space in $F$ is a vector subspace of $F$, that is closed under substitutions.
    \end{definition}
    \vskip 0.1in\noindent

    \vskip 0.1in\noindent
    \begin{definition}
        $T$-ideal in $F$ is an ideal of F that is at the same time a $T$-space.
    \end{definition}
    \vskip 0.1in\noindent

    \vskip 0.1in\noindent
    \begin{definition}
        We say that a $T$-space $M$ is finitely based if there exists a finite subset $B\subset M$ such that T-space generated by $B$ coincides with $M$.
    \end{definition}
    \vskip 0.1in\noindent

    During the 1980s, A.R.\ Kemer's resolution of Specht problem was a significant breakthrough in the PI-theory (\cite{Kemer}, see also simplified version of Kemer's proof in~\cite{Aljadeff-Kanel-Karasik},~\cite{Procesi}):
    A well-known reformulation of Kemer's theorem is:

    \vskip 0.1in\noindent
    \begin{theorem*}
        Any $T$-ideal of the algebra $F$ is finitely based.
    \end{theorem*}
    \vskip 0.1in\noindent

    It's natural to ask the same question for $T$-spaces.

    In 2001, V.V. Shchigolev combined Grishin (\cite{Grishin}) and Kanel-Belov's (\cite{Kanel}) methods.
    Then Shchigolev noticed that methods similar to Kemer's can be applied to localisation of Specht problem for the T-spaces.
    And finally he proved (\cite{Shchigolev}):

    \vskip 0.1in\noindent
    \begin{theorem*} [V.V. Shchigolev, 2001]
        Any $T$-space of the algebra $F$ is finitely based.
    \end{theorem*}
    \vskip 0.1in\noindent

    We will utilize one simple special case of Shchigolev's theorem, which was also proven in author's last year's coursework.


    \section{Main results}

    \subsection{Zubkov's approach}

    \begin{theorem*}[Zubkov, 1987]
        Let $F$ be a free non-abelian pro-$p$ group, $\Delta$ is a pro-$p$ ring.
        $F$ cannot be continuously embedded in $GL^1_2(\Delta)$, when $p>2$.
    \end{theorem*}
    \vskip 0.1in\noindent

    The first non-trivial idea is to introduce the following definition:
    \vskip 0.1in\noindent
    \begin{definition}
        Let $F$ be a free pro-$p$ group, and $G$ be a pro-$p$ group.
        Then every $1\neq w\in F$ such that $w\in Ker(\varphi)$ for all continuous homomorphisms $\varphi: F\to G$ is called a pro-$p$ identity of G.
    \end{definition}
    \vskip 0.1in\noindent
    Then Zubkov defines a ring of generic matrices over $p$-adic numbers $\mathbb{Z}_p$. The formal construction is quite lengthy, see~\cite{Zubkov}.
    Now the reader can suppose that it's some ring, that can be studied easier than a abstract pro-$p$ ring.

    He defines a natural homomorphism $\pi$ from a free pro-$p$ group $F$ generated by $X, Y$ to the pro-$p$ group generated by
    $1 + x_*, 1 + y_*$, where $x_*, y_*$ are generic matrices.

    This homomorphism is called a universal representation:
    \vskip 0.1in\noindent
    \begin{theorem}
        Each $1\neq w(X, Y) \in \ker \pi$ is a pro-$p$ identity of $GL^1_d(\Delta)$ for all pro-$p$ rings $\Delta$.
    \end{theorem}
    \vskip 0.1in\noindent
    This theorem has pretty simple proof using the standard commutative algebra approaches.
    Surprisingly, it remains to prove that
    \vskip 0.1in\noindent
    \begin{theorem}
        The universal representation is not injective for $p\neq 2$.
    \end{theorem}
    \vskip 0.1in\noindent
    Zubkov proved this by investigating the Lie algebra of generic $2\times 2$ matrices through rather lengthy (but not very complicated) calculations.

    \subsection{Ben-Ezra and Zelmanov's approach}
    \begin{theorem*}[Ben-Ezra, Zelmanov, 2020]
        Let $F$ be a free non-abelian pro-$2$ group, $\Delta$ be a pro-$2$ ring with $\mathrm{char}(\Delta)=2$.
        Then $F$ cannot be continuously embedded in $GL^1_2(\Delta)$.
    \end{theorem*}
    \vskip 0.1in\noindent

    The main ideas of Ben-Ezra and Zelmanov's proof are quite similar to Zubkov's.
    But they use a slightly different way to define the universal representation $\pi$.
    They define generic matrices over $\mathbb{Z}/2\mathbb{Z}$ instead of $\mathbb{Z}_2$ as Zubkov did.
    Nevertheless, the analogous theorem about the kernel of universal representation still holds true at least in the case $\mathrm{char}(\Delta)=2$:
    \vskip 0.1in\noindent
    \begin{theorem}
        Each $1\neq w(X, Y) \in \ker \pi$ is a pro-$2$ identity of $GL^1_2(\Delta)$ for all pro-$2$ rings $\Delta$ with $\mathrm{char}(\Delta)=2$.
    \end{theorem}
    \vskip 0.1in\noindent

    Thus, the main result boils down to
    \vskip 0.1in\noindent
    \begin{theorem}
        The universal representation is not injective.
    \end{theorem}
    \vskip 0.1in\noindent
    This was proven using lengthy calculations spanning 20 pages of fairly complex methods.

    \subsection{An extension of Ben-Ezra and Zelmanov's approach}
    We intend to establish a slightly different version of Theorem 3.2.1 for the case where $\mathrm{char}(\Delta)=4$.
    Moreover, we aim to explore the possibility of extending this result to cases where $\mathrm{char}(\Delta)=2^l$.
    We anticipate that similar methodologies and calculations as those used in the original proof will suffice.

    Additionally, we will discuss the case $\mathrm{char}(\Delta)=0$ and how it can be investigated using the aforementioned approaches.

    \subsection{Gelfand's conjecture}
    We will begin by considering a specific case of Shchigolev's result (\cite{Shchigolev}):
    \vskip 0.1in\noindent
    \begin{theorem}
        Any  $T$-space in algebra of commutative polynomials $k[x_1,\ldots,x_n]$ is finitely based.
    \end{theorem}
    \vskip 0.1in\noindent

    Then we will discuss Gelfand's conjecture.

    \vskip 0.1in\noindent
    \begin{conjecture*}
        The homology of the Lie subalgebra of finite codimension in the Lie algebra of algebraic vector fields on an affine algebraic manifold are finite-dimensional in each
        homological degree.
    \end{conjecture*}
    \vskip 0.1in\noindent

    We denote by $\mathcal{W}_n$ the Lie algebra of formal vector fields on an n-dimensional plane $V$.\\
    Well known that
    \[\mathcal{W}_n\simeq \prod\limits_{k=0}^{\infty}S^k V\otimes V^*\]
    The subalgebras $\prod\limits_{k=d}^{\infty}S^k V\otimes V^*$ of a finite codimension are denoted by $L_d(n)$.
    Utilizing classical considerations of homological algebra (which will be omitted), one can reduce Gelfand's conjecture to the following lemma:
    \vskip 0.1in\noindent
    \begin{lemma}
        Any finitely generated $L_d(n)$-module is Noetherian.
    \end{lemma}
    \vskip 0.1in\noindent

    Finally, it is worth noting that the methods from Theorem 3.4.1 can be applied to prove this lemma.


    \bibliographystyle{abbrv}
    \bibliography{proposal}

%        \bibitem{Kemer} {\sl A. Kemer}, {\it Finite basability of identities of associative algebras}, Algebra i Logika 26(5) (1987), 597-641, 650.
%
%        \bibitem{SimpleKemer} {\sl Eli Aljadeff, Alexei Kanel-Belov, Yaakov Karasik}, {\it Kemer’s theorem for affine PI algebras over a field of characteristic zero}, Journal of Pure and Applied Algebra 220, No. 8, 2771--2808 (2016).
%
%        \bibitem{Procesi} {\sl C. Procesi}, {\it “The geometry of polynomial identities”}, Izv. Math., 80:5 (2016), 910–953.
%
%        \bibitem{GrishinSchigolev} {\sl Grishin, A. V.; Shchigolev, V. V.}, {\it $T$-spaces and their applications.} (English. Russian original) J. Math. Sci., New York 134, No. 1, 1799--1878 (2006); translation from Sovrem. Mat. Prilozh. 18, 26--97 (2004).
%
%        \bibitem{Grishin} {\sl A. V. Grishin}, {\it On the finite-basis property for systems of generalized polynomials}, Izv. Akad. Nauk
%        SSSR, Ser. Mat., 37, No. 2, 243–272 (1991).
%
%        \bibitem{Grishin2} {\sl A. V. Grishin}, {\it “On finitely based systems of generalized polynomials”, Math. USSR-Izv.}, 37:2 (1991), 243–272
%
%        \bibitem{Shchigolev} {\sl V. V. Shchigolev}, {\it Finite-basis property of T-spaces over fields of characteristic zero,} Izv. Ross.
%        Akad. Nauk, Ser. Mat., 65, No. 5, 1041–1071 (2001).
%
%        \bibitem{BelovAljadeff} {\sl Kanel–Belov A., Eli Aljadeff}, {\it Representability and Specht problem for $G$-graded algebras}, Advances in Math., 225:5 (2010), 2391--2428 , arXiv: 0903.0362.
%
%
%        \bibitem{Lie} {\sl I. I. Benediktovich, A. E. Zalesskii}, {\it T-ideals of free Lie algebras with polynomial growth of a sequence of codimensions} (Russian), Vestsi Akad. Navuk BSSR Ser. Fiz-Math. Navuk 3 (1980), 5-10; translation: Proceedings of the National
%        Academy of Sciences of Belarus. Series of Physical-Mathematical Sciences 3(3) (1980).
%
%        \bibitem{Jordan} {\sl A. Ja. Vais, E. I. Zelmanov}, {\it Kemer’s theorem for finitely generated Jordan algebras} (Russian), Izv. Vyssh. Uchebn. Zved.
%        Mat. 33(6) (1989), 42-51; translation: Soviet Math. (Iz. VUZ) 33(6) (1989), 38-47.
%
%        \bibitem{Super} {\sl L. Centrone, A. Estrada, A. Ioppolo}, {\it On PI-algebras with additional structures: rationality of Hilbert series and Specht’s
%        problem}, J. Algebra 592 (2022), 300-356.
%
%        \bibitem{ConterKanel} {\sl A. Ya. Belov}, {\it Counterexamples to the Specht problem} (Russian), Sb. Math. 191(3) (2000), 13-24; translation: Sb. math.
%        131 (3-4) (2000), 329-340.
%
%        \bibitem{ConterGrishin} {\sl A. V. Grishin}, {\it Examples of T-spaces and T-ideals over a field of characteristic 2 without the finite basis property}, Fundam.
%        Prikl. Mat. 5(1) (1999), 101-118.
%
%        \bibitem{ConterShchigolev} {\sl V. V. Shchigolev}, {\it Examples of infinitely based T-ideals}, Fundam. Prikl. Mat. 5(1) (1999), 307-312.
%
%        \bibitem{GradedKanel} {\sl E. Aljadeff, A. Kanel-Belov}, {\it Representability and Specht problem for G-graded algebras}, Adv. Math. 225(5) (2010), 2391-
%        2428.
%
%        \bibitem{GradedSviridova} {\sl I. Sviridova}, {\it Identities of pi-algebras graded by a finite abelian group}, Comm. Algebra 39(9) (2011), 3462–3490.
%
%        \bibitem{Gelfand} {\sl I. M. Gelfand}, {\it The cohomology of infinite dimensional Lie algebras; Some questions of integral geometry}, Proceedings of
%        ICM 1970-Nice, T.1 pp.95-111.
%
%        \bibitem{Centrone} {\sl Lucio Centrone, Alexei Kanel-Belov, Anton Khoroshkin, and Ivan Evgenievich Vorobiev}, {\it Specht property for systems of commutative polynomials and Gelfand conjecture},
%
%        \bibitem{Feigin} {\sl Boris Feigin, Alexei Kanel-Belov, Anton Khoroshkin}, {\it On finite dimensionality of homology of subalgebras of vector
%        fields}, arXiv:2211.08510v1
%
%    \end{thebibliography}

\end{document}
