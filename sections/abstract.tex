\noindent
Пусть $F$ свободная некоммутативная про-$p$ группа, и пусть $\Delta$ коммутативное нетерово полное локальное кольцо с максимальным идеалом $I$, такое что
$\Delta/I$ конечное поле характеристики $p$.\\
Определим группу
\[GL_d^1(\Delta) = \ker\left( GL_d(\Delta) \xrightarrow{\Delta\to\Delta/I} GL_d(\Delta/I) \right)\]
\begin{itemize}
    \item E.И.\ Зельманов (\cite{Zelmanov1}, ~\cite{Zelmanov2}) анонсировал доказательство того, что $F$ не может быть непрерывно вложена в $GL_d^1(\Delta)$ для\\$p\gg d$.
    \item А.Н.\ Зубков~\cite{Zubkov} доказал, что $F$ не может быть непрерывно вложена в $GL_2^1(\Delta)$ для $p\neq2$.
    \item Д.\ Бен-Эзра и Е.И.\ Зельманов~\cite{Ben-Ezra-Zelmanov} доказали, что и для\\$d=2, p=2, \mathrm{char}(\Delta)=2$ имеет место такой же результат.
\end{itemize}

Цель данной статьи ~--- сделать обзор подходов для $2\times2$ матриц, и показать, что они обобщаются на случай $p=2$ и $\mathrm{char}(\Delta)=4$.\\
Кроме того, E.И.\ Зельманов показал в ~\cite{Zelmanov1}, что гипотеза о нелинейности про-$p$ групп тесно связана с PI-теорией.

Во второй части работы мы изучаем связь между PI-теорией (будут приведены методы А.В.\ Гришина на элементарном языке)
и гипотезой Гельфанда о конечномерности гомологий алгебр Ли векторных полей.

Таким образом, можно видеть, что работа в основном посвящена изучению комбинаторики подстановок.