Мы рассмотрим применения теории полиномиальных тождеств (PI-теории для краткости) в двух, казалось бы, несвязанных с ней областях:
\begin{itemize}
    \item В 2005, Е.\ Зельманов представил набросок доказательства нелинейности некоторых свободных про-$p$ групп.
    Доказательство во многом опирается на стандартные подходы PI-теории (см.\ ~\cite{Zelmanov1}).
    \item Дополнительно мы рассмотрим неожиданную связь между гипотезой Гельфанда и PI-теорией, а именно с методами Гришина (\cite{Grishin}).
\end{itemize}

Приведем краткую историческую справку и формулировку основных проблем.
Отметим, что параграфы 2, 3, 4 и параграфы 5, 6 независимы.

\subsection{О нелинейности свободных про-$p$-групп}\label{subsec:introduction-pro-p}
Проблема линейности топологических групп изучалась много лет.
Одной из естественных топологий, наряду с $\mathbb{R}^n$ и дискретной, является про-$p$ топология.

Сформулируем центральную гипотезу данной теории, а далее приведем все необходимые определения:

\vskip 0.1in\noindent
\begin{conjecture}
    Некоммутативная свободная про-$p$ группа не может быть вложена в $GL_d(\Delta)$ ни для какого про-$p$ кольца $\Delta$.
\end{conjecture}
\vskip 0.1in\noindent

\vskip 0.1in\noindent
\begin{definition}
    Обратный (проективный) предел конечных $p$-групп называется про-$p$ группой.
\end{definition}
\vskip 0.1in\noindent
Топология индуцируется с топологии тихоновского произведения.

\vskip 0.1in\noindent
\begin{definition}
    Коммутативное нетерово $I$-полное локальное кольцо $\Delta$ с максимальным идеалом $I$ называется про-$p$ кольцом, если $\Delta/I$ конечное поле характеристики $p$.
\end{definition}
\vskip 0.1in\noindent

Таким образом:
\[
    \Delta \cong \varprojlim \Delta/I^n
\]

Рассмотрим конгруенц-подгруппу:
\[
    GL_d^1(\Delta) = \ker\left( GL_d(\Delta) \xrightarrow{\Delta\to\Delta/I} GL_d(\Delta/I) \right)
\]
Можно заметить, что это про-$p$ группа.


Основной вопрос состоит в том, может ли свободная про-$p$ группа быть непрерывна вложена в $GL_d^1(\Delta)$.
Мотивируя данную постановку, отметим, что А.Н.\ Зубков заметил (\cite{Zubkov}), что, доказывая нелинейность про-$p$ групп, мы можем ограничиться рассмотрением только про-$p$ колец и строить вложение в конгруенц-подгруппу.

Свободную про-$p$ группу можно определить классическим способом через универсальное свойство или конструктивно:

\vskip 0.1in\noindent
\begin{definition}
    Свободная про-$p$ группа $F_p(X)$ является пополнением дискретной свободной группы $F(X)$ относительно топологии всех нормальных подгрупп индекса степени $p$.
\end{definition}
\vskip 0.1in\noindent

Существует множество частичных результатов:
\begin{itemize}
    \item В 1987, А.Н.\ Зубков (\cite{Zubkov}) доказал гипотезу для $d=2, p\neq2$.
    \item В 1999, используя глубокие результаты Пинка (\cite{Pink}), Й.\ Барнеа, М.\ Ларсен (\cite{Barnea-Larsen}) подтвердили гипотезу для $\Delta=\left( \mathbb{Z}/p\mathbb{Z} \right)[[t]]$
    \item В 1991, Д.\ Диксон, А.\ Манн, М.П.Ф.\ Ду Сатой, Д.\ Сигал (\cite{DMSD}) доказали гипотезу для всех размеров матриц для целых $p$-адических чисел $\Delta=\mathbb{Z}_p$, $GL_d^1(\mathbb{Z}_p)=\ker\left( GL_2(\mathbb{Z}_p) \xrightarrow{\mathbb{Z}_p\to\mathbb{F}_p} GL_2(\mathbb{F}_p) \right)$
    \item В 2005, E.\ Зельманов (\cite{Zelmanov1}, ~\cite{Zelmanov2}) анонсировал доказательство гипотезы для $p\gg d$, однако до сих пор существует только набросок доказательства.
    \item В 2020, Д.\ Бен-Эзра, Е.\ Зельманов (\cite{Ben-Ezra-Zelmanov}) обобщили результат Зубкова для $d=2, p=2$ и $\mathrm{char}(\Delta)=2$.
\end{itemize}

Мы сконцентрируемся на случае $2\times 2$ матриц, то есть на результатах Зубкова и Бена-Эзры\textemdash Зельманова.

Сейчас опишем общий план доказательства, он частично реализован для произвольных размеров матриц в\ \cite{Zelmanov1}, ~\cite{Zelmanov2}:

Для начала сузим множество рассматриваемых колец.
Можно построить, так называемое, универсальное представление в общие матрицы.
Оказывается, что классическими алгебраическими рассуждениями можно показать, что если какое-то представление точно, то и это универсальное представление точно.

Таким образом, остается исследовать это универсальное представление.
Это делается путем изучения алгебры Ли общих матриц, которая множество раз возникала в PI-теории.
В завершение строится связь между этой алгеброй Ли и образом нашего представления ~--- подобно связи между группой Ли и алгеброй Ли.

\subsection{PI-теория и гипотеза Гельфанда}\label{subsec:introduction-gelfand}
В 1980-х годах решение проблемы Шпехта А.Р. Кемером стало значительным прорывом в теории полиномиальных тождеств (\cite{Kemer}, см.\ также упрощенную версию доказательства Кемера в~\cite{SimpleKemer}, ~\cite{Procesi}):
\vskip 0.1in\noindent
\begin{theorem*} [А.Р. Кемер, 1987]
    Любая ассоциативная алгебра над полем характеристики ноль имеет конечный базис тождеств.
\end{theorem*}
\vskip 0.1in\noindent

Существует хорошо известная переформулировка теоремы Кемера:

\vskip 0.1in\noindent
\begin{theorem*}
    Любой $T$-идеал алгебры $k\langle X\rangle$, где $X$ — счетный алфавит, а $k$ — поле характеристики ноль, конечно базируем.
\end{theorem*}
\vskip 0.1in\noindent

$T$-идеал — это идеал в $k\langle X\rangle$, который замкнут относительно любого эндоморфизма $k\langle X\rangle$.
$T$-пространство — это векторное подпространство в $k\langle X\rangle$, которое замкнуто относительно любого эндоморфизма $k\langle X\rangle$.

Следующий естественный вопрос: можно ли заменить $T$-идеал на $T$-пространство в теореме Кемера?

А.В.\ Гришин заметил, что доказательство Кемера для систем обобщённых многочленов определённого типа, использует исключительно линейные комбинации и подстановки, а умножение является избыточным (см. \cite{Grishin}, ~\cite{Grishin2}, а также обзор~\cite{GrishinSchigolev}).

Методы Гришина хорошо подходят для полупростых алгебр.
А.Я.\ Канель-Белов заметил, что лемма Артина-Риса (классическая лемма в коммутативной алгебре) и теорема Размыслова (см. \cite{GrishinSchigolev},  кроме того эти переходы содержатся в курсовой работе автора прошлого года) могут дополнить рассуждения Гришина.
Он осуществил переход к алгебре меньшей сложности, индуктивно повторяя методы Гришина.
Однако эти методы подходят только для локальной проблемы Шпехта (когда $X$ — конечный алфавит).

В 2001 году В.В. Щиголев объединил методы Канеля-Белова и Гришина, дополнительно заметив, что подход, приведенный у Кемера, можно применить к локализации проблемы Шпехта для $T$-пространств.
И, наконец, он доказал (см. \cite{Shchigolev}):
\vskip 0.1in\noindent
\begin{theorem*} [В.В. Щиголев, 2001]
    Любое $T$-пространство алгебры $k\langle X\rangle$, где $X$ — счетный алфавит, а $k$ — поле характеристики ноль, имеет конечную базу.
\end{theorem*}
\vskip 0.1in\noindent

Существует множество неассоциативных постановок этой проблемы: для алгебр Ли (см. \cite{Lie}), для Йордановых (см. \cite{Jordan}), для супералгебр (см. \cite{Super}).
Также существуют постановки над полем положительной характеристики\ (см. контрпримеры для $T$-идеалов в работах \cite{ConterKanel}, \cite{ConterGrishin}, \cite{ConterShchigolev} Канеля-Белова, Гришина и Щиголева соответственно).
И для алгебр градуированных конечной группой (см.\ работу~\cite{GradedKanel} и работу~\cite{GradedSviridova} для случая коммутативной конечной группы).

В 2022 была обнаружена замечательная связь между PI-теорией (точнее методами Гришина) и гипотезой Гельфанда сформулированной на ICM’70 (см\ \cite{Gelfand}).
\vskip 0.1in\noindent
\begin{conjecture}[Гельфанд, 1970]
    \label{Gelfand}
    Гомологии подалгебры Ли конечной коразмерности алгебры Ли алгебраических векторных полей на афинном алгебраическом многообразии конечномерны.
\end{conjecture}
\vskip 0.1in\noindent
Эта интересная связь была найдена в результате совместной работы А.С.\ Хорошкина, А.Я. Канель-Белова с некоторым участием автора.
Можно найти набросок доказательства Хорошкина в ~\cite{Feigin-Kanel-Khoroshkin}, ~\cite{Centrone-Kanel-Khoroshkin-Vorobiov}.

Дополнительно, мы заметим, что именно результат, которого касалась курсовая работа автора после второго курса (частный случай методов Гришина\ \cite{Grishin}, описанных элементарными методами) помогает в доказательстве гипотезы Гельфанда.