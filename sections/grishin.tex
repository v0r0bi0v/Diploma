Пусть $k$ — поле характеристики ноль, а $F = k\langle x_1, \ldots, x_i, \ldots \rangle$ — свободная, счётно порождённая ассоциативная алгебра над полем $k$.
Пусть $T$ — полугруппа эндоморфизмов (подстановок) $F$.
И алфавит\\ $X = \{ x_1, \ldots, x_i, \ldots \}$.\\
Теперь приведем несколько классических определений.

\vskip 0.1in\noindent
\begin{definition}
    Эндоморфизм $\tau$ алгебры $F$, определяемый правилом $x_i \mapsto g_i$, где $g_i \in F$, называется подстановкой типа $(x_1, \ldots, x_i, \ldots) \mapsto (g_1, \ldots, g_i, \ldots)$.
\end{definition}
\vskip 0.1in\noindent

\vskip 0.1in\noindent
\begin{definition}
    $T$-пространство в $F$ — это векторное подпространство $F$, замкнутое относительно подстановок.
\end{definition}
\vskip 0.1in\noindent

\vskip 0.1in\noindent
\begin{definition}
    $T$-идеал в $F$ — это идеал $F$, который одновременно является $T$-пространством.
\end{definition}
\vskip 0.1in\noindent

Следующая теорема (а вернее, метод ее доказательства) очень полезна при изучении гипотезы Гельфанда:
\vskip 0.1in\noindent
\begin{theorem}
    \label{main}
    Пусть $M$ ~--- подмножество кольца многочленов от $n$ переменных над полем характеристики 0.
    Тогда существует конечное подмножество $\M_0\subseteq M$, такое что $M$ содержится в $T$-пространстве порожденном $M_0$.
\end{theorem}
\vskip 0.1in\noindent
В следующих параграфах приведено элементарное доказательство этой теоремы, отражающее суть методов Гришина.

\subsection{Основные леммы}\label{subsec:grishin-main-lemmas}

Будем говорить, что множество $N_0$ является \textit{базой} $N$, если каждый многочлен из $N$ можно получить применяя к элементам $N$ конечное количество подстановок и линейных комбинаций.
Будем писать \textit{операции} имея в виду линейные комбинации и подстановки.
Свойство существования конечной базы у множества называется конечной базируемостью множества.

Для начала докажем три простых леммы.\vskip 0.1in\noindent
\begin{lemma}
    \label{closure}
    Достаточно доказать теорему~\ref{main} для случая, когда $M$ замкнуто относительно линейных комбинаций и подстановок, то есть является $T$-пространством.
\end{lemma}
\begin{proof}
    Рассмотрим $\overline{M}$ --- множество, получаемое из $M$ применением конечного количества последовательных операций.\\
    Ясно, что $\overline{M}$ замкнуто относительно них.
    Действительно, если взять любую линейную комбинацию многочленов из $\overline{M}$, то она получается из $M$ последовательным применением конечного числа операций и, следовательно, лежит в $\overline{M}$.
    То же самое верно и про подстановку.\\
    Осталось доказать, что если $\overline{M}$ конечно базируемо, то и $M$ --- тоже.
    Пусть $\overline{M_0}$ --- конечная база $\overline{M}$.
    Тогда пусть $M_0$ --- это множество многочленов, из которых мы получили $\overline{M_0}$.
    По построению $M_0$ конечно и является базой $M$.
\end{proof}
Следующая лемма также сужает класс множеств для которых мы будем доказывать теорему: можно считать, что $M$ содержит исключительно однородные многочлены.\vskip 0.1in\noindent
\begin{lemma}
    \label{homogen}
    Из многочлена $F$ конечным количеством операций можно получить его однородные компоненты.
\end{lemma}
\begin{proof}
    Пусть $m$ -- степень $F$.\\
    Сделаем подстановку: $F(\alpha x_1, \alpha x_2, \ldots, \alpha x_n)$.
    Заметим, что при такой подстановке однородная компонента $k$-ой степени умножается на $\alpha^k$.
    Тогда пусть $v_0,\ldots,v_n$ -- однородные компоненты $F$. \\
    Заметим, что все подстановки указанного вида лежат в линейной оболочке $\langle v_0,v_1,\ldots,v_n\rangle$.
    Запишем $F(\alpha x_1, \alpha x_2, \ldots, \alpha x_n)$ в базисе $v_0,\ldots,v_n$:
    \[
        (\alpha^0,\alpha^1,\ldots,\alpha^n)
    \]
    Так как определитель матрицы Вандермонда отличен от $0$, получаем, что если подставить любые $m+1$ различных $\alpha$, можно выразить любую однородную компоненту $v_i$.
\end{proof}
Теперь можно считать, что множество $M$ состоит только из однородных многочленов, однако возникает проблема: нам было бы удобно считать и то, что множество замкнуто относительно операций, и то, что оно состоит из однородных многочленов.
Однако сейчас эти условия, конечно, несовместимы.\\
Эта проблема решается следующим образом.
Вместо операций будем применять конечную последовательность операций, которая из однородного многочлена получает однородный.
Такие последовательности операций будем называть однородной подстановкой.\\
Причем можно считать, что $M$ замкнуто относительно однородных операций, доказательство этого аналогично уже приведенному.

Когда решается задача про конечную базируемость множества многочленов (например, доказывается существование базиса Грёбнера из теоремы Гильберта о базисе) полезной является лемма про светильники ~--- она помогает следующим образом.
Если ввести порядок на старших членах многочленов и доказать, что старшие члены из младших (в некотором смысле), то из этого получим конечную базируемость.\\
Следующая лемма очевидным образом следует из теоремы Гильберта о базисе, но приведем все-таки независимое доказательство.
\vskip 0.1in\noindent
\begin{lemma}
    \label{lamp}
    Пусть в $\mathbb{Z}_+^n$ дано некоторое множество светильников -- точка с координатами $(x_1,\ldots,x_n)$ освещает все $(y_1,\ldots,y_n), y_i\geq x_i \forall i$.
    Тогда можно выбрать конечное количество светильников, которые освещают все остальные.
\end{lemma}
\begin{proof}
    Докажем индукцией по размерности пространства.\\
    При $n=1$ утверждение леммы тривиально.\\
    Пусть лемма верна для $n-1$, докажем для $n$. \\
    Спроецируем все светильники на гиперплоскость $x_n=0$.
    Выберем конечный набор проекций $m_1',\ldots,m_k'$, который освещает все остальные проекции (такой есть в силу индукционного предположения).
    Теперь найдем $m_i$ --- светильник с наименьшей координатой по $x_n$ из тех, что спроецировались в $m_i'$.\\
    Пусть $l$ --- наибольшая координата по $x_n$ среди $m_i$.
    Заметим, что для все светильники с координатой по $x_n$, не меньшей $l$ освещены множеством $m_1,\ldots,m_k$.
    Осталось для каждой гиперплоскости $x_n=c$, где $c=0,\ldots,l-1$ выбрать конечные наборы светильников, которые освещают все остальные, лежащие в соответствующей гиперплоскости.
\end{proof}

\subsection{Ключевая подстановка}\label{subsec:key-substitution}
Следующая подстановка является ключевой:
\begin{equation}
    \label{eq:key-sub}
    P(x_1,\ldots,x_n)\mapsto P_{(t, 1)}\left(x_1+t(x_1)x_1,\ldots,x_n+t(x_n)x_n\right)
\end{equation}
где индекс $(t,1)$ обозначает, что мы линеаризуем по $t$ после того, как, сделаем приведенную подстановку.\\
Докажем, что мы можем отлинеаризовать по $t$.
Будем рассматривать многочлен $P(x_1+t(x_1)x_1,\ldots,x_n+t(x_n)x_n)$ как многочлен от $2n$ переменных -- от $x_i$ и $t(x_i)$.\\
Причем понятно, что переменные $t(x_i)$ можно умножать на константу все сразу -- в подстановке~\ref{eq:key-sub} можно взять $x_i\mapsto x_i+c\cdot t(x_i)x_i$.
Далее все аналогично доказательству леммы~\ref{homogen}.\\
Итак, разберемся, что получилось после отлинеаризованной подстановки (\ref{eq:key-sub}). Для этого рассмотрим, что произойдет с мономом $x_1^{\alpha_1}\ldots x_n^{\alpha_n}$:
\[x_1^{\alpha_1}\ldots x_n^{\alpha_n}\mapsto ((x_1+t(x_1)x_1)^{\alpha_1}\ldots (x_n+t(x_n)x_n)^{\alpha_n}))_{(t,1)}\]
Ясно, что после линеаризации по $t$ останутся только те члены, в которых ровно из одной скобки произведения взяли $t(x_i)$.
То есть при $t(x_i)$ будет $\alpha_i\cdot x_1^{\alpha_1}\ldots x_n^{\alpha_n}$.
Итак:
\begin{equation}
    \label{eq:key_monom}
    x_1^{\alpha_1}\ldots x_n^{\alpha_n}\mapsto\sum\limits_{i=1}^n t(x_i)(\alpha_i\cdot x_1^{\alpha_1}\ldots x_n^{\alpha_n})
\end{equation}
\begin{remark}
    Заметим, что это соответствует действию диагонального векторного поля $\sum t(x_i) \cdot x_i \cdot \partial_i$.
    Это будет полезно для гипотезы Гельфанда.
\end{remark}
\vskip 0.1in\noindent
Следовательно, однородный многочлен $P$ степени $m$ перейдет в:
\begin{equation}
    \label{eq:key_t}
    \sum\limits_{i=1}^n t(x_i)P_i(x_1,\ldots,x_n)
\end{equation}
Причем, если вместо $t$ подставить $\frac{1}{m}$, то из последней формулы~\eqref{eq:key_t} получится ровно многочлен $P$.\\
Далее, нам потребуются лишь $t(x)=x^k$.
Каждому многочлену из множества $M$ мы сопоставили некоторое семейство многочленов:
\begin{equation}
    \label{eq:key_sum}
    \sum\limits_{i=1}^n x_i^k P_i(x_1,\ldots,x_n)
\end{equation}
где $k$ пробегает все целые неотрицательные числа.\\
Обозначим сумму~\eqref{eq:key_sum} как $(P_1,\ldots,P_n)$ и назовем ее координатным представлением многочлена.\\
\vskip 0.1in\noindent
\begin{definition}
    Семейством многочленов будем называть всякое множество многочленов вида $\sum\limits_{i=1}^n x_i^k P_i(x_1,\ldots,x_n)$.
\end{definition}
\vskip 0.1in\noindent

Мы будем делать однородные подстановки специального вида уже к семействам.
Причем будем делать их так, чтобы они инвариантно действовали на каждую координату.
Таким образом, мы сможем доказать конечную базируемость по каждой координате, а потом выведем из этого глобальную конечную базируемость.
Ключевая идея заключается в том, что для каждой отдельной координаты доказывать конечную базируемость куда проще, чем для бескоординатного представления, так как наличие $i$-ой координаты разрешает нам умножать многочлены внутри нее на $x_i$, что существенно упрощает задачу.
\vskip 0.1in\noindent
\begin{remark}
    Мы доказываем конечную базируемость множества семейств многочленов вместо множества многочленов.
\end{remark}
\vskip 0.1in\noindent

Мы докажем конечную базируемость по каждой координате, но для начала докажем, что из этого будет следовать теорема~\ref{main}.

\subsection{Глобальная конечная базируемость из конечной базируемости по каждой координате}\label{subsec:grishin-global-from-local}
Мы рассматриваем вместо каждого многочлена $P(x_1,\ldots,x_n)$ соответствующее ему семейство:
\[\sum\limits_{i=1}^n x_i^k P_i(x_1,\ldots,x_n)\] Пусть $N$ --- множество координатных представлений многочленов из $M$.
Аналогично лемме~\ref{closure}, рассмотрим вместо $N$, его замыкание $\overline{N}$.
То есть будем рассматривать некоторое множество систем многочленов, содержащее в себе координатное представление каждого многочлена из $M$.

Теперь осталось явно описать конечный базис $\overline{N}$.
Сначала рассмотрим конечное количество семейств многочленов, у которых первая координата является конечным базисом первой координаты из $\overline{N}$.

Далее, возьмем $\overline{N}_1\subset \overline{N}$ состоящее только из семейств с нулевой первой координатой (ясно, что это подмножество тоже замкнуто относительно подстановок и линейных комбинаций).
Теперь возьмем те семейства из $\overline{N}_1$, у которых вторая координата является конечной базой вторых координат.

Повторяя эту операцию получим конечную базу всего $\overline{N}$.

\subsection{Конечная базируемость по каждой координате}\label{subsec:local}
Ключевая подстановка решает задачу в случае двух переменных, однако в общем случае нужны тонкие индукционные рассуждения.\\
Докажем, что внутри каждой координаты можно опять сделать подстановку (\ref{eq:key-sub}).
\begin{multline*}
(p_1,...,p_n)
    =\sum\limits_{i=1}^n x^k_i p_i(x_1,\ldots,x_n)\mapsto\\\mapsto [\sum\limits_{i=1}^n (x_i+r(x_i)x_i)^k\cdot p_i(x_1+r(x_1)x_1,\ldots,x_n+r(x_n)x_n))]_{(r,1)}
\end{multline*}Здесь индекс $(r,1)$ означает линеаризацию по $r$, как и раньше.
Рассмотрим первую координату.
Она, конечно, полностью получается из члена
\[[(x_1+r(x_1)x_1)^k\cdot p_1(x_1+r(x_1)x_1,\ldots,x_n+r(x_n)x_n)]_{(r,1)}\]Раскрыв первые скобки получаем
\begin{equation}
    \label{eq:sub_interm}
    x_1^k\cdot(k\cdot r(x_1) p_1(x_1,\ldots,x_n)+ [p_1(x_1+r(x_1)x_1,\ldots,x_n+r(x_n)x_n)]_{(r,1)})
\end{equation}

Заметим, что первое слагаемое нам совсем не нужно, ведь оно преобразует только степень мономов $p_1$ по $x_1$, а мы добиваемся увеличения других степеней.
Так что вычтем из семейства, полученного при подстановке~\eqref{eq:sub_interm} изначальное семейство умноженное на $r(x_i)k$ покоординатно.
Получим, что по первой координате у нас:
\[[p_1(x_1+r(x_1)x_1,\ldots,x_n+r(x_n)x_n)]_{(r,1)}\]Сравним это с подстановкой (\ref{eq:key-sub}), специализуем $r(x)=x^m$ и сделаем вывод, что это равно
\begin{equation}
    \label{eq:key_inside}
    \sum\limits_{i=1}^n x_i^m p_{1i}(x_1,\ldots,x_n)
\end{equation}
Итак, внутри каждой координаты можно опять делать ключевую подстановку.
Назовем ее $\varphi$.\\
Докажем теперь теорему в следующем частном случае.

\subsubsection{Доказательство в случае двух переменных.}
Итак, пусть у нас есть многочлен \[x_1^{m_1}p_1+x_2^{m_1}p_2\]
Мы рассматриваем только случай, когда в $p_1$ все мономы делятся $x_1$, ведь иначе первой координаты вообще не существовало бы, то есть она была бы нулевой (что видно из~\eqref{eq:key_monom})
Мы хотим узнать, что из семейства многочленов $x_1^{m_1}p_1+x_2^{m_1}p_2$ можно получить по первой координате.
Здесь потребуется лемма, которая сведет всю задачу к лемме о светильниках (\ref{lamp}).\vskip 0.1in\noindent
\begin{lemma}
    Если у $p_1$ старший член в лексикографическом порядке $x_2\succ x_1$ это $x_1^{\alpha_1}x_2^{\alpha_2}$, причем $\alpha_2>0$, то для любого монома, делящегося на данный, мы можем получить многочлен, у которого по первой координате будет старшим этот моном.
\end{lemma}
\begin{proof}
    Пусть нам нужно получить моном $x_1^{\alpha_1+\beta_1}x_2^{\alpha_2+\beta_2}$.

    Рассмотрим случай, когда $\beta_2=0$.
    Тогда сдвинем %! suppress = VerticallyCenteredColon
    $m_1 := m_1+\beta_1$.
    Естественно, мы получим многочлен с нужным старшим членом
    Пусть теперь $\beta_2>0$.
    Сделаем нашу подстановку еще раз, как мы описали в~\eqref{eq:key_inside}.
    Получится многочлен
    \[x_1^{m_1}(x_1^{m_2} p_{11}+x_2^{m_2} p_{12})\]Причем и в $p_{11}$, и в $p_{22}$ старший член --- это $x_1^{\alpha_1}x_2^{\alpha_2}$.
    Тогда возьмем $m_2=\beta_2$ и, опять же, заменим $m_1$ на $m_1+ \beta_1$.
    Получается
    \[x_1^{m_1}x_1^{\beta_1}(x_1^{\beta_2} p_{11}+x_2^{\beta_2} p_{12})=x_1^{m_1}(x_1^{\beta_2}x_1^{\beta_1}p_{11}+x_2^{\beta_2}x_1^{\beta_1} p_{12})\]Ясно, что старший моном содержится только во втором слагаемом (ведь $\beta_2>0$). Причем он имеет вид $x_1^{\alpha_1+\beta_1}x_2^{\alpha_2+\beta_2}$.
\end{proof}
\begin{remark}
    \label{remark}
    Полезно заметить, что если все же $\alpha_2=0$, то мы по крайней мере можем получить мономы вида $x_1^{\alpha_1 + \beta_1}$.
\end{remark}
\vskip 0.1in\noindent
Итак, осталось свести доказательство к лемме о светильниках (\ref{lamp}). Рассмотрим множество $M$ и каждому многочлену из него сопоставим точку в $\mathbb{Z}_+^2$ -- набор степеней в его старшем члене по первой координате.
В этой точке расположим светильник: если ордината не равна 0, то он освещает все точки, которые не меньше его по абсциссе и ординате; если же ордината равна 0, то он освещает все точки, которые совпадают с ним по ординате и не меньше по абсциссе (см замечание~\ref{remark}).\\
Здесь видно, что нам не хватает леммы о светильниках и нужна ее расширенная версия:\vskip 0.1in\noindent
\begin{lemma}
    Пусть, как и в лемме~\ref{lamp}, дано множество светильников, только теперь каждый $(x_1,\ldots,x_n)$ освещает точки $(y_1,\ldots,y_n)$, для которых выполнено два условия: $y_i\geq x_i$ и $y_i=0$, если $x_i=0$.
    Тогда, опять, же существует конечное подмножество светильников, которое освещает все остальные.
\end{lemma}
\begin{proof}
    Пусть $K_i$ --- множество точек, у которых ровно $i$ координат ненулевые.
    Внутри $K_i$ можно найти соответствующий ему конечный набор светильников по классической лемме.
    Остается заметить, что $\mathbb{Z}_+^n=\bigcup_{k=0}^n K_i$.
\end{proof}
Теперь, в силу леммы, можно выбрать конечное подмножество светильников, которые мы сопоставили многочленам, освещающее все остальные.\\
Осталось доказать, что многочлены, которым соответствуют эти светильники являются конечной базой по первой координате (обозначим множество этих многочленов за $M_0$) множества $M$. \\
Предположим противное.
Рассмотрим многочлен с наименьшим старшим членом по первой координате, первая координата которого не получается из $M_0$.\\
Однако мы можем получить многочлен с таким же старшим членом по первой координате.
Вычитая один из другого, получаем меньший многочлен, у которого первая координата не получается из $M_0$.
Противоречие.

\subsubsection{Доказательство в общем случае}
Когда мы делаем подстановку $\varphi$ внутри первой координаты, у нас возникает сумма
\[\sum\limits_{i=1}^n x_i^m p_{1i}(x_1,\ldots,x_n)\]
Видно, что первое слагаемое нам совсем не нужно, ведь оно разрешает умножать на $x_1$, а это мы и так делать можем, в виду того, что работаем с первой координатой.
Так что хотелось бы считать $x_1$ коэффициентом и воспользоваться конечной базируемостью уже для меньшего числа переменных.\\
Эта идея реализуется следующим образом.
Во-первых, разрешим делать только $\varphi$ подстановки, чтобы не потерять однородность.
А во вторых, будем доказывать теорему~\ref{main} не над полем, а над произвольным нетеровым кольцом, содержащем подкольцо изоморфное $\mathbb{Q}$ (последнее обобщение позволит нам в индукционном переходе считать $x_1$ коэффициентом, а не переменной).\vskip 0.1in\noindent
\begin{theorem}
    \label{generalization}
    Пусть $M$ --- множество многочленов от $n$ переменных над нетеровым кольцом $R$, содержащем подкольцо изоморфное $\mathbb{Q}$.
    Тогда $M$ конечно базируемо относительно линейных комбинаций и $\varphi$-подстановок.
\end{theorem}\vskip 0.1in\noindent
Ввиду того, что $R$ содержит поле характеристики $0$, леммы~\ref{closure} и~\ref{homogen} дословно переносятся на этот обобщенный случай (ведь рассуждение с матрицей Вандермонда задействует только частный случай $\varphi$-подстановки).
Значит, опять же можно доказывать теорему только для однородных многочленов.\\
Также мы все еще можем разбить многочлен на координаты~\ref{eq:key_sum} и доказать только конечную базируемость по каждой.

Итак, будем доказывать теорему~\ref{generalization} по индукции по числу переменных.

\vskip 0.1in\noindent
{\large\textbf{База индукции.}}\\
Пусть $n=1$.\\
Заметим, что каждый многочлен с помощью $\varphi$-подстановки можно умножить на $x$, а с помощью линейной комбинации умножать на элемент кольца.\\
Следовательно из каждого многочлена из $M$ можно получить идеал порожденный им в $R[x]$.
Получается, что база индукции эквивалентна теореме Гильберта о базисе.

\vskip 0.1in\noindent
{\large\textbf{Переход индукции.}}\\
Пусть теорема верна для $n-1$ переменной, докажем для $n$.\\
Рассмотрим координатные представления многочленов и, без ограничения общности, будем доказывать конечную базируемость по первой.\\
Первая координата имеет вид $x_1^m P$.
То есть, как и раньше, мы можем умножать многочлены внутри первой координаты на $x_1$.
Следовательно, достаточно доказать утверждение теоремы~\ref{generalization} с дополнительной операцией: теперь кроме линейных комбинаций и $\varphi$-подстановок также разрешим умножать на $x_1$.\\
Как описывалось в начале параграфа, мы хотели бы на время забыть, что $\varphi$-подстановка действует на первую координату. \\Введем $\psi$-подстановку:
\[\psi : P(x_1,\ldots,x_n) \mapsto P_{(r,1)}(x_1,x_2+r(x_2)x_2,\ldots,x_n+r(x_n)x_n)\]\vskip 0.1in\noindent
\begin{lemma}
    \label{ind_step}
    Множество многочленов от $n$ переменных конечно базируемо относительно $\psi$-подстановок, умножений на $x_1$ и линейных комбинаций.
\end{lemma}
\begin{proof}
    Заметим, что это множество можно рассматривать, как множество многочленов от $n-1$ переменной ($x_2,\ldots,x_n$) над кольцом $R[x_1]$ (которое является нетеровым).
    Причем все условия теоремы 2 соблюдаются.
    По предположению индукции получаем требуемое.
\end{proof}
Итак, осталось применить лемму.\\
Назовем \textit{шириной} монома его суммарную степень по переменным $x_2,\ldots,x_n$.\\
Мы доказываем конечную базируемость множества $M$ однородных многочленов, причем опять же, как и раньше, будем считать, что оно замкнуто относительно $\varphi$-подстановок и линейных комбинаций.
Тогда пусть $N$ --- множество получаемое из $M$, заменой каждого многочлена на него же, но с удаленными мономами \textbf{не} максимальной ширины.

Применив лемму~\ref{ind_step} получаем, что $N$ конечно базируемо относительно $\psi$-подстановок.
Пусть $f'_1,\ldots,f'_k$ конечный базис.\\
Рассмотрим $f_1,\ldots,f_k$ --- какие-то многочлены, которые заменились на $f'_1,\ldots,f'_k$ при переходе от $M$ к $N$.
Предположим, что это не базис $M$.

Тогда возьмем многочлен минимальной ширины $p\in M$, который нельзя получить с помощью $\varphi$-подстановок и линейных комбинаций (здесь мы пользуемся замкнутостью $M$).
Пусть $p'$ --- соответствующий ему однородный по ширине многочлен из $N$.\\
Проделаем все те же операции с $\{f_i\}$, что делали с $\{f'_i\}$, чтобы получить $p'$, заменив подстановку вида $\psi$ на соответствующую ей $\varphi$-подстановку.
Заметим, что сначала мы можем применить операции подстановок, а потом уже один раз сделать линейную комбинацию, ведь $\varphi$ и $\psi$ -подстановки линейны.

Итак, сделаем все подстановки и заметим, что после линейной комбинации мономы старшей ширины совпадают с $p'$.
Действительно, каждый раз делая  $\varphi$-подстановку вместо $\psi$ у нас возникают <<лишние>> члены не максимальной ширины (ведь отличие между ними в том, что одна бьет по $x_1$ --- переменной не увеличивающей ширину, а другая нет).
Также члены максимальной ширины не сократятся друг с другом или с какими-то другими, потому что они всегда совпадают с теми, что мы получали из $\{f'_i\}$.\\
Получается, что раз $p$ не порождается базисом $\{f_i\}$, а какой-то многочлен с такими же мономами старшей ширины мы получить можем, то существует многочлен $q$ меньшей ширины, чем $p$, который тоже не порождается базисом.
Противоречие.
\vskip 0.1in\noindent
\begin{remark}
    В теореме 2 существенно, что мы доказываем конечную базируемость относительно однородных подстановок.
    Иначе возникли бы эффекты, связанные с тем, что мономы старшей ширины, получаемые из $f'_1,\ldots,f'_k$ могли бы занулиться, что не дало бы свести $\varphi$-подстановки к $\psi$-подстановкам. \\Действительно, мы бы знали только то, что из $f'_1,\ldots,f'_k$ получаются все старшие по ширине компоненты.
    Однако они могли бы не совпасть со старшими компонентами после подстановок и линейной комбинации.
    Ведь последняя могла сократить старшие компоненты, полученные после подстановок (именно в этом месте мы существенно воспользовались однородностью подстановок), и разница между подстановками $\varphi$ и $\psi$ повлияла бы на итоговые старшие компоненты.
\end{remark}