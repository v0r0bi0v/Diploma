%! suppress = EscapeUnderscore
%! suppress = EscapeAmpersand
\begin{theorem}[Зубков, 1989]
    \label{thm:Zubkov-main}
    Некоммутативная свободная про-$p$ группа не может быть непрерывно вложена в $GL^1_2(\Delta)$ для $p\neq 2$.
\end{theorem}

\subsection{Универсальное представление}\label{subsec:zubkov-universal}
Определим алгебру общих матриц над $p$-адическими числами.

Мы будем работать с матрицами $2\times2$, однако все результаты и конструкции этого параграфа дословно переносятся на матрицы произвольного размера.

Рассмотрим формальные степенные ряды от свободных коммутирующих переменных $x_{i,j}, y_{i,j}$ для $i,j \in \{ 1, 2 \}$:
\[
    S = \Z_{p}\langle x_{1,1}, y_{1,1}, \ldots, x_{2,2}, y_{2,2} \rangle
\]
Введем стандартную градуировку $\deg$: любой элемент $S$ записывается в виде $\sum f_i$ где $\deg{(f_i)} = i$.

Рассмотрим идеалы вида
\[
    S_k = \{\sum\limits_k^{\infty} f_i \}
\]

\vskip 0.1in\noindent
\begin{proposition}
    Следующие идеалы
    \[B_{k,n} = S \cdot p^n + S_k \]
    являются идеалами конечного индекса
\end{proposition}
\begin{proof}
    Достаточно доказать, что $\Z_p / p^n\Z_p$ --- конечное кольцо.
    \[
        \ker{(\Z_p \to Z_p / p^n\Z_p )} = \Z_p p_n = \left( 0,\ldots, 0, a_{k+1}, a_{k+2}, \ldots \right)
    \]
    Ясно, что множество целых $p$-адических чисел отличающихся на элементы такого вида конечно и
    $\Z_p / \Z_p p^n \cong \Z / p^n\Z$
\end{proof}

Снабдим $S$ топологией с базой окрестностей нуля состоящей из идеалов $B_{k,n}$.
Ясно, что $S$ является про-$p$ кольцом:
\[
    S / B_{1,1} \cong \Z/p\Z
\]

Наконец, рассмотрим матричное кольцо $M_2(S)$.
Наделив его топологией с базой окрестностей нуля конгруенц-идеалов
\[
    \ker{(M_2(S) \to M_2(S / B_{k,n}))}
\]
Получим, что $M_2(S)$ ~--- про-$p$ кольцо.
%    todo тут хотелось бы аккуратно все эти вещи доказать, почему это про-$p$ кольцо и почему сейчас получится даже про-$p$ группа.
\vskip 0.1in\noindent
\begin{proposition}
    Множество
    \[1 + \ker{(M_2(S) \to M_2(S / S_1))}\]
    является про-$p$ группой.
\end{proposition}
\begin{proof}
    Заметим, что $\ker{(S\to S/S_1)}$ состоит из рядов без свободного члена.
    Получается, что $1 + \ker{(M_2(S) \to M_2(S / S_1))}$ является группой, так как ряд обратим тогда и только тогда, когда его свободный член обратим.

    Также можно заметить, что эта группа полна относительно определенной выше топологии, то есть является про-$p$ группой.
\end{proof}
Рассмотрим общие матрицы $X, Y \in \ker{(M_2(S) \to M_2(S / S_1))}$:
\[
    X=
    \begin{pmatrix}
        x_{1,1} & x_{1,2} \\
        x_{2,1} & x_{2,2}
    \end{pmatrix},
    \quad
    Y=
    \begin{pmatrix}
        y_{1,1} & y_{1,2} \\
        y_{2,1} & y_{2,2}
    \end{pmatrix}
\]
Пусть $F$ --- свободная про-$p$ группа порожденная $x, y$.

Наконец определим универсальное представление:
\[
    \pi:
    \left\{
    \begin{array}{l}
        x \mapsto 1 + X \\
        y \mapsto 1 + Y
    \end{array}
    \right.
\]

Продолжим его на дискретную подгруппу, порожденную $x, y$:
\[
    \pi: \quad \langle x, y \rangle \to \langle 1+X, 1+Y \rangle \subseteq 1 + \ker{(M_2(S) \to M_2(S / S_1))}
\]
А затем можно непрерывно доопределить $\pi$ на всей $F$, построив замыкание $\langle 1+X, 1+Y \rangle$ в топологии $1 + \ker{(M_2(S) \to M_2(S / S_1))}$:
\[
    \pi:\quad F \to G \subseteq 1 + \ker{(M_2(S) \to M_2(S / S_1))}
\]

Итак, следующая теорема позволяет нам изучать только универсальное представление, не задумываясь о других про-$p$ кольцах.

\vskip 0.1in\noindent
\begin{theorem}[Зубков, 1987]
    Пусть $F$ ~--- свободная про-$p$ группа порожденная $x, y$.
    Если существует инъективный непрерывный гомоморфизм $\varphi: F \to GL^1_2(\Delta)$, то и универсальное представление $\pi$ инъективно.
\end{theorem}
\begin{proof}
    Напомним
    \[
        GL^1_2(\Delta) = \ker{(GL(\Delta)\to GL(\Delta/I))}
    \]

    Рассмотрим образы $x, y$:
    \[
        \varphi(x) = 1 + A, \quad
        \varphi(y) = 1 + B
    \]
    Заметим, что
    \[
        A, B \in \ker{(M_2(\Delta)\to M_2(\Delta/I))}
    \]
    так как $1 + A, 1 + B \in \ker{(GL(\Delta \to \Delta / I))}$.
    Пусть
    \[
        A=
        \begin{pmatrix}
            a_{1,1} & a_{1,2} \\
            a_{2,1} & a_{2,2}
        \end{pmatrix},
        \quad
        B=
        \begin{pmatrix}
            b_{1,1} & b_{1,2} \\
            b_{2,1} & b_{2,2}
        \end{pmatrix}
    \]
    Ясно, что $\lim a_{i,j}^n = \lim b_{i,j}^n = 0$.
    Тогда можно построить гомоморфизм $\zeta: x_{i,j} \mapsto a_{i,j}, y_{i,j} \mapsto b_{i,j}$, он индуцирует эпиморфизм
    $\hat{\zeta}: G \to \mathrm{Im}\hspace{0.1cm}{\varphi}$.
    Наконец, получаем коммутативную диаграмму
    \begin{center}
        \begin{tikzcd}
            & F\arrow{dl}{\pi} \arrow{dr}{\varphi} & \\
            G\arrow{rr}{\hat{\zeta}} & & GL_2^1(\Delta)
        \end{tikzcd}
    \end{center}
    Следовательно:
    \[
        \ker{\pi} \subseteq \ker{\varphi} \qedhere
    \]

\end{proof}

\subsection{Неточность универсального представления}\label{subsec:zubkov-non-injective}
Введем обозначения:
\begin{itemize}
    \item $\mathbf{S}$ ~--- кольцо степенных рядов от общих матриц $X, Y$ над $\Z_p$
    \item $\LQ$ ~--- алгебра Ли порожденная общими матрицами $X, Y$ над $\Q_p$
    \item $\LZ$ ~--- алгебра Ли порожденная общими матрицами $X, Y$ над $\Z_p$
    \item $\LQ^{(n)}$ ~--- векторное пространство над $\Q_p$ однородных элементов степени $n$ в алгебре $\LQ$.
    \item $\LZ^{(n)}$ ~---  $\Z_p$-модуль однородных элементов степени $n$ в алгебре $\LZ$.
    \item Для $g\in G$: $\min{g}$ ~--- однородная компонента наименьшей ненулевой степени (можно записать $g=1 + a_n + a_{n+1} + \ldots$)
    \item Будем записывать коммутатор веса $n$ следующим образом $[l_1, \ldots, l_n] = [[l_1, l_2, \ldots, l_{n-1}], l_n]$
\end{itemize}
Приведем сначала план доказательства, чтобы была понятна мотивация каждой леммы:
\begin{enumerate}
    \item Определим $G\supseteq G^{(n)}$, вложенную последовательность нормальных подгрупп попадающую в любую окрестность единицы, такую что для любого $g\in G^{(n)}$
    \[
        \min{g} \in \LQn\cap\Sbf = \LZn
    \]
    \item Таким образом, можно изучать $G^{(n)}/G^{(n+1)}$ как $\Z_p$-модуль $\LZn$.
    Докажем, что $\rank_{\Z_p}{\LZn} = f(n)$ для некоторой $f$.
    \item Доказав, что $G^{(n)}$ ~--- нижний центральный ряд $G$ получим противоречие с формулой Э.\ Витта (см. \cite{Lubotzky}):~\begin{proposition}[Витт, \cite{Lubotzky}]
                                                                                                                                      \label{thm:Vitt}
                                                                                                                                      Пусть F ~--- свободная про-$p$ группа порожденная $m$ образующими, тогда $n$-ый фактор нижнего центрального ряда имеет ранг (как $\Z_p$-модуль)
                                                                                                                                      \[
                                                                                                                                          \frac{1}{n}\sum\limits_{d\mid n} \mu(d) m^{n / d}
                                                                                                                                      \]
                                                                                                                                      где $\mu$ ~--- функция Мебиуса.
    \end{proposition}
\end{enumerate}
Итак, докажем следующее техническое утверждение:
\vskip 0.1in\noindent
\begin{proposition}
    \label{thm:LQn-to-LZn}
    При $p\neq 2$:
    \[
        \LQn \cap \Sbf = \LZn
    \]
\end{proposition}
\begin{remark}
    Данное предложение не верно для $p=2$, что влечет существенные сложности, возникающие для $p=2$.
\end{remark}
\begin{proof}
    Пусть
    \begin{align*}
        & a = 4\xod\xdo + (\xoo - \xdd)^2 \\
        & b = 2(\yod\xdo + \xod\ydo) + (\xoo - \xdd)(\yoo-\ydd)\\
        & c = 4\yod\ydo + (\yoo - \ydd)^2
    \end{align*}
    Легко проверить, что
    \begin{align*}
        & [x,y,x,x] = a[x,y] \\
        & [x, y, y, x] = b[x,y] \\
        & [x,y,y,y] = c[x,y] \\
        & [x,y,x,y] = [x,y,y,x]
    \end{align*}
    Получаем, что любой $l\in \LQn$ имеет вид
    \begin{equation}
        l =
        \begin{cases}
            \sum\limits_{i_a + i_b + i_c = (n - 2) / 2} \lambda_{i_a,i_b,i_c} a^{i_a}b^{i_b}c^{i_c}[x,y], & \text{если $n$ четно} \\
            \sum\limits_{i_a + i_b + i_c = (n - 3) / 2} \alpha_{i_a,i_b,i_c} a^{i_a}b^{i_b}c^{i_c}[x,y,x] +
            \beta_{i_a,i_b,i_c} a^{i_a}b^{i_b}c^{i_c}[x,y,y], & \text{если $n$ нечетно}
        \end{cases}\\
        \label{eq:not-direct-sum}
    \end{equation}
    где $\lambda_{i_a,i_b,i_c},\alpha_{i_a,i_b,i_c},\beta_{i_a,i_b,i_c}\in\Q_p$

    Разберем случай нечетного $n$.
    Рассмотрим какое-то $l\in\LQn \cap \Sbf$ с нецелыми $p$-адическими $\alpha_{i_a,i_b,i_c},\beta_{i_a,i_b,i_c}$.

    Далее рассуждение аналогично классическому доказательству теоремы Гильберта о базисе.
    Введем лексикографический порядок на мономах порожденный отношением
    \[
        \xod>\xoo>\yod>\yoo>\xdo>\xdd>\ydo>\ydd
    \]
    Старшие члены у $a,b,c$: $4\xod\xdo, 2\xod\ydo, 4\yod\ydo$ соответственно.
    Тогда старший член элемента, стоящего в левом верхнем углу $l$ равен старшему члену выражения
    \[
        \sum\limits_{i_a + i_b + i_c = (n - 3) / 2}  2^{2i_a + 2i_c + i_b}
        \left(
        \alpha_{i_a,i_b,i_c}
        \xoo\yod^{i_c}\xdo^{i_a}\ydo^{i_b+i_c+1} +
        \beta_{i_a,i_b,i_c}
        \yod^{i_c}\yoo\xdo^{i_a}\ydo^{i_b+i_c+1}\right)
    \]
    Старшие члены различны и $2^{k}$ обратим в кольце $\Z_p$ (при $p\neq 2$!).

    Следовательно, так как $\alpha_{i_a,i_b,i_c}, \beta_{i_a,i_b,i_c} \in \Q_p\setminus \Z_p$, а значит старший член левого верхнего угла $l$ не лежит в $\Z_p$.
    Получаем противоречие с тем, что $l\in \Sbf$.

    Случай четного $n$ разбирается аналогично.
    \begin{remark}
        Попутно мы доказали, что суммы в формуле~\eqref{eq:not-direct-sum} на самом деле прямые.
    \end{remark}
\end{proof}

Сразу же получаем следствие
\begin{colloraly}
    \label{thm:LZn-rank}
    \[
        rank_{\Z_p} \LZn = \dim_{\Q_p} \LQn =
        \begin{cases}
            \frac{n(n+2)}{8}, & \text{при четном $n$} \\
            \frac{(n-1)(n+1)}{4}, & \text{при нечетном $n$}
        \end{cases}
    \]
\end{colloraly}


Итак, пусть
\[
    G^{(n)} = G \cap \ker{(GL(S) \to GL(S / S_n))}
\]

Следующая лемма является ключевой.
\begin{lemma}
    Пусть $g\in G^{(n)}$, тогда
    \[
        \min g \in \LZn
    \]
\end{lemma}
\begin{proof}
    В силу предложения~\ref{thm:LQn-to-LZn} достаточно доказать, что
    $\min g \in \LQn$.


\end{proof}

Пусть $G=G_1,\ldots,G_n,\ldots$ ~--- нижний центральный ряд.

Следующее предложение завершает доказательство теоремы~\ref{thm:Zubkov-main}
\begin{proposition}
    Нижний центральный ряд совпадает с пересечениями группы $G$ с конгруенц-подгруппами по идеалу $S_n$, то есть
    \[
        G_n = G^{(n)}
    \]
\end{proposition}
\begin{proof}
    Ясно, что $G_n \subseteq G^{(n)}$, так как коммутаторы веса $n$ лежат в $\ker{(GL(S) \to GL(S / S_n))}$ по определению $S_n$.\\
    С другой стороны, пусть $g\in G^{(n)}$:
    \[
        g = 1 + v_n(X, Y) + \text{старшие члены}
    \]
    где $v_n(X,Y)$ ~--- линейная комбинация коммутаторов веса $n$ над $\Z_p$.

\end{proof}
