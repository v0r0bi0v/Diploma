В данном параграфе мы приводим план исследования следующих проблем.

\begin{conjecture}
    \label{conj:4}
    Некоммутативная свободная про-$2$ группа не может быть непрерывно вложена в $GL^1_2(\Delta)$ для про-$2$ кольца $\Delta$ характеристики $4$.
\end{conjecture}
\begin{conjecture}
    \label{conj:2^n}
    Некоммутативная свободная про-$2$ группа не может быть непрерывно вложена в $GL^1_2(\Delta)$ для про-$2$ кольца $\Delta$ характеристики $2^n$.
\end{conjecture}
\begin{conjecture}
    \label{conj:2_n}
    Некоммутативная свободная про-$2$ группа не может быть непрерывно вложена в $GL^1_2(\Delta)$ ни для какого для про-$2$ кольца $\Delta$.
\end{conjecture}
\vskip 0.1in\noindent

Мы все так же, как и раньше можем изучать каждую гипотезу, ограничившись соответствующим ей универсальным представлением:

\vskip 0.1in\noindent
\begin{proposition}
    Если существует непрерывное вложение некоммутативной свободной про-$2$ группы в $GL_2^1(\Delta)$ для какого-то про-$2$ кольца $\Delta$ характеристики $4$, то и универсальное представление в про-$2$ группу, порожденную (как раньше) общими матрицами над кольцом $\Z/4\Z$ инъективно.
\end{proposition}
\begin{proposition}
    Если существует непрерывное вложение некоммутативной свободной про-$2$ группы в $GL_2^1(\Delta)$ для какого-то про-$2$ кольца $\Delta$ характеристики $2^n$, то и универсальное представление в про-$2$ группу, порожденную общими матрицами над кольцом $\Z/2^n\Z$ инъективно.
\end{proposition}
\begin{proposition}
    Если существует непрерывное вложение некоммутативной свободной про-$2$ группы в $GL_2^1(\Delta)$ для какого-то про-$2$ кольца $\Delta$, то и универсальное представление в про-$2$ группу, порожденную общими матрицами над кольцом $\Z_2$ инъективно.
\end{proposition}
\vskip 0.1in\noindent

Доказательства этих предложений дословно переносятся из параграфов~\ref{subsec:zubkov-universal}, ~\ref{subsec:ben-ezra-zelmanov-universal}.\\
Есть основания полагать, что обобщив вычисления Бена-Эзры\textemdash Зельманова на случай характеристики $4$, гипотезу~\ref{conj:2^n} можно будет доказать аналогичным обобщением.

Сконцентрируемся сейчас на гомоморфизме из про-$2$ группы в замыкание (в уже известной нам топологии) группы
\[\langle 1+X, 1+Y \rangle\]
где $X,Y$ ~--- общие матрицы над $\Z_4$.

Вероятно, подход Бена-Эзры\textemdash Зельманова изучения алгебры Ли общих матриц над $\Z_2$ можно обобщить до алгебры Ли общих матриц над $\Z_{4}$ (авторы пишут~\cite{Ben-Ezra-Zelmanov}, что гипотеза~\ref{conj:4} вполне может быть исследована) следующим образом.
Мы повторим все вычисления, учитывая дополнительную градуировку основного кольца: нужно будет в отдельности рассматривать члены делящиеся на 2 и

%\subsection{Универсальное представление}\label{subsec:char-4-universal}
%\subsection{Набросок доказательства}\label{subsec:char-4-non-injective}