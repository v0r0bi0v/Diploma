Д.\ Бен-Эзра и Е.И.\ Зельманов доказали следующую теорему
\begin{theorem}[А.Н.\ Зубков, 1989]
    \label{thm:ben-ezra-zelmanov-main}
    Некоммутативная свободная про-$2$ группа не может быть непрерывно вложена в $GL^1_2(\Delta)$ для про-$2$ кольца $\Delta$ характеристики $2$.
\end{theorem}
Авторы начинают доказательство с все той же идеи универсального объекта.

\subsection{Универсальное представление}\label{subsec:ben-ezra-zelmanov-universal}
Для построения кольца общих матриц авторы взяли основное кольцо $\Z/2\Z$ вместо $2$-адических чисел, как это было у Зубкова.
Введем аналогичные~\ref{subsec:zubkov-universal} обозначения.

\begin{itemize}
    \item $S = \Z/2\Z [[ x_{1,1}, y_{1,1}, \ldots, x_{2,2}, y_{2,2} ]]$
    \item $S_k = \{\sum\limits_k^{\infty} f_i \}$, где $f_i\in S$ однородные многочлены степени $i$.
    \item Наделив $S$ аналогичной топологией и рассмотрим про-$2$ группу $1 + \ker{(M_2(S) \to M_2(S / S_1))}$
    \item Опять же
    \[
        X=
        \begin{pmatrix}
            x_{1,1} & x_{1,2} \\
            x_{2,1} & x_{2,2}
        \end{pmatrix},
        \quad
        Y=
        \begin{pmatrix}
            y_{1,1} & y_{1,2} \\
            y_{2,1} & y_{2,2}
        \end{pmatrix}
    \]
    \item $F$ ~--- свободная про-$p$ группа порожденная $x,y$
    \item Определим универсальное представление $\pi: x \mapsto 1 + X, y \mapsto 1 + Y$ и, как и раньше, продолжим его на всю $F$, образ будет про-$p$ подгруппой $G\subseteq 1 + \ker{(M_2(S) \to M_2(S / S_1))}$.
\end{itemize}

\begin{theorem}
    Пусть $F$ ~--- свободная про-$2$ группа порожденная $x, y$.
    Если существует инъективный непрерывный гомоморфизм $\varphi: F \to GL^1_2(\Delta)$ для про-$2$ кольца $\Delta$ характеристики 2, то и универсальное представление $\pi$ инъективно.
\end{theorem}
\begin{proof}
    Доказательство аналогично теореме\ref{thm:zubkov-universal} основано на универсальности общих матриц.

    Пусть $\sigma: F\to GL_2^1(\Delta)$.
    \[
        \sigma (x) \mapsto 1 +
        \begin{pmatrix}
            a_{1,1} & a_{1,2} \\
            a_{2,1} & a_{2,2}
        \end{pmatrix}, \quad
        \sigma (y) \mapsto 1 +
        \begin{pmatrix}
            b_{1,1} & b_{1,2} \\
            b_{2,1} & b_{2,2}
        \end{pmatrix},
    \]
    Тогда, пользуясь тем, что $\char \Delta = 2$ определим, отображение $\zeta: x_{i,j}\mapsto a_{i,j}$, которое все так же индуцирует эпиморфизм $\hat{\zeta}: G \to \mathrm{Im}\hspace{0.1cm}{\varphi}$.
    И диаграмма коммутативна
    \begin{center}
        \begin{tikzcd}
            & F\arrow{dl}{\pi} \arrow{dr}{\varphi} & \\
            G\arrow{rr}{\hat{\zeta}} & & GL_2^1(\Delta)
        \end{tikzcd}
    \end{center}
    Следовательно:
    \[
        \ker{\pi} \subseteq \ker{\varphi} \qedhere
    \]
\end{proof}

\subsection{Набросок доказательства}\label{subsec:ben-ezra-zelmanov-non-injective}
