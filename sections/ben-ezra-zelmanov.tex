Д.\ Бен-Эзра и Е.И.\ Зельманов доказали следующую теорему
\begin{theorem}[Д.\ Бен-Эзра, Е.И.\ Зельманов, 2020]
    \label{thm:ben-ezra-zelmanov-main}
    Некоммутативная свободная про-$2$ группа не может быть непрерывно вложена в $GL^1_2(\Delta)$ для про-$2$ кольца $\Delta$ характеристики $2$.
\end{theorem}
Авторы начинают доказательство с все той же идеи универсального объекта.

\subsection{Универсальное представление}\label{subsec:ben-ezra-zelmanov-universal}
Для построения кольца общих матриц авторы взяли основное кольцо $\Z/2\Z$ вместо $2$-адических чисел, как это было у Зубкова.
Введем аналогичные~\ref{subsec:zubkov-universal} обозначения.

\begin{itemize}
    \item $S = \Z/2\Z [[ x_{1,1}, y_{1,1}, \ldots, x_{2,2}, y_{2,2} ]]$
    \item $S_k = \{\sum\limits_k^{\infty} f_i \}$, где $f_i\in S$ однородные многочлены степени $i$.
    \item Наделив $S$ аналогичной топологией и рассмотрим про-$2$ группу $1 + \ker{(M_2(S) \to M_2(S / S_1))}$
    \item Опять же
    \[
        X=
        \begin{pmatrix}
            x_{1,1} & x_{1,2} \\
            x_{2,1} & x_{2,2}
        \end{pmatrix},
        \quad
        Y=
        \begin{pmatrix}
            y_{1,1} & y_{1,2} \\
            y_{2,1} & y_{2,2}
        \end{pmatrix}
    \]
    \item $F$ ~--- свободная про-$p$ группа порожденная $x,y$
    \item Определим универсальное представление $\pi: x \mapsto 1 + X, y \mapsto 1 + Y$ и, как и раньше, продолжим его на всю $F$, образ будет про-$p$ подгруппой $G\subseteq 1 + \ker{(M_2(S) \to M_2(S / S_1))}$.
\end{itemize}

\begin{theorem}
    Пусть $F$ ~--- свободная про-$2$ группа порожденная $x, y$.
    Если существует инъективный непрерывный гомоморфизм $\varphi: F \to GL^1_2(\Delta)$ для про-$2$ кольца $\Delta$ характеристики 2, то и универсальное представление $\pi$ инъективно.
\end{theorem}
\begin{proof}
    Доказательство аналогично теореме\ref{thm:zubkov-universal} основано на универсальности общих матриц.

    Пусть $\sigma: F\to GL_2^1(\Delta)$.
    \[
        \sigma (x) \mapsto 1 +
        \begin{pmatrix}
            a_{1,1} & a_{1,2} \\
            a_{2,1} & a_{2,2}
        \end{pmatrix}, \quad
        \sigma (y) \mapsto 1 +
        \begin{pmatrix}
            b_{1,1} & b_{1,2} \\
            b_{2,1} & b_{2,2}
        \end{pmatrix},
    \]
    Тогда, пользуясь тем, что $\char \Delta = 2$ определим, отображение $\zeta: x_{i,j}\mapsto a_{i,j}$, которое все так же индуцирует эпиморфизм $\hat{\zeta}: G \to \mathrm{Im}\hspace{0.1cm}{\varphi}$.
    И диаграмма коммутативна
    \begin{center}
        \begin{tikzcd}
            & F\arrow{dl}{\pi} \arrow{dr}{\varphi} & \\
            G\arrow{rr}{\hat{\zeta}} & & GL_2^1(\Delta)
        \end{tikzcd}
    \end{center}
    Следовательно:
    \[
        \ker{\pi} \subseteq \ker{\varphi} \qedhere
    \]
\end{proof}

\subsection{Разница со случаем $p>2$}\label{subsec:ben-ezra-zelmanov-difference}
В основе доказательства А.Н.\ Зубкова лежит предложение~\ref{thm:LQn-to-LZn} о том,
если элемент алгебры Ли оказался суммой мономов с целыми $p$-адическими коэффициентами, то он лежит и в алгебре Ли с целыми $p$-адическими коэффициентами.\\
Из этого предложения удается доказать, что нахождение элемента $g$ в нижнем центральном ряде группы соответствует тому, что $\min g$ представляется в виде линейной комбинации коммутаторов соответсвующего веса.
Однако для центрального ряда свободной про-$p$ группы существует замечательная формула Витта, с которой получаем противоречие.

Д.\ Бен-Эзра и Е.И.\ Зельманов в приложении своей работы показали, почему подход А.Н.\ Зубкова напрямую нельзя обобщить на случай $p=2$: в некотором смысле про-$2$ группа общих матриц ближе к тому, чтобы быть свободной:
\begin{proposition}[Д. Бен-Эзра, Е.И.\ Зельманов, 2020]
    \label{prp:ben-ezra-zelmanov-difficulty}
    В обозначениях предыдущего параграфа для $p=2$, $G_6 / G_7$ ~--- абелева группа порожденная не менее, чем $9$ образующими.
\end{proposition}
Заметим, что формула Витта как раз дает значение $9$.
Для контраста еще раз приведем предложение, доказанное А.Н.\ Зубковым, оно противоречит формуле Витта для $n=6$:
\begin{proposition}
    При $p>2$ $G_n / G_{n+1} \cong \Z_p^{f(n)}$, где\\
    \[
        f(n) =
        \begin{cases}
            p^{\frac{n(n+2)}{8}}, & \text{при четном $n$} \\
            p^{\frac{(n-1)(n+1)}{4}}, & \text{при нечетном $n$}
        \end{cases}
    \]
\end{proposition}

В дополнение авторы приводят полное описание $\LQn \cap \Sbf$ для $p=2$, которым однако не пользуются в основном тексте, а доказывают через него предложение~\ref{prp:ben-ezra-zelmanov-difficulty}.

\subsection{Набросок доказательства}\label{subsec:ben-ezra-zelmanov-non-injective}
В этом параграфе мы приведем план доказательства Бена-Эзры\textemdash Зельманова ~--- полное доказательство содержит более 20-и страниц вычислений, так что мы постараемся привести основные идеи, опуская вычисления.

\vskip 0.1in\noindent
{\large\textbf{Определение псевдо-общих матриц.}}

Авторы вводят новые матрицы $\tilde{X}, \tilde{Y}$, такие что $\det \tilde{X} = \det \tilde{Y} = 0$, называемые псевдо-общими (pseudo generic matrices), аналогичная идея встречается и в PI-теории.
Эти матрицы определяются следующим образом.

Введем квадратные расширения
\begin{gather*}
    \mu^2 = (1 + \trace(X) + \det(X))^{-1} \cdot (\mu \cdot \trace(X) + \det(X))\\
    \nu^2 = (1 + \trace(Y) + \det(Y))^{-1} \cdot (\nu \cdot \trace(Y) + \det(Y))
\end{gather*}

Легко показать, что эти элементы независимы.
Далее авторы работают в про-$2$ кольце
\[
    \tilde{S} = S + \mu S + \nu S + \mu \nu S
\]
Наконец псевдо-общие матрицы
\begin{gather*}
    \tilde{X} = (1 + X) \cdot (1 + \mu) - 1 \\
    \tilde{Y} = (1 + Y) \cdot (1 + \nu) - 1
\end{gather*}

Пусть $\tilde{G}$ группа топологически порожденная $1 + \tilde{X}, 1 + \tilde{Y}$.
Несложно показать, что $[G, G] = [\tilde{G}, \tilde{G}]$ и из этого следует, что можно доказывать, что достаточно доказать следующую теорему:
\vskip 0.1in\noindent
\begin{theorem}
    Определим гомоморфизм $\tilde{\pi}$ из $F$ в $\tilde{G}$: $x\mapsto 1 + \tilde{X}, y\mapsto 1 + \tilde{Y}$, тогда ограничение $\tilde{\pi}: [F, F] \to [\tilde{G}, \tilde{G}]$ не инъективно.
\end{theorem}

Мы пропустим часть с выводом некоторых достаточно тривиальных свойств матриц $2\times 2$, следующих из теоремы Гамильтона-Кэли.
Отметим, что хорошо известно, что квадрат коммутатора $\tilde{X}, \tilde{Y}$ лежит в центре, является скалярной матрицей.
Поэтому иногда будет воспринимать $[\tilde{X},\tilde{Y}]^2$ как элемент $\tilde{S}$.
И наоборот, если пишем, что элемент $\tilde{S}$ лежит в $M_2(\tilde{S})$, то имеем в виду элементы вида $\tilde{S} \cdot
\begin{pmatrix}
    1 & 0 \\
    0 & 1
\end{pmatrix}$.

\vskip 0.1in\noindent
{\large\textbf{Кольцо псевдо-общих матриц.}}

Введем дискретные кольца
\begin{align*}
    & S = \langle \trace(\tilde{X}), \trace(\tilde{Y}), [\tilde{X}, \tilde{Y}]^2 \rangle \subseteq \tilde{S} \\
    & T = \langle \trace(\tilde{X}), \trace(\tilde{Y}), \trace(\tilde{X}\tilde{Y}) \rangle \subseteq \tilde{S} \\
    & R = \langle \tilde{X}, \tilde{Y}, T \rangle
\end{align*}
Обозначим их замыкания $\hat{S}, \hat{T}, \hat{R}$ соответственно.

Далее авторы исследуют некоторые свойства этих модулей.
Приведем их списком без доказательств, отметив, что используемые техники вполне элементарны и что комбинаторика, связанная со следами изучалась в PI-теории.

\begin{proposition}
    Алгебра $T$ ~--- свободная коммутативная $\Z/2\Z$-алгебра порожденная $\trace(\tilde{X}), \trace(\tilde{Y}), \trace(\tilde{X}\tilde{Y})$.
\end{proposition}
\begin{proposition}\\
    \begin{enumerate}
        \item Алгебра $S$ ~--- свободная коммутативная $\Z/2\Z$-алгебра порожденная $\trace(\tilde{X}), \trace(\tilde{Y}), \trace(\tilde{X}\tilde{Y})$
        \item $S\supseteq T$
        \item $T$ ~--- свободный  $S$-модуль порожденный $1, \trace(\tilde{X}\tilde{Y})$.
    \end{enumerate}
\end{proposition}
\begin{proposition}
    Кольцо $R$ ~--- свободный $T$-модуль порожденный $1, \tilde{X}, \tilde{y}, \tilde{x}\tilde{y}$
\end{proposition}
Вернувшись к общим матрицам, рассмотрим алгебру
\[
    \langle 1, \trace(X), \trace(Y), \trace(XY) \rangle
\]
и заметим, что она снабжена градуировкой однородных компонент.
Построив гомоморфизм из нее в $R$, определенный $X\mapsto \tilde{X}, Y\mapsto\tilde{Y}$.
Он индуцирует градуировку на $R$:
\begin{proposition}
    Можно ввести градуировку на $R$:
    \[
        R = \bigoplus\limits_{n=0}^{\infty} R^{(n)}
    \]
\end{proposition}

Исследуя кольцо общих матриц полезным оказывается рассмотреть следующий $T$-модуль порожденный
\[
    [\tilde{X}, \tilde{Y}] \tilde{X}, \quad [\tilde{X}, \tilde{Y}] \tilde{Y}, \quad [\tilde{X}, \tilde{Y}]^2 \quad [\tilde{X}, \tilde{Y}]\tilde{X}\tilde{Y}
\]
Обозначим его через $J$.

\begin{proposition}\\
    \begin{enumerate}
        \item Модуль $J$ ~--- двусторонний идеал кольца $R$.
        \item $J$ ~--- свободный $T$-модуль порожденный \[[\tilde{X}, \tilde{Y}] \tilde{X}, \quad [\tilde{X}, \tilde{Y}] \tilde{Y}, \quad [\tilde{X}, \tilde{Y}]^2 \quad [\tilde{X}, \tilde{Y}]\tilde{X}\tilde{Y}\]
    \end{enumerate}
\end{proposition}
Обозначим $\hat{J}$ ~--- замыкание $J$ в топологии градуировки $R$.
Определив
\begin{align*}
    & \mathring{S} = \langle \trace(\tilde{Y}), [\tilde{X}, \tilde{Y}]^2 \rangle \\
    & \mathring{T} = \mathring{S} + \trace(\tilde{X}\tilde{Y}) \mathring{S}
\end{align*}
авторы получают следующее описание $\hat{J}$:
\begin{proposition}
    Каждый элемент $\hat{J}$ единственным образом представляется в виде
    \[
        \sum\limits_{n=0}^{\infty}\sum\limits_{i=0}^{\infty}
        u_{n,i} + v_{n,i} + w_{n,i}
    \]
    где
    \begin{align*}
        & u_{n,i} \in (\trace(\tilde{X}))^n \cdot (\mathring{T}\cdot[\tilde{X}, \tilde{Y}]\tilde{X} + \mathring{T}\cdot[\tilde{X}, \tilde{Y}]\tilde{Y}) \\
        & v_{n,i} \in (\trace(\tilde{X}))^n \cdot \mathring{T} \cdot [\tilde{X}, \tilde{Y}]^2 \\
        & w_{n,i} \in (\trace(\tilde{X}))^n \cdot \mathring{T} \cdot [\tilde{X}, \tilde{Y}]\cdot \tilde{X} \cdot \tilde{Y}
    \end{align*}
\end{proposition}

Следующая лемма мотивирует все предыдущие предложения:
